% !TeX program = XeLaTeX
% !TeX root = AdhyatmaRamayanaBook-kindle.tex
\chapt{अरण्यकाण्डः}


\sect{प्रथमः सर्गः}

\uvacha{श्रीमहादेव उवाच}

\twolineshloka
{अथ तत्र दिनं स्थित्वा प्रभाते रघुनन्दनः}
{स्नात्वा मुनिं समामन्त्र्य प्रयाणायोपचक्रमे} %1-1

\twolineshloka
{मुने गच्छामहे सर्वे मुनिमण्डलमण्डितम्}
{विपिनं दण्डकं यत्र त्वमाज्ञातुमिहार्हसि} %1-2

\threelineshloka
{मार्गप्रदर्शनार्थाय शिष्यानाज्ञप्तुमर्हसि}
{श्रुत्वा रामस्य वचनं प्रहस्यात्रिर्महायशाः}
{प्राह तत्र रघुश्रेष्ठं राम राम सुराश्रय} %1-3

\twolineshloka
{सर्वस्य मार्गद्रष्टा त्वं तव को मार्गदर्शकः}
{तथाऽपि दर्शयिष्यन्ति तव लोकानुसारिणः} %1-4

\twolineshloka
{इति शिष्यान् समादिश्य स्वयं किञ्चित्तमन्वगात्}
{रामेण वारितः प्रीत्या अत्रिः स्वभवनं ययौ} %1-5

\twolineshloka
{क्रोशमात्रं ततो गत्वा ददर्श महतीं नदीम्}
{अत्रेः शिष्यानुवाचेदं रामो राजीवलोचनः} %1-6

\twolineshloka
{नद्याः सन्तरणे कश्चिदुपायो विद्यते न वा}
{ऊचुस्ते विद्यते नौका सुदृढा रघुनन्दन} %1-7

\twolineshloka
{तारयिष्यामहे युष्मान् वयमेव क्षणादिह}
{ततो नावि समारोप्य सीतां राघवलक्ष्मणौ} %1-8

\twolineshloka
{क्षणात्सन्तारयामासुर्नदीं मुनिकुमारकाः}
{रामाभिनन्दिताः सर्वे जग्मुरत्रेरथाश्रमम्} %1-9

\twolineshloka
{तावेत्य विपिनं घोरं झिल्लीझङ्कारनादितम्}
{नानामृगगणाकीर्णं सिंहव्याघ्रादिभीषणम्} %1-10

\twolineshloka
{राक्षसैर्घोररूपैश्च सेवितं रोमहर्षणम्}
{प्रविश्य विपिनं घोरं रामो लक्ष्मणमब्रवीत्} %1-11

\twolineshloka
{इतः परं प्रयत्नेन गन्तव्यं सहितेन मे}
{धनुर्गुणेन संयोज्य शरानपि करे दधत्} %1-12

\twolineshloka
{अग्रे यास्याम्यहं पश्चात्त्वमन्वेहि धनुर्धर}
{आवयोर्मध्यगा सीता मायेवाऽऽत्मपरात्मनोः} %1-13

\twolineshloka
{चक्षुश्चारय सर्वत्र दृष्टं रक्षोभयं महत्}
{विद्यते दण्डकारण्ये श्रुतपूर्वमरिन्दम} %1-14

\twolineshloka
{इत्येवं भाषमाणौ तौ जग्मतुः सार्धयोजनम्}
{तत्रैका पुष्करिण्यास्ते कल्हारकुमुदोत्पलैः} %1-15

\twolineshloka
{अम्बुजैः शीतलोदेन शोभमाना व्यदृश्यत}
{तत्समीपमथो गत्वा पीत्वा तत्सलिलं शुभम्} %1-16

\twolineshloka
{ऊषुस्ते सलिलाभ्याशे क्षणं छायामुपाश्रिताः}
{ततो ददृशुरायान्तं महासत्त्वं भयानकम्} %1-17

\twolineshloka
{करालदंष्ट्रवदनं भीषयन्तं स्वगर्जितैः}
{वामांसे न्यस्तशूलाग्रग्रथितानेकमानुषम्} %1-18

\twolineshloka
{भक्षयन्तं गजव्याघ्रमहिषं वनगोचरम्}
{ज्यारोपितं धनुर्धृत्वा रामो लक्ष्मणमब्रवीत्} %1-19

\twolineshloka
{पश्य भ्रातर्महाकायो राक्षसोऽयमुपागतः}
{आयात्यभिमुखं नोऽग्रे भीरूणां भयमावहन्} %1-20

\twolineshloka
{सज्जीकृतधनुस्तिष्ठ मा भैर्जनकनन्दिनि}
{इत्युक्त्वा बाणमादाय स्थितो राम इवाचलः} %1-21

\twolineshloka
{स तु दृष्ट्वा रमानाथं लक्ष्मणं जानकीं तदा}
{अट्टहासं ततः कृत्वा भीषयन्निदमब्रवीत्} %1-22

\twolineshloka
{कौ युवां बाणतूणीरजटावल्कलधारिणौ}
{मुनिवेषधरौ बालौ स्त्रीसहायौ सुदुर्मदौ} %1-23

\twolineshloka
{सुन्दरौ बत मे वक्त्रप्रविष्टकवलोपमौ}
{किमर्थमागतौ घोरं वनं व्यालनिषेवितम्} %1-24

\twolineshloka
{श्रुत्वा रक्षोवचो रामः स्मयमान उवाच तम्}
{अहं रामस्त्वयं भ्राता लक्ष्मणो मम सम्मतः} %1-25

\twolineshloka
{एषा सीता मम प्राणवल्लभा वयमागताः}
{पितृवाक्यं पुरस्कृत्य शिक्षणार्थं भवादृशाम्} %1-26

\twolineshloka
{शृत्वा तद्रामवचनमट्टहासमथाकरोत्}
{व्यादाय वक्त्रं बाहुभ्यां शूलमादाय सत्वरः} %1-27

\twolineshloka
{मां न जानासि राम त्वं विराधं लोकविश्रुतम्}
{मद्भयान्मुनयः सर्वे त्यक्त्वा वनमितो गताः} %1-28

\twolineshloka
{यदि जीवितुमिच्छास्ति त्यक्त्वा सीतां निरायुधौ}
{पलायत न चेच्छीघ्रं भक्षयामि युवामहम्} %1-29

\twolineshloka
{इत्युक्त्वा राक्षसः सीतामादातुमभिदुद्रुवे}
{रामश्चिच्छेद तद्बाहू शरेण प्रहसन्निव} %1-30

\twolineshloka
{ततः क्रोधपरीतात्मा व्यादाय विकटं मुखम्}
{राममभ्यद्रवद्रामश्चिच्छेद परिधावतः} %1-31

\onelineshloka
{पदद्वयं विराधस्य तदद्भुतमिवाभवत्} %1-32


\twolineshloka
{ततः सर्प इवास्येन ग्रसितुं राममापतत्}
{ततोऽर्धचन्द्राकारेण बाणेनास्य महच्छिरः} %1-33

\twolineshloka
{चिच्छेद रुधिरौघेण पपात धरणीतले}
{ततः सीता समालिङ्ग्य प्रशशंस रघूत्तमम्} %1-34

\twolineshloka
{ततो दुन्दुभयो नेदुर्दिवि देवगणेरिताः}
{ननृतुश्चाप्सरा हृष्टा जगुर्गन्धर्वकिन्नराः} %1-35

\fourlineindentedshloka
{विराधकायादतिसुन्दराकृतिः}
{विभ्राजमानो विमलाम्बरावृतः}
{प्रतप्तचामीकरचारुभूषणो}
{व्यदृश्यताग्रे गगने रविर्यथा} %1-36

\fourlineindentedshloka
{प्रणम्य रामं प्रणतार्तिहारिणम्}
{भवप्रवाहोपरमं घृणाकरम्}
{प्रणम्य भूयः प्रणनाम दण्डवतः}
{प्रपन्नसर्वार्तिहरं प्रसन्नधीः} %1-37

\uvacha{विराध उवाच}

\fourlineindentedshloka
{श्रीराम राजीवदलायताक्ष}
{विद्याधरोऽहं विमलप्रकाशः}
{दुर्वाससाकारणकोपमूर्तिना}
{शप्तः पुरा सोऽद्य विमोचितस्त्वया} %1-38

\fourlineindentedshloka
{इतः परं त्वच्चरणारविन्दयोः}
{स्मृतिः सदा मेऽस्तु भवोपशान्तये}
{त्वन्नामसङ्कीर्तनमेव वाणी}
{करोतु मे कर्णपुटं त्वदीयम्} %1-39

\fourlineindentedshloka
{कथामृतं पातु करद्वयं ते}
{पादारविन्दार्चनमेव कुर्यात्}
{शिरश्च ते पादयुगप्रणामम्}
{करोतु नित्यं भवदीयमेवम्} %1-40

\twolineshloka
{नमस्तुभ्यं भगवते विशुद्धज्ञानमूर्तये}
{आत्मारामाय रामाय सीतारामाय वेधसे} %1-41

\twolineshloka
{प्रपन्नं पाहि मां राम यास्यामि त्वदनुज्ञया}
{देवलोकं रघुश्रेष्ठ माया मां मा वृणोतु ते} %1-42

\twolineshloka
{इति विज्ञापितस्तेन प्रसन्नो रघुनन्दनः}
{ददौ वरं तदा प्रीतो विराधाय महामतिः} %1-43

\twolineshloka
{गच्छ विद्याधराशेषमायादोषगुणा जिताः}
{त्वया मद्दर्शनात्सद्यो मुक्तो ज्ञानवतां वरः} %1-44

\twolineshloka
{मद्भक्तिर्दुर्लभा लोके जाता चेन्मुक्तिदा यतः}
{अतस्त्वं भक्तिसम्पन्नः परं याहि ममाऽऽज्ञया} %1-45

\fourlineindentedshloka
{रामेण रक्षोनिधनं सुघोरम्}
{शापाद्विमुक्तिर्वरदानमेवम्}
{विद्याधरत्वं पुनरेव लब्धम्}
{रामं गृणन्नेति नरोऽखिलार्थान्} %1-46

{॥इति श्रीमदध्यात्मरामायणे उमामहेश्वरसंवादे
अरण्यकाण्डे प्रथमः सर्गः॥१॥}
%%%%%%%%%%%%%%%%%%%%



\sect{द्वितीयः सर्गः}

\uvacha{श्रीमहादेव उवाच}

\twolineshloka
{विराधे स्वर्गते रामो लक्ष्मणेन च सीतया}
{जगाम शरभङ्गस्य वनं सर्वसुखावहम्} %2-1

\twolineshloka
{शरभङ्गस्ततो दृष्ट्वा रामं सौमित्रिणा सह}
{आयान्तं सीतया सार्धं सम्भ्रमादुत्थितः सुधीः} %2-2

\twolineshloka
{अभिगम्य सुसम्पूज्य विष्टरेषूपवेशयत्}
{आतिथ्यमकरोत्तेषां कन्दमूलफलादिभिः} %2-3

\twolineshloka
{प्रीत्याऽऽह शरभङ्गोऽपि रामं भक्तिपरायणम्}
{बहुकालमिहैवाऽऽसं तपसे कृतनिश्चयः} %2-4

\twolineshloka
{अद्य मत्तपसा सिद्धं यत्पुण्यं बहु विद्यते}
{तत्सर्वं तव दास्यामि ततो मुक्तिं व्रजाम्यहम्} %2-5

\fourlineindentedshloka
{समर्प्य रामस्य महत्सुपुण्यफलम्}
{विरक्तः शरभङ्गयोगी}
{चितिं समारोहयदप्रमेयम्}
{रामं ससीतं सहसा प्रणम्य} %2-6

\fourlineindentedshloka
{ध्यायन्श्चिरं राममशेषहृत्स्थम्}
{दूर्वादलश्यामलमम्बुजाक्षम्}
{चीराम्बरं स्निग्धजटाकलापम्}
{सीतासहायं सहलक्ष्मणं तम्} %2-7

\fourlineindentedshloka
{को वा दयालुः स्मृतकामधेनुरन्यो}
{जगत्यां रघुनायकादहो}
{स्मृतो मया नित्यमनन्यभाजा}
{ज्ञात्वा स्मृतिं मे स्वयमेव यातः} %2-8

\twolineshloka
{पश्यत्विदानीं देवेशो रामो दाशरथिः प्रभुः}
{दग्ध्वा स्वदेहं गच्छामि ब्रह्मलोकमकल्मषः} %2-9

\twolineshloka
{अयोध्याधिपतिर्मेऽस्तु हृदये राघवः सदा}
{यद्वामाङ्के स्थिता सीता मेघस्येव तटिल्लता} %2-10

\twolineshloka
{इति रामं चिरं ध्यात्वा दृष्ट्वा च पुरतः स्थितम्}
{प्रज्वाल्य सहसा वह्निं दग्ध्वा पञ्चात्मकं वपुः} %2-11

\threelineshloka
{दिव्यदेहधरः साक्षाद्ययौ लोकपतेः पदम्}
{ततो मुनिगणाः सर्वे दण्डकारण्यवासिनः}
{आजग्मू राघवं द्रष्टुं शरभङ्गनिवेशनम्} %2-12

\twolineshloka
{दृष्ट्वा मुनिसमूहं तं जानकीरामलक्ष्मणाः}
{प्रणेमुः सहसा भूमौ} %2-13

\twolineshloka
{आशीर्भिरभिनन्द्याथ रामं सर्वहृदि स्थितम्}
{ऊचुः प्राञ्जलयः सर्वे धनुर्बाणधरं हरिम्} %2-14

\twolineshloka
{भूमेर्भारावताराय जातोऽसि ब्रह्मणार्थितः}
{जानीमस्त्वां हरिं लक्ष्मीं जानकीं लक्ष्मणं तथा} %2-15

\twolineshloka
{शेषांशं शङ्खचक्रे द्वे भरतं सानुजं तथा}
{अतश्चादौ ऋषीणां त्वं दुःखं मोक्तुमिहार्हसि} %2-16

\fourlineindentedshloka
{आगच्छ यामो मुनिसेवितानि}
{वनानि सर्वाणि रघूत्तम क्रमात्}
{द्रष्टुं सुमित्रासुतजानकीभ्याम्}
{तदा दयाऽस्मासु दृढा भविष्यति} %2-17

\twolineshloka
{इति विज्ञापितो रामः कृताञ्जलिपुटैर्विभुः}
{जगाम मुनिभिः सार्धं द्रष्टुं मुनिवनानि सः} %2-18

\twolineshloka
{ददर्श तत्र पतितान्यनेकानि शिरांसि सः}
{अस्थिभूतानि सर्वत्र रामो वचनमब्रवीत्} %2-19

\twolineshloka
{अस्थीनि केषामेतानि किमर्थं पतितानि वै}
{तमूचुर्मुनयो राम ऋषीणां मस्तकानि हि} %2-20

\twolineshloka
{राक्षसैर्भक्षितानीश प्रमत्तानां समाधितः}
{अन्तरायं मुनीनां ते पश्यन्तोऽनुचरन्ति हि} %2-21

\twolineshloka
{श्रुत्वा वाक्यं मुनीनां स भयदैन्यसमन्वितम्}
{प्रतिज्ञामकरोद्रामो वधायाशेषरक्षसाम्} %2-22

\twolineshloka
{पूज्यमानः सदा तत्र मुनिभिर्वनवासिभिः}
{जानक्या सहितो रामो लक्ष्मणेन समन्वितः} %2-23

\twolineshloka
{उवास कतिचित्तत्र वर्षाणि रघुनन्दनः}
{एवं क्रमेण सम्पश्यन्नृषीणामाश्रमान् विभुः} %2-24

\twolineshloka
{सुतीक्ष्णस्याश्रमं प्रागात्प्रख्यातमृषिसङ्कुलम्}
{सर्वर्तुगुणसम्पन्नं सर्वकालसुखावहम्} %2-25

\threelineshloka
{राममागतमाकर्ण्य सुतीक्ष्णः स्वयमागतः}
{अगस्त्यशिष्यो रामस्य मन्त्रोपासनतत्परः}
{विधिवत्पूजयामास भक्त्युत्कण्ठितलोचनः} %2-26

\uvacha{सुतीक्ष्ण उवाच}

\fourlineindentedshloka
{त्वन्मन्त्रजाप्यहमनन्तगुणाप्रमेय}
{सीतापते शिवविरिञ्चिसमाश्रिताङ्घ्रे}
{संसारसिन्धुतरणामलपोतपाद}
{रामाभिराम सततं तव दासदासः} %2-27

\fourlineindentedshloka
{मामद्य सर्वजगतामविगोचरस्त्वम्}
{त्वन्मायया सुतकलत्रगृहान्धकूपे}
{मग्नं निरीक्ष्य मलपुद्गलपिण्डमोह-}
{पाशानुबद्धहृदयं स्वयमागतोऽसि} %2-28

\fourlineindentedshloka
{त्वं सर्वभूतहृदयेषु कृतालयोऽपि}
{त्वन्मन्त्रजाप्यविमुखेषु तनोषि मायाम्}
{त्वन्मन्त्रसाधनपरेष्वपयाति माया}
{सेवानुरूपफलदोऽसि यथा महीपः} %2-29

\fourlineindentedshloka
{विश्वस्य सृष्टिलयसंस्थितिहेतुरेकः}
{त्वं मायया त्रिगुणया विधिरीशविष्णू}
{भासीश मोहितधियां विविधाकृतिस्त्वम्}
{यद्वद्रविः सलिलपात्रगतो ह्यनेकः} %2-30

\fourlineindentedshloka
{प्रत्यक्षतोऽद्य भवतश्चरणारविन्दम्}
{पश्यामि राम तमसः परतः स्थितस्य}
{दृग्रूपतस्त्वमसतामविगोचरोऽपि}
{त्वन्मन्त्रपूतहृदयेषु सदा प्रसन्नः} %2-31

\fourlineindentedshloka
{पश्यामि राम तव रूपमरूपिणोऽपि}
{मायाविडम्बनकृतं सुमनुष्यवेषम्}
{कन्दर्पकोटिसुभगं कमनीयचाप-}
{बाणं दयार्द्रहृदयं स्मितचारुवक्त्रम्} %2-32

\fourlineindentedshloka
{सीतासमेतमजिनाम्बरमप्रधृष्यम्}
{सौमित्रिणा नियतसेवितपादपद्मम्}
{नीलोत्पलद्युतिमनन्तगुणं प्रशान्तम्}
{मद्भागधेयमनिशं प्रणमामि रामम्} %2-33

\fourlineindentedshloka
{जानन्तु राम तव रूपमशेषदेश-}
{कालाद्युपाधिरहितं घनचित्प्रकाशम्}
{प्रत्यक्षतोऽद्य मम गोचरमेतदेव}
{रूपं विभातु हृदये न परं विकाङ्क्षे} %2-34

\twolineshloka
{इत्येवं स्तुवतस्तस्य रामः सस्मितमब्रवीत्}
{मुने जानामि ते चित्तं निर्मलं मदुपासनात्} %2-35

\twolineshloka
{अतोऽहमागतो द्रष्टुं मदृते नान्यसाधनम्}
{मन्मन्त्रोपासका लोके मामेव शरणं गताः} %2-36

\twolineshloka
{निरपेक्षा नान्यगतास्तेषां दृश्योऽहमन्वहम्}
{स्तोत्रमेतत्पठेद्यस्तु त्वत्कृतं मत्प्रियं सदा} %2-37

\twolineshloka
{सद्भक्तिर्मे भवेत्तस्य ज्ञानं च विमलं भवेत्}
{त्वं ममोपासनादेव विमुक्तोऽसीह सर्वतः} %2-38

\threelineshloka
{देहान्ते मम सायूज्यं लप्स्यसे नात्र संशयः}
{गुरुं ते द्रष्टुमिच्छामि ह्यगस्त्यं मुनिनायकम्}
{किञ्चित्कालं तत्र वस्तुं मनो मे त्वरयत्यलम्} %2-39

\twolineshloka
{सुतीक्ष्णोऽपि तथेत्याह श्वो गमिष्यसि राघव}
{अहमप्यागमिष्यामि चिराद् दृष्टो महामुनिः} %2-40

\fourlineindentedshloka
{अथ प्रभाते मुनिना समेतो}
{रामः ससीतः सह लक्ष्मणेन}
{अगस्त्यसम्भाषणलोलमानसः}
{शनैरगस्त्यानुजमन्दिरं ययौ} %2-41

{॥इति श्रीमदध्यात्मरामायणे उमामहेश्वरसंवादे
अरण्यकाण्डे द्वितीयः सर्गः॥२॥
}
%%%%%%%%%%%%%%%%%%%%



\sect{तृतीयः सर्गः}

\twolineshloka
{अथ रामः सुतीक्ष्णेन जानक्या लक्ष्मणेन च}
{अगस्त्यस्यानुजस्थानं मध्याह्ने समपद्यत} %3-1

\twolineshloka
{तेन सम्पूजितः सम्यग्भुक्त्वा मूलफलादिकम्}
{परेद्युः प्रातरुत्थाय जग्मुस्तेऽगस्त्यमण्डलम्} %3-2

\twolineshloka
{सर्वर्तुफलपुष्पाढ्यं नानामृगगणैर्युतम्}
{पक्षिसङ्घैश्च विविधैर्नादितं नन्दनोपमम्} %3-3

\twolineshloka
{ब्रह्मर्षिभिर्देवर्षिभिः सेवितं मुनिमन्दिरैः}
{सर्वतोऽलङ्कृतं साक्षाद् ब्रह्मलोकमिवापरम्} %3-4

\twolineshloka
{बहिरेवाश्रमस्याथ स्थित्वा रामोऽब्रवीन्मुनिम्}
{सुतीक्ष्ण गच्छ त्वं शीघ्रमागतं मां निवेदय} %3-5

\twolineshloka
{अगस्त्यमुनिवर्याय सीतया लक्ष्मणेन च}
{महाप्रसाद इत्युक्त्वा सुतीक्ष्णः प्रययौ गुरोः} %3-6

\twolineshloka
{आश्रमं त्वरया तत्र ऋषिसङ्घसमावृतम्}
{उपविष्टं रामभक्तैर्विशेषेण समायुतम्} %3-7

\twolineshloka
{व्याख्यातराममन्त्रार्थं शिष्येभ्यश्चातिभक्तितः}
{दृष्ट्वाऽगस्त्यं मुनिश्रेष्ठं सुतीक्ष्णः प्रययौ मुनेः} %3-8

\threelineshloka
{दण्डवत्प्रणिपत्याह विनयावनतः सुधीः}
{रामो दाशरथिर्ब्रह्मन् सीतया लक्ष्मणेन च}
{आगतो दर्शनार्थं ते बहिस्तिष्ठति साञ्जलिः} %3-9

\uvacha{अगस्त्य उवाच}

\twolineshloka
{शीघ्रमानय भद्रं ते रामं मम हृदिस्थितम्}
{तमेव ध्यायमानोऽहं काङ्क्षमाणोऽत्र संस्थितः} %3-10

\twolineshloka
{इत्युक्त्वा स्वयमुत्थाय मुनिभिः सहितो द्रुतम्}
{अभ्यगात्परया भक्त्या गत्वा राममथाब्रवीत्} %3-11

\twolineshloka
{आगच्छ राम भद्रं ते दिष्ट्या तेऽद्य समागमः}
{प्रियातिथिर्मम प्राप्तोऽस्यद्य मे सफलं दिनम्} %3-12

\twolineshloka
{रामोऽपि मुनिमायान्तं दृष्ट्वा हर्षसमाकुलः}
{सीतया लक्ष्मणेनापि दण्डवत्पतितो भुवि} %3-13

\twolineshloka
{द्रुतमुत्थाप्य मुनिराड् राममालिङ्ग्य भक्तितः}
{तद्गात्रस्पर्शजाह्लादस्रवन्नेत्रजलाकुलः} %3-14

\twolineshloka
{गृहीत्वा करमेकेन करेण रघुनन्दनम्}
{जगाम स्वाश्रमं हृष्टो मनसा मुनिपुङ्गवः} %3-15

\twolineshloka
{सुखोपविष्टं सम्पूज्य पूजया बहुविस्तरम्}
{भोजयित्वा यथान्यायं भोज्यैर्वन्यैरनेकधा} %3-16

\twolineshloka
{सुखोपविष्टमेकान्ते रामं शशिनिभाननम्}
{कृताञ्जलिरुवाचेदमगस्त्यो भगवानृषिः} %3-17

\twolineshloka
{त्वदागमनमेवाहं प्रतीक्षन् समवस्थितः}
{यदा क्षीरसमुद्रान्ते ब्रह्मणा प्रार्थितः पुरा} %3-18

\threelineshloka
{भूमेर्भारापनुत्त्यर्थं रावणस्य वधाय च}
{तदादि दर्शनाकाङ्क्षी तव राम तपश्चरन्}
{वसामि मुनिभिः सार्धं त्वामेव परिचिन्तयन्} %3-19

\twolineshloka
{सृष्टेः प्रागेक एवासीर्निर्विकल्पोऽनुपाधिकः}
{त्वदाश्रया त्वद्विषया माया ते शक्तिरुच्यते} %3-20

\twolineshloka
{त्वामेव निर्गुणं शक्तिरावृणोति यदा तदा}
{अव्याकृतमिति प्राहुर्वेदान्तपरिनिष्ठिताः} %3-21

\twolineshloka
{मूलप्रकृतिरित्येके प्राहुर्मायेति केचन}
{अविद्या संसृतिर्बन्ध इत्यादि बहुधोच्यते} %3-22

\twolineshloka
{त्वया सङ्क्षोभ्यमाणा सा महत्तत्त्वं प्रसूयते}
{महत्तत्त्वादहङ्कारस्त्वया सञ्चोदितादभूत्} %3-23

\twolineshloka
{अहङ्कारो महत्तत्त्वसंवृतस्त्रिविधोऽभवत्}
{सात्त्विको राजसश्चैव तामसश्चेति भण्यते} %3-24

\twolineshloka
{तामसात्सूक्ष्मतन्मात्राण्यासन् भूतान्यतः परम्}
{स्थूलानि क्रमशो राम क्रमोत्तरगुणानि ह} %3-25

\twolineshloka
{राजसानीन्द्रियाण्येव सात्त्विका देवता मनः}
{तेभ्योऽभवत्सूत्ररूपं लिङ्गं सर्वगतं महत्} %3-26

\twolineshloka
{ततो विराट् समुत्पन्नः स्थूलाद्भूतकदम्बकात्}
{विराजः पुरुषात्सर्वं जगत्स्थावरजङ्गमम्} %3-27

\twolineshloka
{देवतिर्यङ्मनुष्याश्च कालकर्मक्रमेण तु}
{त्वं रजोगुणतो ब्रह्मा जगतः सर्वकारणम्} %3-28

\twolineshloka
{सत्त्वाद्विष्णुस्त्वमेवास्य पालकः सद्भिरुच्यते}
{लये रुद्रस्त्वमेवास्य त्वन्मायागुणभेदतः} %3-29

\twolineshloka
{जाग्रत्स्वप्नसुषुप्त्याख्या वृत्तयो बुद्धिजैर्गुणैः}
{तासां विलक्षणो राम त्वं साक्षी चिन्मयोऽव्ययः} %3-30

\twolineshloka
{सृष्टिलीलां यदा कर्तुमीहसे रघुनन्दन}
{अङ्गीकरोषि मायां त्वं तदा वै गुणवानिव} %3-31

\threelineshloka
{राम माया द्विधा भाति विद्याऽविद्येति ते सदा}
{प्रवृत्तिमार्गनिरता अविद्यावशवर्तिनः}
{निवृत्तिमार्गनिरता वेदान्तार्थविचारकाः} %3-32

\threelineshloka
{त्वद्भक्तिनिरता ये च ते वै विद्यामयाः स्मृताः}
{अविद्यावशगा ये तु नित्यं संसारिणश्च ते}
{विद्याभ्यासरता ये तु नित्यमुक्तास्त एव हि} %3-33

\twolineshloka
{लोके त्वद्भक्तिनिरतास्त्वन्मन्त्रोपासकाश्च ये}
{विद्या प्रादुर्भवेत्तेषां नेतरेषां कदाचन} %3-34

\twolineshloka
{अतस्त्वद्भक्तिसम्पन्ना मुक्ता एव न संशयः}
{त्वद्भक्त्यमृतहीनानां मोक्षः स्वप्नेऽपि नो भवेत्} %3-35

\twolineshloka
{किं राम बहुनोक्तेन सारं किञ्चिद्ब्रवीमि ते}
{साधुसङ्गतिरेवात्र मोक्षहेतुरुदाहृता} %3-36

\twolineshloka
{साधवः समचित्ता ये निःस्पृहा विगतैषिणः}
{दान्ताः प्रशान्तास्त्वद्भक्ता निवृत्ताखिलकामनाः} %3-37

\twolineshloka
{इष्टप्राप्तिविपत्त्योश्च समाः सङ्गविवर्जिताः}
{सन्न्यस्ताखिलकर्माणः सर्वदा ब्रह्मतत्पराः} %3-38

\twolineshloka
{यमादिगुणसम्पन्नाः सन्तुष्टा येन केनचित्}
{सत्सङ्गमो भवेद्यर्हि त्वत्कथाश्रवणे रतिः} %3-39

\twolineshloka
{समुदेति ततो भक्तिस्त्वयि राम सनातने}
{त्वद्भक्तावुपपन्नायां विज्ञानं विपुलं स्फुटम्} %3-40

\twolineshloka
{उदेति मुक्तिमार्गोऽयमाद्यश्चतुरसेवितः}
{तस्माद्राघव सद्भक्तिस्त्वयि मे प्रेमलक्षणा} %3-41

\twolineshloka
{सदा भूयाद्धरे सङ्गस्त्वद्भक्तेषु विशेषतः}
{अद्य मे सफलं जन्म भवत्सन्दर्शनादभूत्} %3-42

\threelineshloka
{अद्य मे क्रतवः सर्वे बभूवुः सफलाः प्रभो}
{दीर्घकालं मया तप्तमनन्यमतिना तपः}
{तस्येह तपसो राम फलं तव यदर्चनम्} %3-43

\twolineshloka
{सदा मे सीतया सार्धं हृदये वस राघव}
{गच्छतस्तिष्ठतो वाऽपि स्मृतिः स्यान्मे सदा त्वयि} %3-44

\twolineshloka
{इति स्तुत्वा रमानाथमगस्त्यो मुनिसत्तमः}
{ददौ चापं महेन्द्रेण रामार्थे स्थापितं पुरा} %3-45

\twolineshloka
{अक्षय्यौ बाणतूणीरौ खड्गो रत्नविभूषितः}
{जहि राघव भूभारभूतं राक्षसमण्डलम्} %3-46

\twolineshloka
{यदर्थमवतीर्णोऽसि मायया मनुजाकृतिः}
{इतो योजनयुग्मे तु पुण्यकाननमण्डितः} %3-47

\twolineshloka
{अस्ति पञ्चवटीनाम्ना आश्रमो गौतमीतटे}
{नेतव्यस्तत्र ते कालः शेषो रघुकुलोद्वह} %3-48

\onelineshloka
{तत्रैव बहुकार्याणि देवानां कुरु सत्पते} %3-49


\fourlineindentedshloka
{श्रुत्वा तदाऽगस्त्यसुभाषितं वचः}
{स्तोत्रं च तत्त्वार्थसमन्वितं विभुः}
{मुनिं समाभाष्य मुदान्वितो ययौ}
{प्रदर्शितं मार्गमशेषविद्धरिः} %3-50

{॥इति श्रीमदध्यात्मरामयणे उमामहेश्वरसंवादे
अरण्यकाण्डे तृतीयः सर्गः॥३॥
}
%%%%%%%%%%%%%%%%%%%%



\sect{चतुर्थः सर्गः}

\uvacha{श्री महादेव उवाच}

\twolineshloka
{मार्गे व्रजन् ददर्शाथ शैलशृङ्गमिव स्थितम्}
{वृद्धं जटायुषं रामः किमेतदिति विस्मितः} %4-1

\twolineshloka
{धनुरानय सौमित्रे राक्षसोऽयं पुरः स्थितः}
{इत्याह लक्ष्मणं रामो हनिष्याम्यृषिभक्षकम्} %4-2

\twolineshloka
{तच्छ्रुत्वा रामवचनं गृध्रराड् भयपीडितः}
{वधार्होऽहं न ते राम पितुस्तेऽहं प्रियः सखा} %4-3

\onelineshloka
{जटायुर्नाम भद्रं ते गृध्रोऽहं प्रियकृत्तव} %4-4


\twolineshloka
{पञ्चवट्यामहं वत्स्ये तवैव प्रियकाम्यया}
{मृगयायां कदाचित्तु प्रयाते लक्ष्मणेऽपि च} %4-5

\twolineshloka
{सीता जनककन्या मे रक्षितव्या प्रयत्नतः}
{श्रुत्वा तद्गृध्रवचनं रामः सस्नेहमब्रवीत्} %4-6

\twolineshloka
{साधु गृध्र महाराज तथैव कुरु मे प्रियम्}
{अत्रैव मे समीपस्थो नातिदूरे वने वसन्} %4-7

\twolineshloka
{इत्यामन्त्रितमालिङ्ग्य ययौ पञ्चवटीं प्रभुः}
{लक्ष्मणेन सह भ्रात्रा सीतया रघुनन्दनः} %4-8

\twolineshloka
{गत्वा ते गौतमीतीरं पञ्चवट्यां सुविस्तरम्}
{मन्दिरं कारयामास लक्ष्मणेन सुबुद्धिना} %4-9

\twolineshloka
{तत्र ते न्यवसन् सर्वे गङ्गाया उत्तरे तटे}
{कदम्बपनसाम्रादिफलवृक्षसमाकुले} %4-10

\twolineshloka
{विविक्ते जनसम्बाधवर्जिते नीरुजस्थले}
{विनोदयन् जनकजां लक्ष्मणेन विपश्चिता} %4-11

\twolineshloka
{अध्युवास सुखं रामो देवलोक इवापरः}
{कन्दमूलफलादीनि लक्ष्मणोऽनुदिनं तयोः} %4-12

\twolineshloka
{आनीय प्रददौ रामसेवातत्परमानसः}
{धनुर्बाणधरो नित्यं रात्रौ जागर्ति सर्वतः} %4-13

\twolineshloka
{स्नानं कुर्वन्त्यनुदिनं त्रयस्ते गौतमीजले}
{उभयोर्मध्यगा सीता कुरुते च गमागमौ} %4-14

\twolineshloka
{आनीय सलिलं नित्यं लक्ष्मणः प्रीतमानसः}
{सेवतेऽहरहः प्रीत्या एवमासन् सुखं त्रयः} %4-15

\twolineshloka
{एकदा लक्ष्मणो राममेकान्ते समुपस्थितम्}
{विनयावनतो भूत्वा पप्रच्छ परमेश्वरम्} %4-16

\twolineshloka
{भगवन् श्रोतुमिच्छामि मोक्षस्यैकान्तिकीं गतिम्}
{त्वत्तः कमलपत्राक्ष सङ्क्षेपाद्वक्तुमर्हसि} %4-17

\twolineshloka
{ज्ञानं विज्ञानसहितं भक्तिवैराग्यबृंहितम्}
{आचक्ष्व मे रघुश्रेष्ठ वक्ता नान्योऽस्ति भूतले} %4-18

\uvacha{श्रीराम उवाच}

\twolineshloka
{शृणु वक्ष्यामि ते वत्स गुह्याद्गुह्यतरं परम्}
{यद्विज्ञाय नरो जह्यात्सद्यो वैकल्पकं भ्रमम्} %4-19

\twolineshloka
{आदौ मायास्वरूपं ते वक्ष्यामि तदनन्तरम्}
{ज्ञानस्य साधनं पश्चाज्ज्ञानविज्ञानसंयुतम्} %4-20

\twolineshloka
{ज्ञेयं च परमात्मानं यज्ज्ञात्वा मुच्यते भयात्}
{अनात्मनि शरीरादावात्मबुद्धिस्तु या भवेत्} %4-21

\twolineshloka
{सैव माया तयैवासौ संसारः परिकल्प्यते}
{रूपे द्वे निश्चिते पूर्वे मायायाः कुलनन्दन} %4-22

\twolineshloka
{विक्षेपावरणे तत्र प्रथमं कल्पयेज्जगत्}
{लिङ्गाद्यब्रह्मपर्यन्तं स्थूलसूक्ष्मविभेदतः} %4-23

\twolineshloka
{अपरं त्वखिलं ज्ञानरूपमावृत्य तिष्ठति}
{मायया कल्पितं विश्वं परमात्मनि केवले} %4-24

\twolineshloka
{रज्जौ भुजङ्गवद्भ्रान्त्या विचारे नास्ति किञ्चन}
{श्रूयते दृश्यते यद्यत्स्मर्यते वा नरैः सदा} %4-25

\twolineshloka
{असदेव हि तत्सर्वं यथा स्वप्नमनोरथौ}
{देह एव हि संसारवृक्षमूलं दृढं स्मृतम्} %4-26

\onelineshloka
{तन्मूलः पुत्रदारादिबन्धः किं तेऽन्यथाऽऽत्मनः} %4-27


\twolineshloka
{देहस्तु स्थूलभूतानां पञ्च तन्मात्रपञ्चकम्}
{अहङ्कारश्च बुद्धिश्च इन्द्रियाणि तथा दश} %4-28

\twolineshloka
{चिदाभासो मनश्चैव मूलप्रकृतिरेव च}
{एतत्क्षेत्रमिति ज्ञेयं देह इत्यभिधीयते} %4-29

\twolineshloka
{एतैर्विलक्षणो जीवः परमात्मा निरामयः}
{तस्य जीवस्य विज्ञाने साधनान्यपि मे शृणु} %4-30

\twolineshloka
{जीवश्च परमात्मा च पर्यायो नात्र भेदधीः}
{मानाभावस्तथा दम्भहिंसादिपरिवर्जनम्} %4-31

\twolineshloka
{पराक्षेपादिसहनं सर्वत्रावक्रता तथा}
{मनोवाक्कायसद्भक्त्या सद्गुरोः परिसेवनम्} %4-32

\twolineshloka
{बाह्याभ्यन्तरसंशुद्धिः स्थिरता सत्क्रियादिषु}
{मनोवाक्कायदण्डश्च विषयेषु निरीहता} %4-33

\twolineshloka
{निरहङ्कारता जन्मजराद्यालोचनं तथा}
{असक्तिः स्नेहशून्यत्वं पुत्रदारधनादिषु} %4-34

\twolineshloka
{इष्टानिष्टागमे नित्यं चित्तस्य समता तथा}
{मयि सर्वात्मके रामे ह्यनन्यविषया मतिः} %4-35

\twolineshloka
{जनसम्बाधरहितशुद्धदेशनिषेवणम्}
{प्राकृतैर्जनसङ्घैश्च ह्यरतिः सर्वदा भवेत्} %4-36

\twolineshloka
{आत्मज्ञाने सदोद्योगो वेदान्तार्थावलोकनम्}
{उक्तैरेतैर्भवेज्ज्ञानं विपरीतैर्विपर्ययः} %4-37

\twolineshloka
{बुद्धिप्राणमनोदेहाहङ्कृतिभ्यो विलक्षणः}
{चिदात्माऽहं नित्यशुद्धो बुद्ध एवेति निश्चयम्} %4-38

\twolineshloka
{येन ज्ञानेन संवित्ते तज्ज्ञानं निश्चितं च मे}
{विज्ञानं च तदैवैतत्साक्षादनुभवेद्यदा} %4-39

\twolineshloka
{आत्मा सर्वत्र पूर्णः स्याच्चिदानन्दात्मकोऽव्ययः}
{बुद्ध्याद्युपाधिरहितः परिणामादिवर्जितः} %4-40

\twolineshloka
{स्वप्रकाशेन देहादीन् भासयन्ननपावृतः}
{एक एवाद्वितीयश्च सत्यज्ञानादिलक्षणः} %4-41

\twolineshloka
{असङ्गः स्वप्रभो द्रष्टा विज्ञानेनावगम्यते}
{आचार्यशास्त्रोपदेशाद्यैक्यज्ञानं यदा भवेत्} %4-42

\twolineshloka
{आत्मनोर्जीवपरयोर्मूलाविद्या तदैव हि}
{लीयते कार्यकरणैः सहैव परमात्मनि} %4-43

\twolineshloka
{सावस्था मुक्तिरित्युक्ता ह्युपचारोऽयमात्मनि}
{इदं मोक्षस्वरूपं ते कथितं रघुनन्दन} %4-44

\twolineshloka
{ज्ञानविज्ञानवैराग्यसहितं मे परात्मनः}
{किन्त्वेतद्दुर्लभं मन्ये मद्भक्तिविमुखात्मनाम्} %4-45

\twolineshloka
{चक्षुष्मतामपि तथा रात्रौ सम्यङ् न दृश्यते}
{पदं दीपसमेतानां दृश्यते सम्यगेव हि} %4-46

\twolineshloka
{एवं मद्भक्तियुक्तानामात्मा सम्यक् प्रकाशते}
{मद्भक्तेः कारणं किञ्चिद्वक्ष्यामि शृणु तत्त्वतः} %4-47

\twolineshloka
{मद्भक्तसङ्गो मत्सेवा मद्भक्तानां निरन्तरम्}
{एकादश्युपवासादि मम पर्वानुमोदनम्} %4-48

\twolineshloka
{मत्कथाश्रवणे पाठे व्याख्याने सर्वदा रतिः}
{मत्पूजापरिनिष्ठा च मम नामानुकीर्तनम्} %4-49

\twolineshloka
{एवं सततयुक्तानां भक्तिरव्यभिचारिणी}
{मयि सञ्जायते नित्यं ततः किमवशिष्यते} %4-50

\twolineshloka
{अतो मद्भक्तियुक्तस्य ज्ञानं विज्ञानमेव च}
{वैराग्यं च भवेच्छीघ्रं ततो मुक्तिमवाप्नुयात्} %4-51

\twolineshloka
{कथितं सर्वमेतत्ते तव प्रश्नानुसारतः}
{अस्मिन्मनः समाधाय यस्तिष्ठेत्स तु मुक्तिभाक्} %4-52

\twolineshloka
{न वक्तव्यमिदं यत्नान्मद्भक्तिविमुखाय हि}
{मद्भक्ताय प्रदातव्यमाहूयापि प्रयत्नतः} %4-53

\twolineshloka
{य इदं तु पठेन्नित्यं श्रद्धाभक्तिसमन्वितः}
{अज्ञानपटलध्वान्तं विधूय परिमुच्यते} %4-54

\fourlineindentedshloka
{भक्तानां मम योगिनां सुविमलस्वान्तातिशान्तात्मनाम्}
{मत्सेवाभिरतात्मनां च विमलज्ञानात्मनां सर्वदा}
{सङ्गं यः कुरुते सदोद्यतमतिस्तत्सेवनानन्यधीः}
{मोक्षस्तस्य करे स्थितोऽहमनिशं दृश्यो भवे नान्यथा} %4-55

{॥इति श्रीमदध्यात्मरामयणे उमामहेश्वरसंवादे
अरण्यकाण्डे चतुर्थः सर्गः॥४॥
}
%%%%%%%%%%%%%%%%%%%%



\sect{पञ्चमः सर्गः}

\uvacha{श्री महादेव उवाच}

\twolineshloka
{तस्मिन् काले महारण्ये राक्षसी कामरूपिणी}
{विचचार महासत्त्वा जनस्थाननिवासिनी} %5-1

\twolineshloka
{एकदा गौतमीतीरे पञ्चवट्यां समीपतः}
{पद्मवज्राङ्कुशाङ्कानि पदानि जगतीपतेः} %5-2

\twolineshloka
{दृष्ट्वा कामपरीतात्मा पादसौन्दर्यमोहिता}
{पश्यन्ती सा शनैरायाद्राघवस्य निवेशनम्} %5-3

\twolineshloka
{तत्र सा तं रमानाथं सीतया सह संस्थितम्}
{कन्दर्पसदृशं रामं दृष्ट्वा कामविमोहिता} %5-4

\twolineshloka
{राक्षसी राघवं प्राह कस्य त्वं कः किमाश्रमे}
{युक्तो जटावल्कलाद्यैः साध्यं किं तेऽत्र मे वद} %5-5

\twolineshloka
{अहं शूर्पणखा नाम राक्षसी कामरूपिणी}
{भगिनी राक्षसेन्द्रस्य रावणस्य महात्मनः} %5-6

\twolineshloka
{खरेण सहिता भ्रात्रा वसाम्यत्रैव कानने}
{राज्ञा दत्तं च मे सर्वं मुनिभक्षा वसाम्यहम्} %5-7

\twolineshloka
{त्वां तु वेदितुमिच्छामि वद मे वदतां वर}
{तामाह रामनामाहमयोध्याधिपतेः सुतः} %5-8

\twolineshloka
{एषा मे सुन्दरी भार्या सीता जनकनन्दिनी}
{स तु भ्राता कनीयान् मे लक्ष्मणोऽतीवसुन्दरः} %5-9

\twolineshloka
{किं कृत्यं ते मया ब्रूहि कार्यं भुवनसुन्दरि}
{इति रामवचः श्रुत्वा कामार्ता साब्रवीदिदम्} %5-10

\twolineshloka
{एहि राम मया सार्धं रमस्व गिरिकानने}
{कामार्ताहं न शक्नोमि त्यक्तुं त्वां कमलेक्षणम्} %5-11

\twolineshloka
{रामः सीतां कटाक्षेण पश्यन् सस्मितमब्रवीत्}
{भार्या ममैषा कल्याणी विद्यते ह्यनपायिनी} %5-12

\twolineshloka
{त्वं तु सापत्न्यदुःखेन कथं स्थास्यसि सुन्दरि}
{बहिरास्ते मम भ्राता लक्ष्मणोऽतीव सुन्दरः} %5-13

\twolineshloka
{तवानुरूपो भविता पतिस्तेनैव सञ्चर}
{इत्युक्ता लक्ष्मणं प्राह पतिर्मे भव सुन्दर} %5-14

\twolineshloka
{भ्रातुराज्ञां पुरस्कृत्य सङ्गच्छावोऽद्य मा चिरम्}
{इत्याह राक्षसी घोरा लक्ष्मणं काममोहिता} %5-15

\twolineshloka
{तामाह लक्ष्मणः साध्वि दासोऽहं तस्य धीमतः}
{दासी भविष्यसि त्वं तु ततो दुःखतरं नु किम्} %5-16

\twolineshloka
{तमेव गच्छ भद्रं ते स तु राजाखिलेश्वरः}
{तच्छ्रुत्वा पुनरप्यागाद्राघवं दुष्टमानसा} %5-17

\twolineshloka
{क्रोधाद्राम किमर्थं मां भ्रामयस्यनवस्थितः}
{इदानीमेव तां सीतां भक्षयामि तवाग्रतः} %5-18

\twolineshloka
{इत्युक्त्वा विकटाकारा जानकीमनुधावति}
{ततो रामाज्ञया खड्गमादाय परिगृह्य ताम्} %5-19

\twolineshloka
{चिच्छेद नासां कर्णौ च लक्ष्मणो लघुविक्रमः}
{ततो घोरध्वनिं कृत्वा रुधिराक्तवपुर्द्रुतम्} %5-20

\twolineshloka
{क्रन्दमाना पपाताग्रे खरस्य परुषाक्षरा}
{किमेतदिति तामाह खरः खरतराक्षरः} %5-21

\twolineshloka
{केनैवं कारितासि त्वं मृत्योर्वक्त्रानुवर्तिना}
{वद मे तं वधिष्यामि कालकल्पमपि क्षणात्} %5-22

\twolineshloka
{तमाह राक्षसी रामः सीतालक्ष्मणसंयुतः}
{दण्डकं निर्भयं कुर्वन्नास्ते गोदावरीतटे} %5-23

\twolineshloka
{मामेवं कृतवान्स्तस्य भ्राता तेनैव चोदितः}
{यदि त्वं कुलजातोऽसि वीरोऽसि जहि तौ रिपू} %5-24

\twolineshloka
{तयोस्तु रुधिरं पास्ये भक्षयैतौ सुदुर्मदौ}
{नो चेत्प्राणान् परित्यज्य यास्यामि यमसादनम्} %5-25

\twolineshloka
{तच्छ्रुत्वा त्वरितं प्रागात्खरः क्रोधेन मूर्च्छितः}
{चतुर्दश सहस्राणि रक्षसां भीमकर्मणाम्} %5-26

\twolineshloka
{चोदयामास रामस्य समीपं वधकाङ्क्षया}
{खरश्च त्रिशिराश्चैव दूषणश्चैव राक्षसः} %5-27

\twolineshloka
{सर्वे रामं ययुः शीघ्रं नानाप्रहरणोद्यताः}
{श्रुत्वा कोलाहलं तेषां रामः सौमित्रिमब्रवीत्} %5-28

\twolineshloka
{श्रूयते विपुलः शब्दो नूनमायान्ति राक्षसाः}
{भविष्यति महद्युद्धं नूनमद्य मया सह} %5-29

\twolineshloka
{सीतां नीत्वा गुहां गत्वा तत्र तिष्ठ महाबल}
{हन्तुमिच्छाम्यहं सर्वान् राक्षसान् घोररूपिणः} %5-30

\twolineshloka
{अत्र किञ्चिन्न वक्तव्यं शापितोऽसि ममोपरि}
{तथेति सीतामादाय लक्ष्मणो गह्वरं ययौ} %5-31

\twolineshloka
{रामः परिकरं बद्ध्वा धनुरादाय निष्ठुरम्}
{तूणीरावक्षयशरौ बद्ध्वायत्तोऽभवत्प्रभुः} %5-32

\twolineshloka
{तत आगत्य रक्षांसि रामस्योपरि चिक्षिपुः}
{आयुधानि विचित्राणि पाषाणान् पादपानपि} %5-33

\twolineshloka
{तानि चिच्छेद रामोऽपि लीलया तिलशः क्षणात्}
{ततो बाणसहस्रेण हत्वा तान् सर्वराक्षसान्} %5-34

\twolineshloka
{खरं त्रिशिरसं चैव दूषणं चैव राक्षसम्}
{जघान प्रहरार्धेन सर्वानेव रघूत्तमः} %5-35

\twolineshloka
{लक्ष्मणोऽपि गुहामध्यात्सीतामादाय राघवे}
{समर्प्य राक्षसान् दृष्ट्वा हतान् विस्मयमाययौ} %5-36

\twolineshloka
{सीता रामं समालिङ्ग्य प्रसन्नमुखपङ्कजा}
{शस्त्रव्रणानि चाङ्गेषु ममार्ज जनकात्मजा} %5-37

\twolineshloka
{साऽपि दुद्राव दृष्ट्वा तान् हतान् राक्षसपुङ्गवान्}
{लङ्कां गत्वा सभामध्ये क्रोशन्ती पादसन्निधौ} %5-38

\twolineshloka
{रावणस्य पपातोर्व्यां भगिनी तस्य रक्षसः}
{दृष्ट्वा तां रावणः प्राह भगिनीं भयविह्वलाम्} %5-39

\twolineshloka
{उत्तिष्ठोत्तिष्ठ वत्से त्वं विरूपकरणं तव}
{कृतं शक्रेण वा भद्रे यमेन वरुणेन वा} %5-40

\twolineshloka
{कुबेरेणाथवा ब्रूहि भस्मीकुर्यां क्षणेन तम्}
{राक्षसी तमुवाचेदं त्वं प्रमत्तो विमूढधीः} %5-41

\twolineshloka
{पानासक्तः स्त्रीविजितः षण्ढः सर्वत्र लक्ष्यसे}
{चारचक्षुर्विहीनस्त्वं कथं राजा भविष्यसि} %5-42

\twolineshloka
{खरश्च निहतः सङ्ख्ये दूषणस्त्रिशिरास्तथा}
{चतुर्दश सहस्राणि राक्षसानां महात्मनाम्} %5-43

\threelineshloka
{निहतानि क्षणेनैव रामेणासुरशत्रुणा}
{जनस्थानमशेषेण मुनीनां निर्भयं कृतम्}
{न जानासि विमूढस्त्वमत एव मयोच्यते} %5-44

\uvacha{रावण उवाच}

\twolineshloka
{को वा रामः किमर्थं वा कथं तेनासुरा हताः}
{सम्यक्कथय मे तेषां मूलघातं करोम्यहम्} %5-45

\uvacha{शूर्पणखोवाच}

\twolineshloka
{जनस्थानादहं याता कदाचित् गौतमीतटे}
{तत्र पञ्चवटी नाम पुरा मुनिजनाश्रया} %5-46

\twolineshloka
{तत्राश्रमे मया दृष्टो रामो राजीवलोचनः}
{धनुर्बाणधरः श्रीमान् जटावल्कलमण्डितः} %5-47

\twolineshloka
{कनीयाननुजस्तस्य लक्ष्मणोऽपि तथाविधः}
{तस्य भार्या विशालाक्षी रूपिणी श्रीरिवापरा} %5-48

\twolineshloka
{देवगन्धर्वनागानां मनुष्याणां तथाविधा}
{न दृष्टा न श्रुता राजन् द्योतयन्ती वनं शुभा} %5-49

\twolineshloka
{आनेतुमहमुद्युक्ता तां भार्यार्थं तवानघ}
{लक्ष्मणो नाम तद्भ्राता चिच्छेद मम नासिकाम्} %5-50

\twolineshloka
{कर्णौ च नोदितस्तेन रामेण स महाबलः}
{ततोऽहमतिदुःखेन रुदती खरमन्वगाम्} %5-51

\twolineshloka
{सोऽपि रामं समासाद्य युद्धं राक्षसयूथपैः}
{अतः क्षणेन रामेण तेनैव बलशालिना} %5-52

\twolineshloka
{सर्वे तेन विनष्टा वै राक्षसा भीमविक्रमाः}
{यदि रामो मनः कुर्यात्त्रैलोक्यं निमिषार्धतः} %5-53

\twolineshloka
{भस्मीकुर्यान्न सन्देह इति भाति मम प्रभो}
{यदि सा तव भार्या स्यात्सफलं तव जीवितम्} %5-54

\twolineshloka
{अतो यतस्व राजेन्द्र यथा ते वल्लभा भवेत्}
{सीता राजीवपत्राक्षी सर्वलोकैकसुन्दरी} %5-55

\twolineshloka
{साक्षाद्रामस्य पुरतः स्थातुं त्वं न क्षमः प्रभो}
{मायया मोहयित्वा तु प्राप्स्यसे तां रघूत्तमम्} %5-56

\threelineshloka
{श्रुत्वा तत्सूक्तवाक्यैश्च दानमानादिभिस्तथा}
{आश्वास्य भगिनीं राजा प्रविवेश स्वकं गृहम्}
{तत्र चिन्तापरो भूत्वा निद्रां रात्रौ न लब्धवान्} %5-57

\fourlineindentedshloka
{एकेन रामेण कथं मनुष्य-}
{मात्रेण नष्टः सबलः खरो मे}
{भ्राता कथं मे बलवीर्यदर्प-}
{युतो विनष्टो बत राघवेण} %5-58

\fourlineindentedshloka
{यद्वा न रामो मनुजः परेशो}
{मां हन्तुकामः सबलं बलौघैः}
{सम्प्रार्थितोऽयं द्रुहिणेन पूर्वम्}
{मनुष्यरूपोऽद्य रघोः कुलेऽभूत्} %5-59

\fourlineindentedshloka
{वध्यो यदि स्यां परमात्मनाऽहम्}
{वैकुण्ठराज्यं परिपालयेऽहम्}
{नो चेदिदं राक्षसराज्यमेव}
{भोक्ष्ये चिरं राममतो व्रजामि} %5-60

\fourlineindentedshloka
{इत्थं विचिन्त्याखिलराक्षसेन्द्रो}
{रामं विदित्वा परमेश्वरं हरिम्}
{विरोधबुद्ध्यैव हरिं प्रयामि}
{द्रुतं न भक्त्या भगवान् प्रसीदेत्} %5-61

{॥इति श्रीमदध्यात्मरामयणे उमामहेश्वरसंवादे
अरण्यकाण्डे पञ्चमः सर्गः॥५॥
}
%%%%%%%%%%%%%%%%%%%%



\sect{षष्ठः सर्गः}

\twolineshloka
{विचिन्त्यैवं निशायां स प्रभाते रथमास्थितः}
{रावणो मनसा कार्यमेकं निश्चित्य बुद्धिमान्} %6-1

\twolineshloka
{ययौ मारीचसदनं परं पारमुदन्वतः}
{मारीचस्तत्र मुनिवज्जटावल्कलधारकः} %6-2

\twolineshloka
{ध्यायन् हृदि परात्मानं निर्गुणं गुणभासकम्}
{समाधिविरमेऽपश्यद्रावणं गृहमागतम्} %6-3

\twolineshloka
{द्रुतमुत्थाय चालिङ्ग्य पूजयित्वा यथाविधि}
{कृतातिथ्यं सुखासीनं मारीचो वाक्यमब्रवीत्} %6-4

\twolineshloka
{समागमनमेतत्ते रथेनैकेन रावण}
{चिन्तापर इवाभासि हृदि कार्यं विचिन्तयन्} %6-5

\twolineshloka
{ब्रूहि मे न हि गोप्यं चेत्करवाणि तव प्रियम्}
{न्याय्यं चेद्ब्रूहि राजेन्द्र वृजिनं मां स्पृशेन्न हि} %6-6

\uvacha{रावण उवाच}

\twolineshloka
{अस्ति राजा दशरथः साकेताधिपतिः किल}
{रामनामा सुतस्तस्य ज्येष्ठः सत्यपराक्रमः} %6-7

\twolineshloka
{विवासयामास सुतं वनं वनजनप्रियम्}
{भार्यया सहितं भ्रात्रा लक्ष्मणेन समन्वितम्} %6-8

\twolineshloka
{स आस्ते विपिने घोरे पञ्चवट्याश्रमे शुभे}
{तस्य भार्या विशालाक्षी सीता लोकविमोहिनी} %6-9

\twolineshloka
{रामो निरपराधान्मे राक्षसान् भीमविक्रमान्}
{खरं च हत्वा विपिने सुखमास्तेऽतिनिर्भयः} %6-10

\twolineshloka
{भगिन्याः शूर्पणखाया निर्दोषायाश्च नासिकाम्}
{कर्णौ चिच्छेद दुष्टात्मा वने तिष्ठति निर्भयः} %6-11

\twolineshloka
{अतस्त्वया सहायेन गत्वा तत्प्राणवल्लभाम्}
{आनयिष्यामि विपिने रहिते राघवेण ताम्} %6-12

\twolineshloka
{त्वं तु मायामृगो भूत्वा ह्याश्रमादपनेष्यसि}
{रामं च लक्ष्मणं चैव तदा सीतां हराम्यहम्} %6-13

\twolineshloka
{त्वं तु तावत्सहायं मे कृत्वा स्थास्यसि पूर्ववत्}
{इत्येवं भाषमाणं तं रावणं वीक्ष्य विस्मितः} %6-14

\twolineshloka
{केनेदमुपदिष्टं ते मूलघातकरं वचः}
{स एव शत्रुर्वध्यश्च यस्त्वन्नाशं प्रतीक्षते} %6-15

\twolineshloka
{रामस्य पौरुषं स्मृत्वा चित्तमद्यापि रावण}
{बालोऽपि मां कौशिकस्य यज्ञसंरक्षणाय सः} %6-16

\twolineshloka
{आगतस्त्विषुणैकेन पातयामास सागरे}
{योजनानां शतं रामस्तदादि भयविह्वलः} %6-17

\onelineshloka
{स्मृत्वा स्मृत्वा तदेवाहं रामं पश्यामि सर्वतः} %6-18


\fourlineindentedshloka
{दण्डकेऽपि पुनरप्यहं वने}
{पूर्ववैरमनुचिन्तयन् हृदि}
{तीक्ष्णशृङ्गमृगरूपमेकदा}
{मादृशैर्बहुभिरावृतोऽभ्ययाम्} %6-19

\fourlineindentedshloka
{राघवं जनकजासमन्वितम्}
{लक्ष्मणेन सहितं त्वरान्वितः}
{आगतोऽहमथ हन्तुमुद्यतो}
{मां विलोक्य शरमेकमक्षिपत्} %6-20

\fourlineindentedshloka
{तेन विद्धहृदयोऽहमुद्भ्रमनः}
{राक्षसेन्द्र पतितोऽस्मि सागरे}
{तत्प्रभृत्यहमिदं समाश्रितः}
{स्थानमूर्जितमिदं भयार्दितः} %6-21

\fourlineindentedshloka
{राममेव सततं विभावये}
{भीतभीत इव भोगराशितः}
{राजरत्नरमणीरथादिकम्}
{श्रोत्रयोर्यदि गतं भयं भवेत्} %6-22

\fourlineindentedshloka
{राम आगत इहेतिशङ्कया}
{बाह्यकार्यमपि सर्वमत्यजम्}
{निद्रया परिवृतो यदा स्वपे}
{राममेव मनसानुचिन्तयन्} %6-23

\fourlineindentedshloka
{स्वप्नदृष्टिगतराघवं तदा}
{बोधितो विगतनिद्र आस्थितः}
{तद्भवानपि विमुच्य चाग्रहम्}
{राघवं प्रति गृहं प्रयाहि भोः} %6-24

\fourlineindentedshloka
{रक्ष राक्षसकुलं चिरागतम्}
{तत्स्मृतौ सकलमेव नश्यति}
{तव हितं वदतो मम भाषितम्}
{परिगृहाण परात्मनि राघवे}%6-25

\begin{minipage}{\linewidth}
\centering
\hspace{-8ex}{त्यज विरोधमतिं भज भक्तितः}\\
{परमकारुणिको रघुनन्दनः।} 
\fourlineindentedshloka
{अहमशेषमिदं मुनिवाक्यतः}
{अशृणवमादियुगे परमेश्वरः}
{ब्रह्मणार्थित उवाच तं हरिः}
{किं तवेप्सितमहं करवाणि तत्} %6-26
\end{minipage}

\fourlineindentedshloka
{ब्रह्मणोक्तमरविन्दलोचन}
{त्वं प्रयाहि भुवि मानुषं वपुः}
{दशरथात्मजभावमञ्जसा}
{जहि रिपुं दशकन्धरं हरे} %6-27

\twolineshloka
{अतो न मानुषो रामः साक्षान्नारायणोऽव्ययः}
{मायामानुषवेषेण वनं यातोऽतिनिर्भयः} %6-28

\twolineshloka
{भूभारहरणार्थाय गच्छ तात गृहं सुखम्}
{श्रुत्वा मारीचवचनं रावणः प्रत्यभाषत} %6-29

\twolineshloka
{परमात्मा यदा रामः प्रार्थितो ब्रह्मणा किल}
{मां हन्तुं मानुषो भूत्वा यत्नादिह समागतः} %6-30

\twolineshloka
{करिष्यत्यचिरादेव सत्यसङ्कल्प ईश्वरः}
{अतोऽहं यत्नतः सीतामानेष्याम्येव राघवात्} %6-31

\twolineshloka
{वधे प्राप्ते रणे वीर प्राप्स्यामि परमं पदम्}
{यद्वा रामं रणे हत्वा सीतां प्राप्स्यामि निर्भयः} %6-32

\twolineshloka
{तदुत्तिष्ठ महाभाग विचित्रमृगरूपधृक्}
{रामं सलक्ष्मणं शीघ्रमाश्रमादतिदूरतः} %6-33

\twolineshloka
{आक्रम्य गच्छ त्वं शीघ्रं सुखं तिष्ठ यथा पुरा}
{अतः परं चेद्यत्किञ्चिद्भाषसे मद्विभीषणम्} %6-34

\twolineshloka
{हनिष्याम्यसिनानेन त्वामत्रैव न संशयः}
{मारीचस्तद्वचः श्रुत्वा स्वात्मन्येवान्वचिन्तयत्} %6-35

\twolineshloka
{यदि मां राघवो हन्यात्तदा मुक्तो भवार्णवात्}
{मां हन्याद्यदि चेद्दुष्टस्तदा मे निरयो ध्रुवम्} %6-36

\twolineshloka
{इति निश्चित्य मरणं रामादुत्थाय वेगतः}
{अब्रवीद्रावणं राजन् करोम्याज्ञां तव प्रभो} %6-37

\twolineshloka
{इत्युक्त्वा रथमास्थाय गतो रामाश्रमं प्रति}
{शुद्धजाम्बूनदप्रख्यो मृगोऽभूद्रौप्यबिन्दुकः} %6-38

\twolineshloka
{रत्नशृङ्गो मणिखुरो नीलरत्नविलोचनः}
{विद्युत्प्रभो विमुग्धास्यो विचचार वनान्तरे} %6-39

\onelineshloka
{रामाश्रमपदस्यान्ते सीतादृष्टिपथे चरन्} %6-40


\fourlineindentedshloka
{क्षणं च धावत्यवतिष्ठते क्षणम्}
{समीपमागत्य पुनर्भयावृतः}
{एवं स मायामृगवेषरूपधृकः}
{चचार सीतां परिमोहयन् खलः} %6-41

{॥इति श्रीमदध्यात्मरामयणे उमामहेश्वरसंवादे
अरण्यकाण्डे षष्ठः सर्गः॥६॥
}
%%%%%%%%%%%%%%%%%%%%



\sect{सप्तमः सर्गः}

\uvacha{श्रीमहादेव उवाच}

\twolineshloka
{अथ रामोऽपि तत्सर्वं ज्ञात्वा रावणचेष्टितम्}
{उवाच सीतामेकान्ते शृणु जानकि मे वचः} %7-1

\twolineshloka
{रावणो भिक्षुरूपेण आगमिष्यति तेऽन्तिकम्}
{त्वं तु छायां त्वदाकारां स्थापयित्वोटजे विश} %7-2

\twolineshloka
{अग्नावदृश्यरूपेण वर्षं तिष्ठ ममाऽऽज्ञया}
{रावणस्य वधान्ते मां पूर्ववत्प्राप्स्यसे शुभे} %7-3

\twolineshloka
{श्रुत्वा रामोदितं वाक्यं साऽपि तत्र तथाऽकरोत्}
{मायासीतां बहिः स्थाप्य स्वयमन्तर्दधेऽनले} %7-4

\twolineshloka
{मायासीता तदाऽपश्यन्मृगं मायाविनिर्मितम्}
{हसन्ती राममभ्येत्य प्रोवाच विनयान्विता} %7-5

\threelineshloka
{पश्य राम मृगं चित्रं कानकं रत्नभूषितम्}
{विचित्रबिन्दुभिर्युक्तं चरन्तमकुतोभयम्}
{बद्ध्वा देहि मम क्रीडामृगो भवतु सुन्दरः} %7-6

\twolineshloka
{तथेति धनुरादाय गच्छन् लक्ष्मणमब्रवीत्}
{रक्ष त्वमतियत्नेन सीतां मत्प्राणवल्लभाम्} %7-7

\twolineshloka
{मायिनः सन्ति विपिने राक्षसा घोरदर्शनाः}
{अतोऽत्रावहितः साध्वीं रक्ष सीतामनिन्दिताम्} %7-8

\twolineshloka
{लक्ष्मणो राममाहेदं देवायं मृगरूपधृक्}
{मारीचोऽत्र न सन्देह एवम्भूतो मृगः कुतः} %7-9

\uvacha{श्रीराम उवाच}

\twolineshloka
{यदि मारीच एवायं तदा हन्मि न संशयः}
{मृगश्चेदानयिष्यामि सीताविश्रमहेतवे} %7-10

\twolineshloka
{गमिष्यामि मृगं बद्ध्वा ह्यानयिष्यामि सत्वरः}
{त्वं प्रयत्नेन सन्तिष्ठ सीतासंरक्षणोद्यतः} %7-11

\twolineshloka
{इत्युक्त्वा प्रययौ रामो मायामृगमनुद्रुतः}
{माया यदाश्रया लोकमोहिनी जगदाकृतिः} %7-12

\twolineshloka
{निर्विकारश्चिदात्माऽपि पूर्णोऽपि मृगमन्वगात्}
{भक्तानुकम्पी भगवानिति सत्यं वचो हरिः} %7-13

\twolineshloka
{कर्तुं सीताप्रियार्थाय जानन्नपि मृगं ययौ}
{अन्यथा पूर्णकामस्य रामस्य विदितात्मनः} %7-14

\onelineshloka
{मृगेण वा स्त्रिया वाऽपि किं कार्यं परमात्मनः} %7-15


\threelineshloka
{कदाचिद् दृश्यतेऽभ्याशे क्षणं धावति लीयते}
{दृश्यते च ततो दूरादेवं राममपाहरत्}
{ततो रामोऽपि विज्ञाय राक्षसोऽयमिति स्फुटम्} %7-16

\twolineshloka
{विव्याध शरमादाय राक्षसं मृगरूपिणम्}
{पपात रुधिराक्तास्यो मारीचः पूर्वरूपधृक्} %7-17

\twolineshloka
{हा हतोऽस्मि महाबाहो त्राहि लक्ष्मण मां द्रुतम्}
{इत्युक्त्वा रामवद्वाचा पपात रुधिराशनः} %7-18

\twolineshloka
{यन्नामाज्ञोऽपि मरणे स्मृत्वा तत्साम्यमाप्नुयात्}
{किमुताग्रे हरिं पश्यन्स्तेनैव निहतोऽसुरः} %7-19

\twolineshloka
{तद्देहादुत्थितं तेजः सर्वलोकस्य पश्यतः}
{राममेवाविशद्देवा विस्मयं परमं ययुः} %7-20

\twolineshloka
{किं कर्म कृत्वा किं प्राप्तः पातकी मुनिहिंसकः}
{अथवा राघवस्यायं महिमा नात्र संशयः} %7-21

\twolineshloka
{रामबाणेन संविद्धः पूर्वं राममनुस्मरन्}
{भयात्सर्वं परित्यज्य गृहवित्तादिकं च यत्} %7-22

\twolineshloka
{हृदि रामं सदा ध्यात्वा निर्धूताशेषकल्मषः}
{अन्ते रामेण निहतः पश्यन् राममवाप सः} %7-23

\twolineshloka
{द्विजो वा राक्षसो वाऽपि पापी वा धार्मिकोऽपि वा}
{त्यजन् कलेवरं रामं स्मृत्वा याति परं पदम्} %7-24

\twolineshloka
{इति तेऽन्योन्यमाभाष्य ततो देवा दिवं ययुः}
{रामस्तच्चिन्तयामास म्रियमाणोऽसुराधमः} %7-25

\twolineshloka
{हा लक्ष्मणेति मद्वाक्यमनुकुर्वन्ममार किम्}
{श्रुत्वा मद्वाक्यसदृशं वाक्यं सीताऽपि किं भवेत्} %7-26

\twolineshloka
{इति चिन्तापरीतात्मा रामो दूरान्न्यवर्तत}
{सीता तद्भाषितं श्रुत्वा मारीचस्य दुरात्मनः} %7-27

\twolineshloka
{भीतातिदुःखसंविग्ना लक्ष्मणं त्विदमब्रवीत्}
{गच्छ लक्ष्मण वेगेन भ्राता तेऽसुरपीडितः} %7-28

\twolineshloka
{हा लक्ष्मणेति वचनं भ्रातुस्ते न शृणोषि किम्}
{तामाह लक्ष्मणो देवि रामवाक्यं न तद्भवेत्} %7-29

\twolineshloka
{यः कश्चिद्राक्षसो देवि म्रियमाणोऽब्रवीद्वचः}
{रामस्त्रैलोक्यमपि यः क्रुद्धो नाशयति क्षणात्} %7-30

\twolineshloka
{स कथं दीनवचनं भाषतेऽमरपूजितः}
{क्रुद्धा लक्ष्मणमालोक्य सीता बाष्पविलोचना} %7-31

\twolineshloka
{प्राह लक्ष्मण दुर्बुद्धे भ्रातुर्व्यसनमिच्छसि}
{प्रेषितो भरतेनैव रामनाशाभिकाङ्क्षिणा} %7-32

\twolineshloka
{मां नेतुमागतोऽसि त्वं रामनाश उपस्थिते}
{न प्राप्स्यसे त्वं मामद्य पश्य प्राणान्स्त्यजाम्यहम्} %7-33

\twolineshloka
{न जानातीदृशं रामस्त्वां भार्याहरणोद्यतम्}
{रामादन्यं न स्पृशामि त्वां वा भरतमेव वा} %7-34

\twolineshloka
{इत्युक्त्वा वध्यमाना सा स्वबाहुभ्यां रुरोद ह}
{तच्छ्रुत्वा लक्ष्मणः कर्णौ पिधायातीव दुःखितः} %7-35

\twolineshloka
{मामेवं भाषसे चण्डि धिक् त्वां नाशमुपैष्यसि}
{इत्युक्त्वा वनदेवीभ्यः समर्प्य जनकात्मजाम्} %7-36

\twolineshloka
{ययौ दुःखातिसंविग्नो राममेव शनैः शनैः}
{ततोऽन्तरं समालोक्य रावणो भिक्षुवेषधृक्} %7-37

\twolineshloka
{सीतासमीपमगमत् स्फुरद्दण्डकमण्डलुः}
{सीता तमवलोक्याऽऽशु नत्वा सम्पूज्य भक्तितः} %7-38

\twolineshloka
{कन्दमूलफलादीनि दत्त्वा स्वागतमब्रवीत्}
{मुने भुङ्क्ष्व फलादीनि विश्रमस्व यथासुखम्} %7-39

\twolineshloka
{इदानीमेव भर्ता मे ह्यागमिष्यति ते प्रियम्}
{करिष्यति विशेषेण तिष्ठ त्वं यदि रोचते} %7-40

\uvacha{भिक्षुरुवाच}

\threelineshloka
{का त्वं कमलपत्राक्षि को वा भर्ता तवानघे}
{किमर्थमत्र ते वासो वने राक्षससेविते}
{ब्रूहि भद्रे ततः सर्वं स्ववृत्तान्तं निवेदये} %7-41

\uvacha{सीतोवाच}

\twolineshloka
{अयोध्याधिपतिः श्रीमान् राजा दशरथो महान्}
{तस्य ज्येष्ठः सुतो रामः सर्वलक्षणलक्षितः} %7-42

\twolineshloka
{तस्याहं धर्मतः पत्नी सीता जनकनन्दिनी}
{तस्य भ्राता कनीयान्श्च लक्ष्मणो भ्रातृवत्सलः} %7-43

\twolineshloka
{पितुराज्ञां पुरस्कृत्य दण्डके वस्तुमागतः}
{चतुर्दश समास्त्वां तु ज्ञातुमिच्छामि मे वद} %7-44

\uvacha{भिक्षुरुवाच}

\twolineshloka
{पौलस्त्यतनयोऽहं तु रावणो राक्षसाधिपः}
{त्वत्कामपरितप्तोऽहं त्वां नेतुं पुरमागतः} %7-45

\twolineshloka
{मुनिवेषेण रामेण किं करिष्यसि मां भज}
{भुङ्क्ष्व भोगान् मया सार्धं त्यज दुःखं वनोद्भवम्} %7-46

\twolineshloka
{श्रुत्वा तद्वचनं सीता भीता किञ्चिदुवाच तम्}
{यद्येवं भाषसे मां त्वं नाशमेष्यसि राघवात्} %7-47

\twolineshloka
{आगमिष्यति रामोऽपि क्षणं तिष्ठ सहानुजः}
{मां को धर्षयितुं शक्तो हरेर्भार्यां शशो यथा} %7-48

\onelineshloka
{रामबाणैर्विभिन्नस्त्वं पतिष्यसि महीतले} %7-49


\threelineshloka
{इति सीतावचः श्रुत्वा रावणः क्रोधमूर्च्छितः}
{स्वरूपं दर्शयामास महापर्वतसन्निभम्}
{दशास्यं विंशतिभुजं कालमेघसमद्युतिम्} %7-50

\twolineshloka
{तद्दृष्ट्वा वनदेव्यश्च भूतानि च वितत्रसुः}
{ततो विदार्य धरणीं नखैरुद्धृत्य बाहुभिः} %7-51

\twolineshloka
{तोलयित्वा रथे क्षिप्त्वा ययौ क्षिप्रं विहायसा}
{हा राम हा लक्ष्मणेति रुदती जनकात्मजा} %7-52

\twolineshloka
{भयोद्विग्नमना दीना पश्यन्ती भुवमेव सा}
{श्रुत्वा तत्क्रन्दितं दीनं सीतायाः पक्षिसत्तमः} %7-53

\twolineshloka
{जटायुरुत्थितः शीघ्रं नगाग्रात्तीक्ष्णतुण्डकः}
{तिष्ठ तिष्ठेति तं प्राह को गच्छति ममाग्रतः} %7-54

\twolineshloka
{मुषित्वा लोकनाथस्य भार्यां शून्याद्वनालयात्}
{शुनको मन्त्रपूतं त्वं पुरोडाशमिवाध्वरे} %7-55

\twolineshloka
{इत्युक्त्वा तीक्ष्णतुण्डेन चूर्णयामास तद्रथम्}
{वाहान् बिभेद पादाभ्यां चूर्णयामास तद्धनुः} %7-56

\twolineshloka
{ततः सीतां परित्यज्य रावणः खड्गमाददे}
{चिच्छेद पक्षौ सामर्षः पक्षिराजस्य धीमतः} %7-57

\twolineshloka
{पपात किञ्चिच्छेषेण प्राणेन भुवि पक्षिराट्}
{पुनरन्यरथेनाशु सीतामादाय रावणः} %7-58

\twolineshloka
{क्रोशन्ती रामरामेति त्रातारं नाधिगच्छति}
{हा राम हा जगन्नाथ मां न पश्यसि दुःखिताम्} %7-59

\twolineshloka
{रक्षसा नीयमानां स्वां भार्यां मोचय राघव}
{हा लक्ष्मण महाभाग त्राहि मामपराधिनीम्} %7-60

\twolineshloka
{वाकःशरेण हतस्त्वं मे क्षन्तुमर्हसि देवर}
{इत्येवं क्रोशमानां तां रामागमनशङ्कया} %7-61

\onelineshloka
{जगाम वायुवेगेन सीतामादाय सत्वरः} %7-62


\threelineshloka
{विहायसा नीयमाना सीतापश्यदधोमुखी}
{पर्वताग्रे स्थितान् पञ्च वानरान् वारिजानना}
{उत्तरीयार्धखण्डेन विमुच्याभरणादिकम्} %7-63

\onelineshloka
{बद्ध्वा चिक्षेप रामाय कथयन्त्विति पर्वते} %7-64


\threelineshloka
{ततः समुद्रमुल्लङ्घ्य लङ्कां गत्वा स रावणः}
{स्वान्तःपुरे रहस्ये तामशोकविपिनेऽक्षिपत्}
{राक्षसीभिः परिवृतां मातृबुद्ध्यान्वपालयत्} %7-65

\fourlineindentedshloka
{कृशाऽतिदीना परिकर्मवर्जिता}
{दुःखेन शुष्यद्वदनाऽतिविह्वला}
{हा राम रामेति विलप्यमाना}
{सीता स्थिता राक्षसवृन्दमध्ये} %7-66

{॥इति श्रीमदध्यात्मरामयणे उमामहेश्वरसंवादे
अरण्यकाण्डे सप्तमः सर्गः॥७॥
}
%%%%%%%%%%%%%%%%%%%%



\sect{अष्टमः सर्गः}

\uvacha{श्रीमहादेव उवाच}

\twolineshloka
{रामो मायाविनं हत्वा राक्षसं कामरूपिणम्}
{प्रतस्थे स्वाश्रमं गन्तुं ततो दूराद्ददर्श तम्} %8-1

\twolineshloka
{आयान्तं लक्ष्मणं दीनं मुखेन परिशुष्यता}
{राघवश्चिन्तयामास स्वात्मन्येव महामतिः} %8-2

\twolineshloka
{लक्ष्मणस्तन्न जानाति मायासीतां मया कृताम्}
{ज्ञात्वाऽप्येनं वञ्चयित्वा शोचामि प्राकृतो यथा} %8-3

\twolineshloka
{यद्यहं विरतो भूत्वा तूष्णीं स्थास्यामि मन्दिरे}
{तदा राक्षसकोटीनां वधोपायः कथं भवेत्} %8-4

\threelineshloka
{यदि शोचामि तां दुःखसन्तप्तः कामुको यथा}
{तदा क्रमेणानुचिन्वन् सीतां यास्येऽसुरालयम्}
{रावणं सकुलं हत्वा सीतामग्नौ स्थितां पुनः} %8-5

\onelineshloka
{मयैव स्थापितां नीत्वा यातायोध्यामतन्द्रितः} %8-6


\threelineshloka
{अहं मनुष्यभावेन जातोऽस्मि ब्रह्मणार्थितः}
{मनुष्यभावमापन्नः किञ्चित्कालं वसामि कौ}
{ततो मायामनुष्यस्य चरितं मेऽनुशृण्वताम्} %8-7

\twolineshloka
{मुक्तिः स्यादप्रयासेन भक्तिमार्गानुवर्तिनाम्}
{निश्चित्यैवं तदा दृष्ट्वा लक्ष्मणं वाक्यमब्रवीत्} %8-8

\twolineshloka
{किमर्थमागतोऽसि त्वं सीतां त्यक्त्वा मम प्रियाम्}
{नीता वा भक्षिता वाऽपि राक्षसैर्जनकात्मजा} %8-9

\twolineshloka
{लक्ष्मणः प्राञ्जलिः प्राह सीताया दुर्वचो रुदन्}
{हा लक्ष्मणेति वचनं राक्षसोक्तं श्रुतं तया} %8-10

\threelineshloka
{त्वद्वाक्यसदृशं श्रुत्वा मां गच्छेति त्वराब्रवीत्}
{रुदन्ती सा मया प्रोक्ता देवि राक्षसभाषितम्}
{नेदं रामस्य वचनं स्वस्था भव शुचिस्मिते} %8-11

\twolineshloka
{इत्येवं सान्त्विता साध्वी मया प्रोवाच मां पुनः}
{यदुक्तं दुर्वचो राम न वाच्यं पुरतस्तव} %8-12

\onelineshloka
{कर्णौ पिधाय निर्गत्य यातोऽहं त्वां समीक्षितुम्} %8-13


\threelineshloka
{रामस्तु लक्ष्मणं प्राह तथाऽप्यनुचितं कृतम्}
{त्वया स्त्रीभाषितं सत्यं कृत्वा त्यक्ता शुभानना}
{नीता वा भक्षिता वाऽपि राक्षसैर्नात्र संशयः} %8-14

\twolineshloka
{इति चिन्तापरो रामः स्वाश्रमं त्वरितो ययौ}
{तत्रादृष्ट्वा जनकजां विललापातिदुःखितः} %8-15

\twolineshloka
{हा प्रिये क्व गतासि त्वं नासि पूर्ववदाश्रमे}
{अथवा मद्विमोहार्थं लीलया क्व विलीयसे} %8-16

\twolineshloka
{इत्याचिन्वन् वनं सर्वं नापश्यज्जानकीं तदा}
{वनदेव्यः कुतः सीतां ब्रुवन्तु मम वल्लभाम्} %8-17

\twolineshloka
{मृगाश्च पक्षिणो वृक्षा दर्शयन्तु मम प्रियाम्}
{इत्येवं विलपन्नेव रामः सीतां न कुत्रचित्} %8-18

\twolineshloka
{सर्वज्ञः सर्वथा क्वापि नापश्यद्रघुनन्दनः}
{आनन्दोऽप्यन्वशोचत्तामचलोऽप्यनुधावति} %8-19

\twolineshloka
{निर्ममो निरहङ्कारोऽप्यखण्डानन्दरूपवान्}
{मम जायेति सीतेति विललापातिदुःखितः} %8-20

\twolineshloka
{एवं मायामनुचरन्नसक्तोऽपि रघूत्तमः}
{आसक्त इव मूढानां भाति तत्त्वविदां न हि} %8-21

\twolineshloka
{एवं विचिन्वन् सकलं वनं रामः सलक्ष्मणः}
{भग्नं रथं छत्रचापं कूबरं पतितं भुवि} %8-22

\twolineshloka
{दृष्ट्वा लक्ष्मणमाहेदं पश्य लक्ष्मण केनचित्}
{नीयमानां जनकजां तं जित्वाऽन्यो जहार ताम्} %8-23

\twolineshloka
{ततः कञ्चिद्भुवो भागं गत्वा पर्वतसन्निभम्}
{रुधिराक्तवपुर्दृष्ट्वा रामो वाक्यमथाब्रवीत्} %8-24

\twolineshloka
{एष वै भक्षयित्वा तां जानकीं शुभदर्शनाम्}
{शेते विविक्तेऽतितृप्तः पश्य हन्मि निशाचरम्} %8-25

\twolineshloka
{चापमानय शीघ्रं मे बाणं च रघुनन्दन}
{तच्छ्रुत्वा रामवचनं जटायुः प्राह भीतवत्} %8-26

\twolineshloka
{मां न मारय भद्रं ते म्रियमाणं स्वकर्मणा}
{अहं जटायुस्ते भार्याहारिणं समनुद्रुतः} %8-27

\twolineshloka
{रावणं तत्र युद्धं मे बभूवारिविमर्दन}
{तस्य वाहान् रथं चापं छित्त्वाऽहं तेन घातितः} %8-28

\twolineshloka
{पतितोऽस्मि जगन्नाथ प्राणान्स्त्यक्ष्यामि पश्य माम्}
{तच्छ्रुत्वा राघवो दीनं कण्ठप्राणं ददर्श ह} %8-29

\onelineshloka
{हस्ताभ्यां संस्पृशन् रामो दुःखाश्रुवृतलोचनः} %8-30


\twolineshloka
{जटायो ब्रूहि मे भार्या केन नीता शुभानना}
{मत्कार्यार्थं हतोऽसि त्वमतो मे प्रियबान्धवः} %8-31

\twolineshloka
{जटायुः सन्नया वाचा वक्त्राद्रक्तं समुद्वमन्}
{उवाच रावणो राम राक्षसो भीमविक्रमः} %8-32

\twolineshloka
{आदाय मैथिलीं सीतां दक्षिणाभिमुखो ययौ}
{इतो वक्तुं न मे शक्तिः प्राणान्स्त्यक्ष्यामि तेऽग्रतः} %8-33

\twolineshloka
{दिष्ट्या दृष्टोऽसि राम त्वं म्रियमाणेन मेऽनघ}
{परमात्मासि विष्णुस्त्वं मायामनुजरूपधृक्} %8-34

\twolineshloka
{अन्तकालेऽपि दृष्ट्वा त्वां मुक्तोऽहं रघुसत्तम}
{हस्ताभ्यां स्पृश मां राम पुनर्यास्यामि ते पदम्} %8-35

\twolineshloka
{तथेति रामः पस्पर्श तदङ्गं पाणिना स्मयन्}
{ततः प्राणान् परित्यज्य जटायुः पतितो भुवि} %8-36

\twolineshloka
{रामस्तमनुशोचित्वा बन्धुवत् साश्रुलोचनः}
{लक्ष्मणेन समानाय्य काष्ठानि प्रददाह} %8-37

\twolineshloka
{स्नात्वा दुःखेन रामोऽपि लक्ष्मणेन समन्वितः}
{हत्वा वने मृगं तत्र मांसखण्डान् समन्ततः} %8-38

\twolineshloka
{शाद्वले प्राक्षिपद्रामः पृथक् पृथगनेकधा}
{भक्षन्तु पक्षिणः सर्वे तृप्तो भवतु पक्षिराट्} %8-39

\twolineshloka
{इत्युक्त्वा राघवः प्राह जटायो गच्छ मत्पदम्}
{मत्सारूप्यं भजस्वाद्य सर्वलोकस्य पश्यतः} %8-40

\twolineshloka
{ततोऽनन्तरमेवासौ दिव्यरूपधरः शुभः}
{विमानवरमारुह्य भास्वरं भानुसन्निभम्} %8-41

\twolineshloka
{शङ्खचक्रगदापद्मकिरीटवरभूषणैः}
{द्योतयन् स्वप्रकाशेन पीताम्बरधरोऽमलः} %8-42

\threelineshloka
{चतुर्भिः पार्षदैर्विष्णोस्तादृशैरभिपूजितः}
{स्तूयमानो योगिगणैः राममाभाष्य सत्वरः}
{कृताञ्जलिपुटो भूत्वा तुष्टाव रघुनन्दनम्} %8-43

\uvacha{जटायुरुवाच}

\fourlineindentedshloka
{अगणितगुणमप्रमेयमाद्यम्}
{सकलजगत्स्थितिसंयमादिहेतुम्}
{उपरमपरमं परात्मभूतम्}
{सततमहं प्रणतोऽस्मि रामचन्द्रम्} %8-44

\fourlineindentedshloka
{निरवधिसुखमिन्दिराकटाक्षम्}
{क्षपितसुरेन्द्रचतुर्मुखादिदुःखम्}
{नरवरमनिशं नतोऽस्मि रामम्}
{वरदमहं वरचापबाणहस्तम्} %8-45

\fourlineindentedshloka
{त्रिभुवनकमनीयरूपमीड्यम्}
{रविशतभासुरमीहितप्रदानम्}
{शरणदमनिशं सुरागमूले}
{कृतनिलयं रघुनन्दनं प्रपद्ये} %8-46

\fourlineindentedshloka
{भवविपिनदवाग्निनामधेयम्}
{भवमुखदैवतदैवतं दयालुम्}
{दनुजपतिसहस्रकोटिनाशम्}
{रवितनयासदृशं हरिं प्रपद्ये} %8-47

\fourlineindentedshloka
{अविरतभवभावनातिदूरम्}
{भवविमुखैर्मुनिभिः सदैव दृश्यम्}
{भवजलधिसुतारणाङ्घ्रिपोतम्}
{शरणमहं रघुनन्दनं प्रपद्ये} %8-48

\fourlineindentedshloka
{गिरिशगिरिसुतामनोनिवासम्}
{गिरिवरधारिणमीहिताभिरामम्}
{सुरवरदनुजेन्द्रसेविताङ्घ्रिम्}
{सुरवरदं रघुनायकं प्रपद्ये} %8-49

\fourlineindentedshloka
{परधनपरदारवर्जितानाम्}
{परगुणभूतिषु तुष्टमानसानाम्}
{परहितनिरतात्मनां सुसेव्यम्}
{रघुवरमम्बुजलोचनं प्रपद्ये} %8-50

\fourlineindentedshloka
{स्मितरुचिरविकासिताननाब्ज-}
{मतिसुलभं सुरराजनीलनीलम्}
{सितजलरुहचारुनेत्रशोभम्}
{रघुपतिमीशगुरोर्गुरुं प्रपद्ये} %8-51

\fourlineindentedshloka
{हरिकमलजशम्भुरूपभेदात्}
{त्वमिह विभासि गुणत्रयानुवृत्तः}
{रविरिव जलपूरितोदपात्रे-}
{ष्वमरपतिस्तुतिपात्रमीशमीडे} %8-52

\fourlineindentedshloka
{रतिपतिशतकोटिसुन्दराङ्गम्}
{शतपथगोचरभावनाविदूरम्}
{यतिपतिहृदये सदा विभातम्}
{रघुपतिमार्तिहरं प्रभुं प्रपद्ये} %8-53

\twolineshloka
{इत्येवं स्तुवतस्तस्य प्रसन्नोऽभूद्रघूत्तमः}
{उवाच गच्छ भद्रं ते मम विष्णोः परं पदम्} %8-54

\twolineshloka
{शृणोति य इदं स्तोत्रं लिखेद्वा नियतः पठेत्}
{स याति मम सारूप्यं मरणे मत्स्मृतिं लभेत्} %8-55

\fourlineindentedshloka
{इति राघवभाषितं तदा}
{श्रुतवान् हर्षसमाकुलो द्विजः}
{रघुनन्दनसाम्यमास्थितः}
{प्रययौ ब्रह्मसुपूजितं पदम्} %8-56

{॥इति श्रीमदध्यात्मरामयणे उमामहेश्वरसंवादे
अरण्यकाण्डे अष्टमः सर्गः॥८॥
}
%%%%%%%%%%%%%%%%%%%%



\sect{नवमः सर्गः}

\uvacha{श्रीमहादेव उवाच}

\twolineshloka
{ततो रामो लक्ष्मणेन जगाम विपिनान्तरम्}
{पुनर्दुःखं समाश्रित्य सीतान्वेषणतत्परः} %9-1

\twolineshloka
{तत्राद्भुतसमाकारो राक्षसः प्रत्यदृश्यत}
{वक्षस्येव महावक्त्रश्चक्षुरादिविवर्जितः} %9-2

\twolineshloka
{बाहू योजनमात्रेण व्यापृतौ तस्य रक्षसः}
{कबन्धो नाम दैत्येन्द्रः सर्वसत्त्वविहिंसकः} %9-3

\twolineshloka
{तद्बाह्वोर्मध्यदेशे तौ चरन्तौ रामलक्ष्मणौ}
{ददर्शतुर्महासत्त्वं तद्बाहुपरिवेष्टितौ} %9-4

\twolineshloka
{रामः प्रोवाच विहसन् पश्य लक्ष्मण राक्षसम्}
{शिरःपादविहीनोऽयं यस्य वक्षसि चाननम्} %9-5

\twolineshloka
{बाहुभ्यां लभ्यते यद्यत्तत्तद्भक्षन् स्थितो ध्रुवम्}
{आवामप्येतयोर्बाह्वोर्मध्ये सङ्कलितौ ध्रुवम्} %9-6

\twolineshloka
{गन्तुमन्यत्र मार्गो न दृश्यते रघुनन्दन}
{किं कर्तव्यमितोऽस्माभिरिदानीं भक्षयेत्स नौ} %9-7

\twolineshloka
{लक्ष्मणस्तमुवाचेदं किं विचारेण राघव}
{आवामेकैकमव्यग्रौ छिन्द्यावास्य भुजौ ध्रुवम्} %9-8

\twolineshloka
{तथेति रामः खड्गेन भुजं दक्षिणमच्छिनत्}
{तथैव लक्ष्मणो वामं चिच्छेद भुजमञ्जसा} %9-9

\twolineshloka
{ततोऽतिविस्मितो दैत्यः कौ युवां सुरपुङ्गवौ}
{मद्बाहुच्छेदकौ लोके दिवि देवेषु वा कुतः} %9-10

\twolineshloka
{ततोऽब्रवीद्धसन्नेव रामो राजीवलोचनः}
{अयोध्याधिपतिः श्रीमान् राजा दशरथो महान्} %9-11

\twolineshloka
{रामोऽहं तस्य पुत्रोऽसौ भ्राता मे लक्ष्मणः सुधीः}
{मम भार्या जनकजा सीता त्रैलोक्यसुन्दरी} %9-12

\twolineshloka
{आवां मृगयया यातौ तदा केनापि रक्षसा}
{नीतां सीतां विचिन्वन्तौ चागतौ घोरकानने} %9-13

\twolineshloka
{बाहुभ्यां वेष्टितावत्र तव प्राणरिरक्षया}
{छिन्नौ तव भुजौ त्वं च को वा विकटरूपधृक्} %9-14

\uvacha{कबन्ध उवाच}

\twolineshloka
{धन्योऽहं यदि रामस्त्वमागतोऽसि ममान्तिकम्}
{पुरा गन्धर्वराजोऽहं रूपयौवनदर्पितः} %9-15

\twolineshloka
{विचरन्ल्लोकमखिलं वरनारीमनोहरः}
{तपसा ब्रह्मणो लब्धमवध्यत्वं रघूत्तम} %9-16

\twolineshloka
{अष्टावक्रं मुनिं दृष्ट्वा कदाचिदहसं पुरा}
{क्रुद्धोऽसावाह दुष्ट त्वं राक्षसो भव दुर्मते} %9-17

\twolineshloka
{अष्टावक्रः पुनः प्राह वन्दितो मे दयापरः}
{शापस्यान्तं च मे प्राह तपसा द्योतितप्रभः} %9-18

\twolineshloka
{त्रेतायुगे दाशरथिर्भूत्वा नारायणः स्वयम्}
{आगमिष्यति ते बाहू छिद्येते योजनायतौ} %9-19

\twolineshloka
{तेन शापाद्विनिर्मुक्तो भविष्यसि यथा पुरा}
{इति शप्तोऽहमद्राक्षं राक्षसीं तनुमात्मनः} %9-20

\twolineshloka
{कदाचिद्देवराजानमभ्यद्रवमहं रुषा}
{सोऽपि वज्रेण मां राम शिरोदेशेऽभ्यताडयत्} %9-21

\twolineshloka
{तदा शिरो गतं कुक्षिं पादौ च रघुनन्दन}
{ब्रह्मदत्तवरान्मृत्युर्नाभून्मे वज्रताडनात्} %9-22

\twolineshloka
{मुखाभावे कथं जीवेदयमित्यमराधिपम्}
{ऊचुः सर्वे दयाविष्टा मां विलोक्याऽऽस्यवर्जितम्} %9-23

\twolineshloka
{ततो मां प्राह मघवा जठरे ते मुखं भवेत्}
{बाहू ते योजनायामौ भविष्यत इतो व्रज} %9-24

\twolineshloka
{इत्युक्तोऽत्र वसन्नित्यं बाहुभ्यां वनगोचरान्}
{भक्षयाम्यधुना बाहू खण्डितौ मे त्वयाऽनघ} %9-25

\twolineshloka
{इतः परं मां श्वभ्रास्ये निक्षिपाग्नीन्धनावृते}
{अग्निना दह्यमानोऽहं त्वया रघुकुलोत्तम} %9-26

\onelineshloka
{पूर्वरूपमनुप्राप्य भार्यामार्गं वदामि ते} %9-27


\threelineshloka
{इत्युक्ते लक्ष्मणेनाशु श्वभ्रं निर्मित्य तत्र तम्}
{निक्षिप्य प्रादहत्काष्ठैस्ततो देहात्समुत्थितः}
{कन्दर्पसदृशाकारः सर्वाभरणभूषितः} %9-28

\twolineshloka
{रामं प्रदक्षिणं कृत्वा साष्टाङ्गं प्रणिपत्य च}
{कृताञ्जलिरुवाचेदं भक्तिगद्गदया गिरा} %9-29

\uvacha{गन्धर्व उवाच}

\twolineshloka
{स्तोतुमुत्सहते मेऽद्य मनो रामातिसम्भ्रमात्}
{त्वामनन्तमनाद्यन्तं मनोवाचामगोचरम्} %9-30

\threelineshloka
{सूक्ष्मं ते रूपमव्यक्तं देहद्वयविलक्षणम्}
{दृग्रूपमितरत्सर्वं दृश्यं जडमनात्मकम्}
{तत्कथं त्वां विजानीयाद्व्यतिरिक्तं मनः प्रभो} %9-31

\twolineshloka
{बुद्ध्यात्माभासयोरैक्यं जीव इत्यभिधीयते}
{बुद्ध्यादि साक्षी ब्रह्मैव तस्मिन्निर्विषयेऽखिलम्} %9-32

\twolineshloka
{आरोप्यतेऽज्ञानवशान्निर्विकारेऽखिलात्मनि}
{हिरण्यगर्भस्ते सूक्ष्मं देहं स्थूलं विराट् स्मृतम्} %9-33

\twolineshloka
{भावनाविषयो राम सूक्ष्मं ते ध्यातृमङ्गलम्}
{भूतं भव्यं भविष्यच्च यत्रेदं दृश्यते जगत्} %9-34

\twolineshloka
{स्थूलेऽण्डकोशे देहे ते महदादिभिरावृते}
{सप्तभिरुत्तरगुणैर्वैराजो धारणाश्रयः} %9-35

\twolineshloka
{त्वमेव सर्वकैवल्यं लोकास्तेऽवयवाः स्मृताः}
{पातालं ते पादमूलं पार्ष्णिस्तव महातलम्} %9-36

\twolineshloka
{रसातलं ते गुल्फौ तु तलातलमितीर्यते}
{जानुनी सुतलं राम ऊरू ते वितलं तथा} %9-37

\twolineshloka
{अतलं च मही राम जघनं नाभिगं नभः}
{उरःस्थलं ते ज्योतींषि ग्रीवा ते मह उच्यते} %9-38

\twolineshloka
{वदनं जनलोकस्ते तपस्ते शङ्खदेशगम्}
{सत्यलोको रघुश्रेष्ठ शीर्षण्यास्ते सदा प्रभो} %9-39

\twolineshloka
{इन्द्रादयो लोकपाला बाहवस्ते दिशः श्रुती}
{अश्विनौ नासिके राम वक्त्रं तेऽग्निरुदाहृतः} %9-40

\twolineshloka
{चक्षुस्ते सविता राम मनश्चन्द्र उदाहृतः}
{भ्रूभङ्ग एव कालस्ते बुद्धिस्ते वाक्पतिर्भवेत्} %9-41

\twolineshloka
{रुद्रोऽहङ्काररूपस्ते वाचश्छन्दांसि तेऽव्यय}
{यमस्ते दंष्ट्रदेशस्थो नक्षत्राणि द्विजालयः} %9-42

\twolineshloka
{हासो मोहकरी माया सृष्टिस्तेऽपाङ्गमोक्षणम्}
{धर्मः पुरस्तेऽधर्मश्च पृष्ठभाग उदीरितः} %9-43

\twolineshloka
{निमिषोन्मेषणे रात्रिर्दिवा चैव रघूत्तम}
{समुद्राः सप्त ते कुक्षिर्नाड्यो नद्यस्तव प्रभो} %9-44

\twolineshloka
{रोमाणि वृक्षौषधयो रेतो वृष्टिस्तव प्रभो}
{महिमा ज्ञानशक्तिस्ते एवं स्थूलं वपुस्तव} %9-45

\twolineshloka
{यदस्मिन् स्थूलरूपे ते मनः सन्धार्यते नरैः}
{अनायासेन मुक्तिः स्यादतोऽन्यन्नहि किञ्चन} %9-46

\twolineshloka
{अतोऽहं राम रूपं ते स्थूलमेवानुभावये}
{यस्मिन् ध्याते प्रेमरसः सरोमपुलको भवेत्} %9-47

\twolineshloka
{तदैव मुक्तिः स्याद्राम यदा ते स्थूलभावकः}
{तदप्यास्तां तवैवाहमेतद्रूपं विचिन्तये} %9-48

\twolineshloka
{धनुर्बाणधरं श्यामं जटावल्कलभूषितम्}
{अपीच्यवयसं सीतां विचिन्वन्तं सलक्ष्मणम्} %9-49

\onelineshloka
{इदमेव सदा मे स्यान्मानसे रघुनन्दन} %9-50


\threelineshloka
{सर्वज्ञः शङ्करः साक्षात्पार्वत्या सहितः सदा}
{त्वद्रूपमेव सततं ध्यायन्नास्ते रघूत्तम}
{मुमूर्षूणां तदा काश्यां तारकं ब्रह्मवाचकम्} %9-51

\twolineshloka
{रामरामेत्युपदिशन् सदा सन्तुष्टमानसः}
{अतस्त्वं जानकीनाथ परमात्मा सुनिश्चितः} %9-52

\twolineshloka
{सर्वे ते मायया मूढास्त्वां न जानन्ति तत्त्वतः}
{नमस्ते रामभद्राय वेधसे परमात्मने} %9-53

\twolineshloka
{अयोध्याधिपते तुभ्यं नमः सौमित्रिसेवित}
{त्राहि त्राहि जगन्नाथ मां माया नावृणोतु ते} %9-54

\uvacha{श्रीराम उवाच}

\twolineshloka
{तुष्टोऽहं देवगन्धर्व भक्त्या स्तुत्या च तेऽनघ}
{याहि मे परमं स्थानं योगिगम्यं सनातनम्} %9-55

\fourlineindentedshloka
{जपन्ति ये नित्यमनन्यबुद्ध्या}
{भक्त्या त्वदुक्तं स्तवमागमोक्तम्}
{तेऽज्ञानसम्भूतभवं विहाय}
{मां यान्ति नित्यानुभवानुमेयम्} %9-56

{॥इति श्रीमदध्यात्मरामयणे उमामहेश्वरसंवादे
अरण्यकाण्डे नवमः सर्गः॥९॥
}
%%%%%%%%%%%%%%%%%%%%



\sect{दशमः सर्गः}

\uvacha{श्रीमहादेव उवाच}

\twolineshloka
{लब्ध्वा वरं स गन्धर्वः प्रयास्यन् राममब्रवीत्}
{शबर्यास्ते पुरोभागे आश्रमे रघुनन्दन} %10-1

\twolineshloka
{भक्त्या त्वत्पादकमले भक्तिमार्गविशारदा}
{तां प्रयाहि महाभाग सर्वं ते कथयिष्यति} %10-2

\twolineshloka
{इत्युक्त्वा प्रययौ सोऽपि विमानेनार्कवर्चसा}
{विष्णोः पदं रामनामस्मरणे फलमीदृशम्} %10-3

\twolineshloka
{त्यक्त्वा तद्विपिनं घोरं सिंहव्याघ्रादिदूषितम्}
{शनैरथाश्रमपदं शबर्या रघुनन्दनः} %10-4

\twolineshloka
{शबरी राममालोक्य लक्ष्मणेन समन्वितम्}
{आयान्तमाराद्धर्षेण प्रत्युत्थायाचिरेण सा} %10-5

\twolineshloka
{पतित्वा पादयोरग्रे हर्षपूर्णाश्रुलोचना}
{स्वागतेनाभिनन्द्याथ स्वासने सन्न्यवेशयत्} %10-6

\twolineshloka
{रामलक्ष्मणयोः सम्यक्पादौ प्रक्षाल्य भक्तितः}
{तज्जलेनाभिषिच्याङ्गमथार्घ्यादिभिरादृता} %10-7

\twolineshloka
{सम्पूज्य विधिवद्रामं ससौमित्रिं सपर्यया}
{सङ्गृहीतानि दिव्यानि रामार्थं शबरी मुदा} %10-8

\twolineshloka
{फलान्यमृतकल्पानि ददौ रामाय भक्तितः}
{पादौ सम्पूज्य कुसुमैः सुगन्धैः सानुलेपनैः} %10-9

\twolineshloka
{कृतातिथ्यं रघुश्रेष्ठमुपविष्टं सहानुजम्}
{शबरी भक्तिसम्पन्ना प्राञ्जलिर्वाक्यमब्रवीत्} %10-10

\twolineshloka
{अत्राश्रमे रघुश्रेष्ठ गुरवो मे महर्षयः}
{स्थिताः शुश्रूषणं तेषां कुर्वती समुपस्थिता} %10-11

\twolineshloka
{बहुवर्षसहस्राणि गतास्ते ब्रह्मणः पदम्}
{गमिष्यन्तोऽब्रुवन्मां त्वं वसात्रैव समाहिता} %10-12

\twolineshloka
{रामो दाशरथिर्जातः परमात्मा सनातनः}
{राक्षसानां वधार्थाय ऋषीणां रक्षणाय च} %10-13

\twolineshloka
{आगमिष्यति चैकाग्रध्याननिष्ठा स्थिरा भव}
{इदानीं चित्रकूटाद्रावाश्रमे वसति प्रभुः} %10-14

\twolineshloka
{यावदागमनं तस्य तावद्रक्ष कलेवरम्}
{दृष्ट्वैव राघवं दग्ध्वा देहं यास्यसि तत्पदम्} %10-15

\twolineshloka
{तथैवाकरवं राम त्वद्ध्यानैकपरायणा}
{प्रतीक्ष्यागमनं तेऽद्य सफलं गुरुभाषितम्} %10-16

\twolineshloka
{तव सन्दर्शनं राम गुरूणामपि मे न हि}
{योषिन्मूढाऽप्रमेयात्मन् हीनजातिसमुद्भवा} %10-17

\twolineshloka
{तव दासस्य दासानां शतसङ्ख्योत्तरस्य वा}
{दासीत्वे नाधिकारोऽस्ति कुतः साक्षात्तवैव हि} %10-18

\twolineshloka
{कथं रामाद्य मे दृष्टस्त्वं मनोवागगोचरः}
{स्तोतुं न जाने देवेश किं करोमि प्रसीद मे} %10-19

\uvacha{श्रीराम उवाच}

\twolineshloka
{पुंस्त्वे स्त्रीत्वे विशेषो वा जातिनामाश्रमादयः}
{न कारणं मद्भजने भक्तिरेव हि कारणम्} %10-20

\twolineshloka
{यज्ञदानतपोभिर्वा वेदाध्ययनकर्मभिः}
{नैव द्रष्टुमहं शक्यो मद्भक्तिविमुखैः सदा} %10-21

\onelineshloka
{तस्माद्भामिनि सङ्क्षेपाद्वक्ष्येऽहं भक्तिसाधनम्} %10-22


\threelineshloka
{सतां सङ्गतिरेवात्र साधनं प्रथमं स्मृतम्}
{द्वितीयं मत्कथालापस्तृतीयं मद्गुणेरणम्}
{व्याख्यातृत्वं मद्वचसां चतुर्थं साधनं भवेत्} %10-23

\twolineshloka
{आचार्योपासनं भद्रे सद्बुद्ध्याऽमायया सदा}
{पञ्चमं पुण्यशीलत्वं यमादि नियमादि च} %10-24

\twolineshloka
{निष्ठा मत्पूजने नित्यं षष्ठं साधनमीरितम्}
{मम मन्त्रोपासकत्वं साङ्गं सप्तममुच्यते} %10-25

\twolineshloka
{मद्भक्तेष्वधिका पूजा सर्वभूतेषु मन्मतिः}
{बाह्यार्थेषु विरागित्वं शमादिसहितं तथा} %10-26

\twolineshloka
{अष्टमं नवमं तत्त्वविचारो मम भामिनि}
{एवं नवविधा भक्तिः साधनं यस्य कस्य वा} %10-27

\twolineshloka
{स्त्रियो वा पुरुषस्यापि तिर्यग्योनिगतस्य वा}
{भक्तिः सञ्जायते प्रेमलक्षणा शुभलक्षणे} %10-28

\twolineshloka
{भक्तौ सञ्जातमात्रायां मत्तत्त्वानुभवस्तदा}
{ममानुभवसिद्धस्य मुक्तिस्तत्रैव जन्मनि} %10-29

\twolineshloka
{स्यात्तस्मात्कारणं भक्तिर्मोक्षस्येति सुनिश्चितम्}
{प्रथमं साधनं यस्य भवेत्तस्य क्रमेण तु} %10-30

\twolineshloka
{भवेत्सर्वं ततो भक्तिर्मुक्तिरेव सुनिश्चितम्}
{यस्मान्मद्भक्तियुक्ता त्वं ततोऽहं त्वामुपस्थितः} %10-31

\twolineshloka
{इतो मद्दर्शनान्मुक्तिस्तव नास्त्यत्र संशयः}
{यदि जानासि मे ब्रूहि सीता कमललोचना} %10-32

\onelineshloka
{कुत्रास्ते केन वा नीता प्रिया मे प्रियदर्शना} %10-33


\uvacha{शबर्युवाच}

\twolineshloka
{देव जानासि सर्वज्ञ सर्वं त्वं विश्वभावन}
{तथाऽपि पृच्छसे यन्मां लोकाननुसृतः प्रभो} %10-34

\twolineshloka
{ततोऽहमभिधास्यामि सीता यत्राधुना स्थिता}
{रावणेन हृता सीता लङ्कायां वर्ततेऽधुना} %10-35

\twolineshloka
{इतः समीपे रामाऽऽस्ते पम्पानाम सरोवरम्}
{ऋष्यमूकगिरिर्नाम तत्समीपे महानगः} %10-36

\twolineshloka
{चतुर्भिर्मन्त्रिभिः सार्धं सुग्रीवो वानराधिपः}
{भीतभीतः सदा यत्र तिष्ठत्यतुलविक्रमः} %10-37

\twolineshloka
{वालिनश्च भयाद् भ्रातुस्तदगम्यमृषेर्भयात्}
{वालिनस्तत्र गच्छ त्वं तेन सख्यं कुरु प्रभो} %10-38

\twolineshloka
{सुग्रीवेण स सर्वं ते कार्यं सम्पादयिष्यति}
{अहमग्निं प्रवेक्ष्यामि तवाग्रे रघुनन्दन} %10-39

\twolineshloka
{मुहूर्तं तिष्ठ राजेन्द्र यावद्दग्ध्वा कलेवरम्}
{यास्यामि भगवन् राम तव विष्णोः परं पदम्} %10-40

\threelineshloka
{इति रामं समामन्त्र्य प्रविवेश हुताशनम्}
{क्षणान्निर्धूय सकलमविद्याकृतबन्धनम्}
{रामप्रसादाच्छबरी मोक्षं प्रापातिदुर्लभम्} %10-41

\twolineshloka
{किं दुर्लभं जगन्नाथे श्रीरामे भक्तवत्सले}
{प्रसन्नेऽधमजन्माऽपि शबरी मुक्तिमाप सा} %10-42

\twolineshloka
{किं पुनर्ब्राह्मणा मुख्याः पुण्याः श्रीरामचिन्तकाः}
{मुक्तिं यान्तीति तद्भक्तिर्मुक्तिरेव न संशयः} %10-43

\fourlineindentedshloka
{भक्तिर्मुक्तिविधायिनी भगवतः श्रीरामचन्द्रस्य हे}
{लोकाः कामदुघाङ्घ्रिपद्मयुगलं सेवध्वमत्युत्सुकाः}
{नानाज्ञानविशेषमन्त्रविततिं त्यक्त्वा सुदूरे भृशम्}
{रामं श्यामतनुं स्मरारिहृदये भान्तं भजध्वं बुधाः} %10-44

{॥इति श्रीमदध्यात्मरामायणे उमामहेश्वरसंवादे
अरण्यकाण्डे दशमः सर्गः॥१०॥
}
%%%%%%%%%%%%%%%%%%%%

इति श्रीमदध्यात्मरामायणे अरण्यकाण्डः समाप्तः॥
