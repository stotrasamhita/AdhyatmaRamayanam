% !TeX program = XeLaTeX
% !TeX root = AdhyatmaRamayanaBook-kindle.tex
\chapt{युद्धकाण्डः}


\sect{प्रथमः सर्गः}

\uvacha{श्रीमहादेव उवाच}

\twolineshloka
{यथावद्भाषितं वाक्यं श्रुत्वा रामो हनूमतः}
{उवाचानन्तरं वाक्यं हर्षेण महताऽऽवृतः} %1-1

\twolineshloka
{कार्यं कृतं हनुमता देवैरपि सुदुष्करम्}
{मनसाऽपि यदन्येन स्मर्तुं शक्यं न भूतले} %1-2

\twolineshloka
{शतयोजनविस्तीर्णं लङ्घयेत्कः पयोनिधिम्}
{लङ्कां च राक्षसैर्गुप्तां को वा धर्षयितुं क्षमः} %1-3

\twolineshloka
{भृत्यकार्यं हनुमता कृतं सर्वमशेषतः}
{सुग्रीवस्येदृशो लोके न भूतो न भविष्यति} %1-4

\twolineshloka
{अहं च रघुवंशश्च लक्ष्मणश्च कपीश्वरः}
{जानक्या दर्शनेनाद्य रक्षिताः स्मो हनूमता} %1-5

\twolineshloka
{सर्वथा सुकृतं कार्यं जानक्याः परिमार्गणम्}
{समुद्रं मनसा स्मृत्वा सीदतीव मनो मम} %1-6

\twolineshloka
{कथं नक्रझषाकीर्णं समुद्रं शतयोजनम्}
{लङ्घयित्वा रिपुं हन्यां कथं द्रक्ष्यामि जानकीम्} %1-7

\twolineshloka
{श्रुत्वा तु रामवचनं सुग्रीवः प्राह राघवम्}
{समुद्रं लङ्घयिष्यामो महानक्रझषाकुलम्} %1-8

\twolineshloka
{लङ्कां च विधमिष्यामो हनिष्यामोऽद्य रावणम्}
{चिन्तां त्यज रघुश्रेष्ठ चिन्ता कार्यविनाशिनी} %1-9

\twolineshloka
{एतान् पश्य महासत्त्वान् शूरान् वानरपुङ्गवान्}
{त्वत्प्रियार्थं समुद्युक्तान् प्रवेष्टुमपि पावकम्} %1-10

\twolineshloka
{समुद्रतरणे बुद्धिं कुरुष्व प्रथमं ततः}
{दृष्ट्वा लङ्कां दशग्रीवो हत इत्येव मन्महे} %1-11

\twolineshloka
{नहि पश्याम्यहं कञ्चित्त्रिषु लोकेषु राघव}
{गृहीतधनुषो यस्ते तिष्ठेदभिमुखो रणे} %1-12

\twolineshloka
{सर्वथा नो जयो राम भविष्यति न संशयः}
{निमित्तानि च पश्यामि तथा भूतानि सर्वशः} %1-13

\twolineshloka
{सुग्रीववचनं श्रुत्वा भक्तिवीर्यसमन्वितम्}
{अङ्गीकृत्याब्रवीद्रामो हनूमन्तं पुरःस्थितम्} %1-14

\twolineshloka
{येन केन प्रकारेण लङ्घयामो महार्णवम्}
{लङ्कास्वरूपं मे ब्रूहि दुःसाध्यं देवदानवैः} %1-15

\twolineshloka
{ज्ञात्वा तस्य प्रतीकारं करिष्यामि कपीश्वर}
{श्रुत्वा रामस्य वचनं हनूमान् विनयान्वितः} %1-16

\twolineshloka
{उवाच प्राञ्जलिर्देव यथा दृष्टं ब्रवीमि ते}
{लङ्का दिव्या पुरी देव त्रिकूटशिखरे स्थिता} %1-17

\twolineshloka
{स्वर्णप्राकारसहिता स्वर्णाट्टालकसंयुता}
{परिखाभिः परिवृता पूर्णाभिर्निर्मलोदकैः} %1-18

\twolineshloka
{नानोपवनशोभाढ्या दिव्यवापीभिरावृता}
{गृहैर्विचित्रशोभाढ्यैर्मणिस्तम्भमयैः शुभैः} %1-19

\twolineshloka
{पश्चिमद्वारमासाद्य गजवाहाः सहस्रशः}
{उत्तरे द्वारि तिष्ठन्ति साश्ववाहाः सपत्तयः} %1-20

\twolineshloka
{तिष्ठन्त्यर्बुदसङ्ख्याकाः प्राच्यामपि तथैव च}
{रक्षिणो राक्षसा वीरा द्वारं दक्षिणमाश्रिताः} %1-21

\twolineshloka
{मध्यकक्षेऽप्यसङ्ख्याता गजाश्वरथपत्तयः}
{रक्षयन्ति सदा लङ्कां नानास्त्रकुशलाः प्रभो} %1-22

\twolineshloka
{सङ्क्रमैर्विविधैर्लङ्का शतघ्नीभिश्च संयुता}
{एवं स्थितेऽपि देवेश शृणु मे तत्र चेष्टितम्} %1-23

\twolineshloka
{दशाननबलौघस्य चतुर्थांशो मया हतः}
{दग्ध्वा लङ्कां पुरीं स्वर्णप्रासादो धर्षितो मया} %1-24

\twolineshloka
{शतघ्न्यः सङ्क्रमाश्चैव नाशिता मे रघूत्तम}
{देव त्वद्दर्शनादेव लङ्का भस्मीकृता भवेत्} %1-25

\twolineshloka
{प्रस्थानं कुरु देवेश गच्छामो लवणाम्बुधेः}
{तीरं सह महावीरैर्वानरौघैः समन्ततः} %1-26

\twolineshloka
{श्रुत्वा हनूमतो वाक्यमुवाच रघुनन्दनः}
{सुग्रीव सैनिकान् सर्वान् प्रस्थानायाभिनोदय} %1-27

\twolineshloka
{इदानीमेव विजयो मुहूर्तः परिवर्तते}
{अस्मिन्मुहूर्ते गत्वाऽहं लङ्कां राक्षससङ्कुलाम्} %1-28

\twolineshloka
{सप्राकारां सुदुर्धर्षां नाशयामि सरावणाम्}
{आनेष्यामि च सीतां मे दक्षिणाक्षि स्फुरत्यधः} %1-29

\twolineshloka
{प्रयातु वाहिनी सर्वा वानराणां तरस्विनाम्}
{रक्षन्तु यूथपाः सेनामग्रे पृष्ठे च पार्श्वयोः} %1-30

\twolineshloka
{हनूमन्तमथारुह्य गच्छाम्यग्रेऽङ्गदं ततः}
{आरुह्य लक्ष्मणो यातु सुग्रीव त्वं मया सह} %1-31

\twolineshloka
{गजो गवाक्षो गवयो मैन्दो द्विविद एव च}
{नलो नीलः सुषेणश्च जाम्बवांश्च तथाऽपरे} %1-32

\twolineshloka
{सर्वे गच्छन्तु सर्वत्र सेनायाः शत्रुघातिनः}
{इत्याज्ञाप्य हरीन् रामः प्रतस्थे सहलक्ष्मणः} %1-33

\twolineshloka
{सुग्रीवसहितो हर्षात्सेनामध्यगतो विभुः}
{वारणेन्द्रनिभाः सर्वे वानराः कामरूपिणः} %1-34

\twolineshloka
{क्ष्वेलन्तः परिगर्जन्तो जग्मुस्ते दक्षिणां दिशम्}
{भक्षयन्तो ययुः सर्वे फलानि च मधूनि च} %1-35

\twolineshloka
{ब्रुवन्तो राघवस्याग्रे हनिष्यामोऽद्य रावणम्}
{एवं ते वानरश्रेष्ठा गच्छन्त्यतुलविक्रमाः} %1-36

\twolineshloka
{हरिभ्यामुह्यमानौ तौ शुशुभाते रघूत्तमौ}
{नक्षत्रैः सेवितौ यद्वच्चन्द्रसूर्याविवाम्बरे} %1-37

\twolineshloka
{आवृत्य पृथिवीं कृत्स्नां जगाम महती चमूः}
{प्रस्फोटयन्तः पुच्छाग्रानुद्वहन्तश्च पादपान्} %1-38

\twolineshloka
{शैलानारोहयन्तश्च जग्मुर्मारुतवेगतः}
{असङ्ख्याताश्च सर्वत्र वानराः परिपूरिताः} %1-39

\twolineshloka
{हृष्टास्ते जग्मुरत्यर्थं रामेण परिपालिताः}
{गता चमूर्दिवारात्रं क्वचिन्नासज्जत क्षणम्} %1-40

\twolineshloka
{काननानि विचित्राणि पश्यन्मलयसह्ययोः}
{ते सह्यं समतिक्रम्य मलयं च तथा गिरिम्} %1-41

\twolineshloka
{आययुश्चानुपूर्व्येण समुद्रं भीमनिःस्वनम्}
{अवतीर्य हनूमन्तं रामः सुग्रीवसंयुतः} %1-42

\twolineshloka
{सलिलाभ्याशमासाद्य रामो वचनमब्रवीत्}
{आगताः स्मो वयं सर्वे समुद्रं मकरालयम्} %1-43

\twolineshloka
{इतो गन्तुमशक्यं नो निरुपायेन वानराः}
{अत्र सेनानिवेशोऽस्तु मन्त्रयामोऽस्य तारणे} %1-44

\twolineshloka
{श्रुत्वा रामस्य वचनं सुग्रीवः सागरान्तिके}
{सेनां न्यवेशयत् क्षिप्रं रक्षितां कपिकुञ्जरैः} %1-45

\twolineshloka
{ते पश्यन्तो विषेदुस्तं सागरं भीमदर्शनम्}
{महोन्नततरङ्गाढ्यं भीमनक्रभयङ्करम्} %1-46

\twolineshloka
{अगाधं गगनाकारं सागरं वीक्ष्य दुःखिताः}
{तरिष्यामः कथं घोरं सागरं वरुणालयम्} %1-47

\twolineshloka
{हन्तव्योऽस्माभिरद्यैव रावणो राक्षसाधमः}
{इति चिन्ताकुलाः सर्वे रामपार्श्वे व्यवस्थिताः} %1-48

\twolineshloka
{रामः सीतामनुस्मृत्य दुःखेन महताऽऽवृतः}
{विलप्य जानकीं सीतां बहुधा कार्यमानुषः} %1-49

\twolineshloka
{अद्वितीयश्चिदात्मैकः परमात्मा सनातनः}
{यस्तु जानाति रामस्य स्वरूपं तत्त्वतो जनः} %1-50

\twolineshloka
{तं न स्पृशति दुःखादि किमुतानन्दमव्ययम्}
{दुःखहर्षभयक्रोधलोभमोहमदादयः} %1-51

\twolineshloka
{अज्ञानलिङ्गान्येतानि कुतः सन्ति चिदात्मनि}
{देहाभिमानिनो दुःखं न देहस्य चिदात्मनः} %1-52

\threelineshloka
{सम्प्रसादे द्वयाभावात्सुखमात्रं हि दृश्यते}
{बुद्ध्याद्यभावात्संशुद्धे दुःखं तत्र न दृश्यते}
{अतो दुःखादिकं सर्वं बुद्धेरेव न संशयः} %1-53

\fourlineindentedshloka
{रामः परात्मा पुरुषः पुराणो}
{नित्योदितो नित्यसुखो निरीहः}
{तथाऽपि मायागुणसङ्गतोऽसौ}
{सुखीव दुःखीव विभाव्यतेऽबुधैः} %1-54

\iti{युद्धकाण्डे}{प्रथमः}
%%%%%%%%%%%%%%%%%%%%



\sect{द्वितीयः सर्गः}

\uvacha{श्रीमहादेव उवाच}

\twolineshloka
{लङ्कायां रावणो दृष्ट्वा कृतं कर्म हनूमता}
{दुष्करं दैवतैर्वाऽपि ह्रिया किञ्चिदवाङ्मुखः} %2-1

\twolineshloka
{आहूय मन्त्रिणः सर्वानिदं वचनमब्रवीत्}
{हनूमता कृतं कर्म भवद्भिर्दृष्टमेव तत्} %2-2

\twolineshloka
{प्रविश्य लङ्कां दुर्धर्षां दृष्ट्वा सीतां दुरासदाम्}
{हत्वा च राक्षसान् वीरानक्षं मन्दोदरीसुतम्} %2-3

\twolineshloka
{दग्ध्वा लङ्कामशेषेण लङ्घयित्वा च सागरम्}
{युष्मान् सर्वानतिक्रम्य स्वस्थोऽगात्पुनरेव सः} %2-4

\twolineshloka
{किं कर्तव्यमितोऽस्माभिर्यूयं मन्त्रविशारदाः}
{मन्त्रयध्वं प्रयत्नेन यत्कृतं मे हितं भवेत्} %2-5

\twolineshloka
{रावणस्य वचः श्रुत्वा राक्षसास्तमथाब्रुवन्}
{देव शङ्का कुतो रामात्तव लोकजितो रणे} %2-6

\twolineshloka
{इन्द्रस्तु बद्ध्वा निक्षिप्तः पुत्रेण तव पत्तने}
{जित्वा कुबेरमानीय पुष्पकं भुज्यते त्वया} %2-7

\twolineshloka
{यमो जितः कालदण्डाद्भयं नाभूत्तव प्रभो}
{वरुणो हुङ्कृतेनैव जितः सर्वेऽपि राक्षसाः} %2-8

\twolineshloka
{मयो महासुरो भीत्या कन्यां दत्त्वा स्वयं तव}
{त्वद्वशे वर्ततेऽद्यापि किमुतान्ये महासुराः} %2-9

\twolineshloka
{हनूमद्धर्षणं यत्तु तदवज्ञाकृतं च नः}
{वानरोऽयं किमस्माकमस्मिन् पौरुषदर्शने} %2-10

\twolineshloka
{इत्युपेक्षितमस्माभिर्धर्षणं तेन किं भवेत्}
{वयं प्रमत्ताः किं तेन वञ्चिताः स्मो हनूमता} %2-11

\twolineshloka
{जानीमो यदि तं सर्वे कथं जीवन् गमिष्यति}
{आज्ञापय जगत्कृत्स्नमवानरममानुषम्} %2-12

\twolineshloka
{कृत्वाऽऽयास्यामहे सर्वे प्रत्येकं वा नियोजय}
{कुम्भकर्णस्तदा प्राह रावणं राक्षसेश्वरम्} %2-13

\twolineshloka
{आरब्धं यत्त्वया कर्म स्वात्मनाशाय केवलम्}
{न दृष्टोऽसि तदा भाग्यात्त्वं रामेण महात्मना} %2-14

\twolineshloka
{यदि पश्यति रामस्त्वां जीवन्नायासि रावण}
{रामो न मानुषो देवः साक्षान्नारायणोऽव्ययः} %2-15

\twolineshloka
{सीता भगवती लक्ष्मी रामपत्नी यशस्विनी}
{राक्षसानां विनाशाय त्वयाऽऽनीता सुमध्यमा} %2-16

\twolineshloka
{विषपिण्डमिवागीर्य महामीनो यथा तथा}
{आनीता जानकी पश्चात्त्वया किं वा भविष्यति} %2-17

\twolineshloka
{यद्यप्यनुचितं कर्म त्वया कृतमजानता}
{सर्वं समं करिष्यामि स्वस्थचित्तो भव प्रभो} %2-18

\threelineshloka
{कुम्भकर्णवचः श्रुत्वा वाक्यमिन्द्रजिदब्रवीत्}
{देहि देव ममानुज्ञां हत्वा रामं सलक्ष्मणम्}
{सुग्रीवं वानरांश्चैव पुनर्यास्यामि तेऽन्तिकम्} %2-19

\fourlineindentedshloka
{तत्राऽऽगतो भागवतप्रधानो}
{विभीषणो बुद्धिमतां वरिष्ठः}
{श्रीरामपादद्वय एकतानः}
{प्रणम्य देवारिमुपोपविष्टः} %2-20

\fourlineindentedshloka
{विलोक्य कुम्भश्रवणादिदैत्यान्}
{मत्तप्रमत्तानतिविस्मयेन}
{विलोक्य कामातुरमप्रमत्तो}
{दशाननं प्राह विशुद्धबुद्धिः} %2-21

\fourlineindentedshloka
{न कुम्भकर्णेन्द्रजितौ च राजन्}
{तथा महापार्श्वमहोदरौ तौ}
{निकुम्भकुम्भौ च तथाऽतिकायः}
{स्थातुं न शक्ता युधि राघवस्य} %2-22

\fourlineindentedshloka
{सीताभिधानेन महाग्रहेण}
{ग्रस्तोऽसि राजन् न च ते विमोक्षः}
{तामेव सत्कृत्य महाधनेन}
{दत्त्वाऽभिरामाय सुखी भव त्वम्} %2-23

\fourlineindentedshloka
{यावन्न रामस्य शिताः शिलीमुखा}
{लङ्कामभिव्याप्य शिरांसि रक्षसाम्}
{छिन्दन्ति तावद्रघुनायकस्य भो-}
{स्तां जानकीं त्वं प्रतिदातुमर्हसि} %2-24

\fourlineindentedshloka
{यावन्नगाभाः कपयो महाबला}
{हरीन्द्रतुल्या नखदंष्ट्रयोधिनः}
{लङ्कां समाक्रम्य विनाशयन्ति ते}
{तावद्द्रुतं देहि रघूत्तमाय ताम्} %2-25

\fourlineindentedshloka
{जीवन्न रामेण विमोक्ष्यसे त्वम्}
{गुप्तः सुरेन्द्रैरपि शङ्करेण}
{न देवराजाङ्कगतो न मृत्योः}
{पाताललोकानपि सम्प्रविष्टः} %2-26

\twolineshloka
{शुभं हितं पवित्रं च विभीषणवचः खलः}
{प्रतिजग्राह नैवासौ म्रियमाण इवौषधम्} %2-27

\twolineshloka
{कालेन नोदितो दैत्यो विभीषणमथाब्रवीत्}
{मद्दत्तभोगैः पुष्टाङ्गो मत्समीपे वसन्नपि} %2-28

\twolineshloka
{प्रतीपमाचरत्येष ममैव हितकारिणः}
{मित्रभावेन शत्रुर्मे जातो नास्त्यत्र संशयः} %2-29

\twolineshloka
{अनार्येण कृतघ्नेन सङ्गतिर्मे न युज्यते}
{विनाशमभिकाङ्क्षन्ति ज्ञातीनां ज्ञातयः सदा} %2-30

\twolineshloka
{योऽन्यस्त्वेवंविधं ब्रूयाद्वाक्यमेकं निशाचरः}
{हन्मि तस्मिन् क्षणे एव धिक् त्वां रक्षःकुलाधमम्} %2-31

\twolineshloka
{रावणेनैवमुक्तः सन् परुषं स विभीषणः}
{उत्पपात सभामध्याद्गदापाणिर्महाबलः} %2-32

\threelineshloka
{चतुर्भिर्मन्त्रिभिः सार्धं गगनस्थोऽब्रवीद्वचः}
{क्रोधेन महताऽऽविष्टो रावणं दशकन्धरम्}
{मा विनाशमुपैहि त्वं प्रियवादिनमेव माम्} %2-33

\twolineshloka
{धिक्करोषि तथाऽपि त्वं ज्येष्ठो भ्राता पितुः समः}
{कालो राघवरूपेण जातो दशरथालये} %2-34

\twolineshloka
{काली सीताभिधानेन जाता जनकनन्दिनी}
{तावुभावागतावत्र भूर्मेभारापनुत्तये} %2-35

\twolineshloka
{तेनैव प्रेरितस्त्वं तु न शृणोषि हितं मम}
{श्रीरामः प्रकृतेः साक्षात्परस्तात्सर्वदा स्थितः} %2-36

\twolineshloka
{बहिरन्तश्च भूतानां समः सर्वत्र संस्थितः}
{नामरूपादिभेदेन तत्तन्मय इवामलः} %2-37

\twolineshloka
{यथा नानाप्रकारेषु वृक्षेष्वेको महानलः}
{तत्तदाकृतिभेदेन भिद्यतेऽज्ञानचक्षुषाम्} %2-38

\twolineshloka
{पञ्चकोशादिभेदेन तत्तन्मय इवाबभौ}
{नीलपीतादियोगेन निर्मलः स्फटिको यथा} %2-39

\twolineshloka
{स एव नित्यमुक्तोऽपि स्वमायागुणबिम्बितः}
{कालः प्रधानं पुरुषोऽव्यक्तं चेति चतुर्विधः} %2-40

\twolineshloka
{प्रधानपुरुषाभ्यां स जगत्कृत्स्नं सृजत्यजः}
{कालरूपेण कलनां जगतः कुरुतेऽव्ययः} %2-41

{कालरूपी स भगवान् रामरूपेण मायया} %2-42


\twolineshloka
{ब्रह्मणा प्रार्थितो देवस्त्वद्वधार्थमिहागतः}
{तदन्यथा कथं कुर्यात्सत्यसङ्कल्प ईश्वरः} %2-43

\twolineshloka
{हनिष्यति त्वां रामस्तु सपुत्रबलवाहनम्}
{हन्यमानं न शक्नोमि द्रष्टुं रामेण रावण} %2-44

\twolineshloka
{त्वां राक्षसकुलं कृत्स्नं ततो गच्छामि राघवम्}
{मयि याते सुखीभूत्वा रमस्व भवने चिरम्} %2-45

\fourlineindentedshloka
{विभीषणो रावणवाक्यतः क्षणा-}
{द्विसृज्य सर्वं सपरिच्छदं गृहम्}
{जगाम रामस्य पदारविन्दयोः}
{सेवाभिकाङ्क्षी परिपूर्णमानसः} %2-46

\iti{युद्धकाण्डे}{द्वितीयः}
%%%%%%%%%%%%%%%%%%%%



\sect{तृतीयः सर्गः}

\uvacha{श्रीमहादेव उवाच}

\twolineshloka
{विभीषणो महाभागश्चतुर्भिर्मन्त्रिभिः सह}
{आगत्य गगने रामसम्मुखे समवस्थितः} %3-1

\twolineshloka
{उच्चैरुवाच भोः स्वामिन् राम राजीवलोचन}
{रावणस्यानुजोऽहं ते दारहर्तुर्विभीषणः} %3-2

\twolineshloka
{नाम्ना भ्रात्रा निरस्तोऽहं त्वामेव शरणं गतः}
{हितमुक्तं मया देव तस्य चाविदितात्मनः} %3-3

\twolineshloka
{सीतां रामाय वैदेहीं प्रेषयेति पुनः पुनः}
{उक्तोऽपि न शृणोत्येव कालपाशवशं गतः} %3-4

\twolineshloka
{हन्तुं मां खड्गमादाय प्राद्रवद्राक्षसाधमः}
{ततोऽचिरेण सचिवैश्चतुर्भिः सहितो भयात्} %3-5

\twolineshloka
{त्वामेव भवमोक्षाय मुमुक्षुः शरणं गतः}
{विभीषणवचः श्रुत्वा सुग्रीवो वाक्यमब्रवीत्} %3-6

\twolineshloka
{विश्वासार्हो न ते राम मायावी राक्षसाधमः}
{सीताहर्तुर्विशेषेण रावणस्यानुजो बली} %3-7

\twolineshloka
{मन्त्रिभिः सायुधैरस्मान् विवरे निहनिष्यति}
{तदाज्ञापय मे देव वानरैर्हन्यतामयम्} %3-8

\twolineshloka
{ममैवं भाति मे राम बुद्ध्या किं निश्चितं वद}
{श्रुत्वा सुग्रीववचनं रामः सस्मितमब्रवीत्} %3-9

\twolineshloka
{यदीच्छामि कपिश्रेष्ठ लोकान् सर्वान् सहेश्वरान्}
{निमिषार्धेन संहन्यां सृजामि निमिषार्धतः} %3-10

\onelineshloka
{अतो मयाऽभयं दत्तं शीघ्रमानय राक्षसम्} %3-11

\twolineshloka
{सकृदेव प्रपन्नाय तवास्मीति च याचते}
{अभयं सर्वभूतेभ्यो ददाम्येतद्व्रतं मम} %3-12

\twolineshloka
{रामस्य वचनं श्रुत्वा सुग्रीवो हृष्टमानसः}
{विभीषणमथानाय्य दर्शयामास राघवम्} %3-13

\twolineshloka
{विभीषणस्तु साष्टाङ्गं प्रणिपत्य रघूत्तमम्}
{हर्षगद्गदया वाचा भक्त्या च परयान्वितः} %3-14

\twolineshloka
{रामं श्यामं विशालाक्षं प्रसन्नमुखपङ्कजम्}
{धनुर्बाणधरं शान्तं लक्ष्मणेन समन्वितम्} %3-15

\onelineshloka
{कृताञ्जलिपुटो भूत्वा स्तोतुं समुपचक्रमे} %3-16


\uvacha{विभीषण उवाच}

\twolineshloka
{नमस्ते राम राजेन्द्र नमः सीतामनोरम}
{नमस्ते चण्डकोदण्ड नमस्ते भक्तवत्सल} %3-17

\twolineshloka
{नमोऽनन्ताय शान्ताय रामायामिततेजसे}
{सुग्रीवमित्राय च ते रघूणां पतये नमः} %3-18

\twolineshloka
{जगदुत्पत्तिनाशानां कारणाय महात्मने}
{त्रैलोक्यगुरवेऽनादिगृहस्थाय नमो नमः} %3-19

\twolineshloka
{त्वमादिर्जगतां राम त्वमेव स्थितिकारणम्}
{त्वमन्ते निधनस्थानं स्वेच्छाचारस्त्वमेव हि} %3-20

\twolineshloka
{चराचराणां भूतानां बहिरन्तश्च राघव}
{व्याप्यव्यापकरूपेण भवान् भाति जगन्मयः} %3-21

\twolineshloka
{त्वन्मायया हृतज्ञाना नष्टात्मानो विचेतसः}
{गतागतं प्रपद्यन्ते पापपुण्यवशात्सदा} %3-22

\twolineshloka
{तावत्सत्यं जगद्भाति शुक्तिकारजतं यथा}
{यावन्न ज्ञायते ज्ञानं चेतसाऽनन्यगामिना} %3-23

\twolineshloka
{त्वदज्ञानात्सदा युक्ताः पुत्रदारगृहादिषु}
{रमन्ते विषयान् सर्वानन्ते दुःखप्रदान् विभो} %3-24

\twolineshloka
{त्वमिन्द्रोऽग्निर्यमो रक्षो वरुणश्च तथाऽनिलः}
{कुबेरश्च तथा रुद्रस्त्वमेव पुरुषोत्तम} %3-25

\twolineshloka
{त्वमणोरप्यणीयांश्च स्थूलात् स्थूलतरः प्रभो}
{त्वं पिता सर्वलोकानां माता धाता त्वमेव हि} %3-26

\twolineshloka
{आदिमध्यान्तरहितः परिपूर्णोऽच्युतोऽव्ययः}
{त्वं पाणिपादरहितश्चक्षुःश्रोत्रविवर्जितः} %3-27

\twolineshloka
{श्रोता द्रष्टा ग्रहीता च जवनस्त्वं खरान्तक}
{कोशेभ्यो व्यतिरिक्तस्त्वं निर्गुणो निरुपाश्रयः} %3-28

\twolineshloka
{निर्विकल्पो निर्विकारो निराकारो निरीश्वरः}
{षड्भावरहितोऽनादिः पुरुषः प्रकृतेः परः} %3-29

\twolineshloka
{मायया गृह्यमाणस्त्वं मनुष्य इव भाव्यसे}
{ज्ञात्वा त्वां निर्गुणमजं वैष्णवा मोक्षगामिनः} %3-30

\twolineshloka
{अहं त्वत्पादसद्भक्तिनिःश्रेणीं प्राप्य राघव}
{इच्छामि ज्ञानयोगाख्यं सौधमारोढुमीश्वर} %3-31

\twolineshloka
{नमः सीतापते राम नमः कारुणिकोत्तम}
{रावणारे नमस्तुभ्यं त्राहि मां भवसागरात्} %3-32

\twolineshloka
{ततः प्रसन्नः प्रोवाच श्रीरामो भक्तवत्सलः}
{वरं वृणीष्व भद्रं ते वाञ्छितं वरदोऽस्म्यहम्} %3-33

\uvacha{विभीषण उवाच}

\twolineshloka
{धन्योऽस्मि कृतकृत्योऽस्मि कृतकार्योऽस्मि राघव}
{त्वत्पाददर्शनादेव विमुक्तोऽस्मि न संशयः} %3-34

\twolineshloka
{नास्ति मत्सदृशो धन्यो नास्ति मत्सदृशः शुचिः}
{नास्ति मत्सदृशो लोके राम त्वन्मूर्तिदर्शनात्} %3-35

\twolineshloka
{कर्मबन्धविनाशाय त्वज्ज्ञानं भक्तिलक्षणम्}
{त्वद्ध्यानं परमार्थं च देहि मे रघुनन्दन} %3-36

\twolineshloka
{न याचे राम राजेन्द्र सुखं विषयसम्भवम्}
{त्वत्पादकमले सक्ता भक्तिरेव सदास्तु मे} %3-37

\twolineshloka
{ओमित्युक्त्वा पुनः प्रीतो रामः प्रोवाच राक्षसम्}
{शृणु वक्ष्यामि ते भद्रं रहस्यं मम निश्चितम्} %3-38

\twolineshloka
{मद्भक्तानां प्रशान्तानां योगिनां वीतरागिणाम्}
{हृदये सीतया नित्यं वसाम्यत्र न संशयः} %3-39

\twolineshloka
{तस्मात्त्वं सर्वदा शान्तः सर्वकल्मषवर्जितः}
{मां ध्यात्वा मोक्ष्यसे नित्यं घोरसंसारसागरात्} %3-40

\twolineshloka
{स्तोत्रमेतत्पठेद्यस्तु लिखेद्यः शृणुयादपि}
{मत्प्रीतये ममाभीष्टं सारूप्यं समवाप्नुयात्} %3-41

\twolineshloka
{इत्युक्त्वा लक्ष्मणं प्राह श्रीरामो भक्तभक्तिमान्}
{पश्यत्विदानीमेवैष मम सन्दर्शने फलम्} %3-42

\twolineshloka
{लङ्काराज्येऽभिषेक्ष्यामि जलमानय सागरात्}
{यावच्चन्द्रश्च सूर्यश्च यावत्तिष्ठति मेदिनी} %3-43

\twolineshloka
{यावन्मम कथा लोके तावद्राज्यं करोत्वसौ}
{इत्युक्त्वा लक्ष्मणेनाम्बु ह्यानाय्य कलशेन तम्} %3-44

\twolineshloka
{लङ्काराज्याधिपत्यार्थमभिषेकं रमापतिः}
{कारयामास सचिवैर्लक्ष्मणेन विशेषतः} %3-45

\twolineshloka
{साधु साध्विति ते सर्वे वानरास्तुष्टुवुर्भृशम्}
{सुग्रीवोऽपि परिष्वज्य विभीषणमथाब्रवीत्} %3-46

\threelineshloka
{विभीषण वयं सर्वे रामस्य परमात्मनः}
{किङ्करास्तत्र मुख्यस्त्वं भक्त्या रामपरिग्रहात्}
{रावणस्य विनाशे त्वं साहाय्यं कर्तुमर्हसि} %3-47

\uvacha{विभीषण उवाच}

\twolineshloka
{अहं कियान् सहायत्वे रामस्य परमात्मनः}
{किं तु दास्यं करिष्येऽहं भक्त्या शक्त्या ह्यमायया} %3-48

\twolineshloka
{दशग्रीवेण सन्दिष्टः शुको नाम महासुरः}
{संस्थितो ह्यम्बरे वाक्यं सुग्रीवमिदमब्रवीत्} %3-49

\twolineshloka
{त्वामाह रावणो राजा भ्रातरं राक्षसाधिपः}
{महाकुलप्रसूतस्त्वं राजाऽसि वनचारिणाम्} %3-50

\twolineshloka
{मम भ्रातृसमानस्त्वं तव नास्त्यर्थविप्लवः}
{अहं यदहरं भार्यां राजपुत्रस्य किं तव} %3-51

\twolineshloka
{किष्किन्धां याहि हरिभिर्लङ्का शक्या न दैवतैः}
{प्राप्तुं किं मानवैरल्पसत्त्वैर्वानरयूथपैः} %3-52

\twolineshloka
{तं प्रापयन्तं वचनं तूर्णमुत्प्लुत्य वानराः}
{प्रापद्यन्त तदा क्षिप्रं निहन्तुं दृढमुष्टिभिः} %3-53

\twolineshloka
{वानरैर्हन्यमानस्तु शुको राममथाब्रवीत्}
{न दूतान् घ्नन्ति राजेन्द्र वानरान् वारय प्रभो} %3-54

\twolineshloka
{रामः श्रुत्वा तदा वाक्यं शुकस्य परिदेवितम्}
{मा वधिष्टेति रामस्तान् वारयामास वानरान्} %3-55

\twolineshloka
{पुनरम्बरमासाद्य शुकः सुग्रीवमब्रवीत्}
{ब्रूहि राजन् दशग्रीवं किं वक्ष्यामि व्रजाम्यहम्} %3-56

\uvacha{सुग्रीव उवाच}

\twolineshloka
{यथा वाली मम भ्राता तथा त्वं राक्षसाधम}
{हन्तव्यस्त्वं मया यत्नात्सपुत्रबलवाहनः} %3-57

\twolineshloka
{ब्रूहि मे रामचन्द्रस्य भार्यां हृत्वा क्व यास्यसि}
{ततो रामाज्ञया धृत्वा शुकं बध्वाऽन्वरक्षयत्} %3-58

\twolineshloka
{शार्दूलोऽपि ततः पूर्वं दृष्ट्वा कपिबलं महत्}
{यथावत्कथयामास रावणाय स राक्षसः} %3-59

\twolineshloka
{दीर्घचिन्तापरो भूत्वा निःश्वसन्नास मन्दिरे}
{ततः समुद्रमावेक्ष्य रामो रक्तान्तलोचनः} %3-60

\twolineshloka
{पश्य लक्ष्मण दुष्टोऽसौ वारिधिर्मामुपागतम्}
{नाभिनन्दति दुष्टात्मा दर्शनार्थं ममानघ} %3-61

\twolineshloka
{जानाति मानुषोऽयं मे किं करिष्यति वानरैः}
{अद्य पश्य महाबाहो शोषयिष्यामि वारिधिम्} %3-62

\twolineshloka
{पादेनैव गमिष्यन्ति वानरा विगतज्वराः}
{इत्युक्त्वा क्रोधताम्राक्ष आरोपितधर्नुर्धरः} %3-63

\twolineshloka
{तूणीराद्बाणमादाय कालाग्निसदृशप्रभम्}
{सन्धाय चापमाकृष्य रामो वाक्यमथाब्रवीत्} %3-64

\twolineshloka
{पश्यन्तु सर्वभूतानि रामस्य शरविक्रमम्}
{इदानीं भस्मसात्कुर्यां समुद्रं सरितां पतिम्} %3-65

\twolineshloka
{एवं ब्रुवति रामे तु सशैलवनकानना}
{चचाल वसुधा द्यौश्च दिशश्च तमसावृताः} %3-66

\twolineshloka
{चुक्षुभे सागरो वेलां भयाद्योजनमत्यगात्}
{तिमिनक्रझषा मीनाः प्रतप्ताः परितत्रसुः} %3-67

\twolineshloka
{एतस्मिन्नन्तरे साक्षात्सागरो दिव्यरूपधृक्}
{दिव्याभरणसम्पन्नः स्वभासा भासयन् दिशः} %3-68

\twolineshloka
{स्वान्तःस्थदिव्यरत्नानि कराभ्यां परिगृह्य सः}
{पादयोः पुरतः क्षिप्त्वा रामस्योपायनं बहु} %3-69

\twolineshloka
{दण्डवत्प्रणित्याह रामं रक्तान्तलोचनम्}
{त्राहि त्राहि जगन्नाथ राम त्रैलोक्यरक्षक} %3-70

\twolineshloka
{जडोऽहं राम ते सृष्टः सृजता निखिलं जगत्}
{स्वभावमन्यथा कर्तुं कः शक्तो देवनिर्मितम्} %3-71

\twolineshloka
{स्थूलानि पञ्चभूतानि जडान्येव स्वभावतः}
{सृष्टानि भवतैतानि त्वदाज्ञां लङ्घयन्ति न} %3-72

\twolineshloka
{तामसादहमो राम भूतानि प्रभवन्ति हि}
{कारणानुगमात्तेषां जडत्वं तामसं स्वतः} %3-73

\twolineshloka
{निर्गुणस्त्वं निराकारो यदा मायागुणान् प्रभो}
{लीलयाऽङ्गीकरोषि त्वं तदा वैराजनामवान्} %3-74

\twolineshloka
{गुणात्मनो विराजश्च सत्त्वाद्देवा बभूविरे}
{रजोगुणात्प्रजेशाद्या मन्योर्भूतपतिस्तव} %3-75

{त्वामहं मायया छन्नं लीलया मानुषाकृतिम्} %3-76


\twolineshloka
{जडबुद्धिर्जडो मूर्खः कथं जानामि निर्गुणम्}
{दण्ड एव हि मूर्खाणां सन्मार्गप्रापकः प्रभो} %3-77

\threelineshloka
{भूतानाममरश्रेष्ठ पशूनां लगुडो यथा}
{शरणं ते व्रजामीशं शरण्यं भक्तवत्सल}
{अभयं देहि मे राम लङ्कामार्गं ददामि ते} %3-78

\uvacha{श्रीराम उवाच}

\twolineshloka
{अमोघोऽयं महाबाणः कस्मिन् देशे निपात्यताम्}
{लक्ष्यं दर्शय मे शीघ्रं बाणस्यामोघपातिनः} %3-79

\twolineshloka
{रामस्य वचनं श्रुत्वा करे दृष्ट्वा महाशरम्}
{महोदधिर्महातेजा राघवं वाक्यमब्रवीत्} %3-80

\twolineshloka
{रामोत्तरप्रदेशे तु द्रुमकुल्य इति श्रुतः}
{प्रदेशस्तत्र बहवः पापात्मानो दिवानिशम्} %3-81

\twolineshloka
{बाधन्ते मां रघुश्रेष्ठ तत्र ते पात्यतां शरः}
{रामेण सृष्टो बाणस्तु क्षणादाभीरमण्डलम्} %3-82

\twolineshloka
{हत्वा पुनः समागत्य तूणीरे पूर्ववत्स्थितः}
{ततोऽब्रवीद्रघुश्रेष्ठं सागरो विनयान्वितः} %3-83

\twolineshloka
{नलः सेतुं करोत्वस्मिन् जले मे विश्वकर्मणः}
{सुतो धीमान् समर्थोऽस्मिन् कार्ये लब्धवरो हरिः} %3-84

\twolineshloka
{कीर्तिं जानन्तु ते लोकाः सर्वलोकमलापहाम्}
{इत्युक्त्वा राघवं नत्वा ययौ सिन्धुरदृश्यताम्} %3-85

\twolineshloka
{ततो रामस्तु सुग्रीवलक्ष्मणाभ्यां समन्वितः}
{नलमाज्ञापयच्छीघ्रं वानरैः सेतुबन्धने} %3-86

\fourlineindentedshloka
{ततोऽतिहृष्टः प्लवगेन्द्रयूथपैर्-}
{महानगेन्द्रप्रतिमैर्युतो नलः}
{बबन्ध सेतुं शतयोजनायतम्}
{सुविस्तृतं पर्वतपादपैर्दृढम्} %3-87

\iti{युद्धकाण्डे}{तृतीयः}
%%%%%%%%%%%%%%%%%%%%



\sect{चतुर्थः सर्गः}

\uvacha{श्रीमहादेव उवाच}

\twolineshloka
{सेतुमारभमाणस्तु तत्र रामेश्वरं शिवम्}
{संस्थाप्य पूजयित्वाऽऽह रामो लोकहिताय च} %4-1

\twolineshloka
{प्रणमेत्सेतुबन्धं यो दृष्ट्वा रामेश्वरं शिवम्}
{ब्रह्महत्यादिपापेभ्यो मुच्यते मदनुग्रहात्} %4-2

\twolineshloka
{सेतुबन्धे नरः स्नात्वा दृष्ट्वा रामेश्वरं हरम्}
{सङ्कल्पनियतो भूत्वा गत्वा वाराणसीं नरः} %4-3

\twolineshloka
{आनीय गङ्गासलिलं रामेशमभिषिच्य च}
{समुद्रे क्षिप्ततद्भारो ब्रह्म प्राप्नोत्यसंशयम्} %4-4

\twolineshloka
{कृतानि प्रथमेनाह्ना योजनानि चतुर्दश}
{द्वितीयेन तथा चाह्ना योजनानि तु विंशतिः} %4-5

\twolineshloka
{तृतीयेन तथा चाह्ना योजनान्येकविंशतिः}
{चतुर्थेन तथा चाह्ना द्वाविंशतिरिति श्रुतम्} %4-6

\twolineshloka
{पञ्चमेन त्रयोविंशद्योजनानि समन्ततः}
{बबन्ध सागरे सेतुं नलो वानरसत्तमः} %4-7

\twolineshloka
{तेनैव जग्मुः कपयो योजनानां शतं द्रुतम्}
{असङ्ख्याताः सुवेलाद्रिं रुरुधुः प्लवगोत्तमाः} %4-8

\twolineshloka
{आरुह्य मारुतिं रामो लक्ष्मणोऽप्यङ्गदं तथा}
{दिदृक्षू राघवो लङ्कामारुरोहाचलं महत्} %4-9

\twolineshloka
{दृष्ट्वा लङ्कां सुविस्तीर्णां नानाचित्रध्वजाकुलाम्}
{चित्रप्रासादसम्बाधां स्वर्णप्राकारतोरणाम्} %4-10

\twolineshloka
{परिखाभिः शतघ्नीभिः सङ्क्रमैश्च विराजिताम्}
{प्रासादोपरि विस्तीर्णप्रदेशे दशकन्धरः} %4-11

\twolineshloka
{मन्त्रिभिः सहितो वीरैः किरीटदशकोज्ज्वलः}
{नीलाद्रिशिखराकारः कालमेघसमप्रभः} %4-12

\twolineshloka
{रत्नदण्डैः सितच्छत्रैरनेकैः परिशोभितः}
{एतस्मिन्नन्तरे बद्धो मुक्तो रामेण वै शुकः} %4-13

\twolineshloka
{वानरैस्ताडितः सम्यग् दशाननमुपागतः}
{प्रहसन् रावणः प्राह पीडितः किं परैः शुक} %4-14

\threelineshloka
{रावणस्य वचः श्रुत्वा शुको वचनमब्रवीत्}
{सागरस्योत्तरे तीरेऽब्रवं ते वचनं यथा}
{तत उत्प्लुत्य कपयो गृहीत्वा मां क्षणात्ततः} %4-15

\twolineshloka
{मुष्टिभिर्नखदन्तैश्च हन्तुं लोप्तुं प्रचक्रमुः}
{ततो मां राम रक्षेति क्रोशन्तं रघुपुङ्गवः} %4-16

\twolineshloka
{विसृज्यतामिति प्राह विसृष्टोऽहं कपीश्वरैः}
{ततोऽहमागतो भीत्या दृष्ट्वा तद्वानरं बलम्} %4-17

\twolineshloka
{राक्षसानां बलौघस्य वानरेन्द्रबलस्य च}
{नैतयोर्विद्यते सन्धिर्देवदानवयोरिव} %4-18

\twolineshloka
{पुरप्राकारमायान्ति क्षिप्रमेकतरं कुरु}
{सीतां वाऽस्मै प्रयच्छाऽऽशु युद्धं वा दीयतां प्रभो} %4-19

\twolineshloka
{मामाह रामस्त्वं ब्रूहि रावणं मद्वचः शुक}
{यद्बलं च समाश्रित्य सीतां मे हृतवानसि} %4-20

\twolineshloka
{तद्दर्शय यथाकामं ससैन्यः सहबान्धवः}
{श्वःकाले नगरीं लङ्कां सप्राकारां सतोरणाम्} %4-21

\twolineshloka
{राक्षसं च बलं पश्य शरैर्विध्वंसितं मया}
{घोररोषमहं मोक्ष्ये बलं धारय रावण} %4-22

\twolineshloka
{इत्युक्त्वोपररामाथ रामः कमललोचनः}
{एकस्थानगता यत्र चत्वारः पुरुषर्षभाः} %4-23

\twolineshloka
{श्रीरामो लक्ष्मणश्चैव सुग्रीवश्च विभीषणः}
{एत एव समर्थास्ते लङ्कां नाशयितुं प्रभो} %4-24

\twolineshloka
{उत्पाट्य भस्मीकरणे सर्वे तिष्ठन्तु वानराः}
{तस्य यादृग्बलं दृष्टं रूपं प्रहरणानि च} %4-25

\twolineshloka
{वधिष्यति पुरं सर्वमेकस्तिष्ठन्तु ते त्रयः}
{पश्य वानरसेनां तामसङ्ख्यातां प्रपूरिताम्} %4-26

\twolineshloka
{गर्जन्ति वानरास्तत्र पश्य पर्वतसन्निभाः}
{न शक्यास्ते गणयितुं प्राधान्येन ब्रवीमि ते} %4-27

\twolineshloka
{एष योऽभिमुखो लङ्कां नदंस्तिष्ठति वानरः}
{यूथपानां सहस्राणां शतेन परिवारितः} %4-28

\twolineshloka
{सुग्रीवसेनाधिपतिर्नीलो नामाग्निनन्दनः}
{एष पर्वतशृङ्गाभः पद्मकिञ्जल्कसन्निभः} %4-29

\twolineshloka
{स्फोटयत्यभिसंरब्धो लाङ्गूलं च पुनः पुनः}
{युवराजोऽङ्गदो नाम वालिपुत्रोऽतिवीर्यवान्} %4-30

\twolineshloka
{येन दृष्टा जनकजा रामस्यातीववल्लभा}
{हनूमानेष विख्यातो हतो येन तवाऽऽत्मजः} %4-31

\twolineshloka
{श्वेतो रजतसङ्काशो महाबुद्धिपराक्रमः}
{तूर्णं सुग्रीवमागत्य पुनर्गच्छति वानरः} %4-32

\twolineshloka
{यस्त्वेष सिंहसङ्काशः पश्यत्यतुलविक्रमः}
{रम्भो नाम महासत्त्वो लङ्कां नाशयितुं क्षमः} %4-33

\twolineshloka
{एष पश्यति वै लङ्कां दिधक्षन्निव वानरः}
{शरभो नाम राजेन्द्र कोटियूथपनायकः} %4-34

\twolineshloka
{पनसश्च महावीर्यो मैन्दश्च द्विविदस्तथा}
{नलश्च सेतुकर्ताऽसौ विश्वकर्मसुतो बली} %4-35

\twolineshloka
{वानराणां वर्णने वा सङ्ख्याने वा क ईश्वरः}
{शूराः सर्वे महाकायाः सर्वे युद्धाभिकाङ्क्षिणः} %4-36

\twolineshloka
{शक्ताः सर्वे चूर्णयितुं लङ्कां रक्षोगणैः सह}
{एतेषां बलसङ्ख्यानं प्रत्येकं वच्मि ते शृणु} %4-37

\twolineshloka
{एषां कोटिसहस्राणि नव पञ्च च सप्त च}
{तथा शङ्खसहस्राणि तथाऽर्बुदशतानि च} %4-38

\twolineshloka
{सुग्रीवसचिवानां ते बलमेतत्प्रकीर्तितम्}
{अन्येषां तु बलं नाहं वक्तुं शक्तोऽस्मि रावण} %4-39

\twolineshloka
{रामो न मानुषः साक्षादादिनारायणः परः}
{सीता साक्षाज्जगद्धेतुश्चिच्छक्तिर्जगदात्मिका} %4-40

\twolineshloka
{ताभ्यामेव समुत्पन्नं जगत्स्थावरजङ्गमम्}
{तस्माद्रामश्च सीता च जगतस्तस्थुषश्च तौ} %4-41

\twolineshloka
{पितरौ पृथिवीपाल तयोर्वैरी कथं भवेत्}
{अजानता त्वयाऽऽनीता जगन्मातैव जानकी} %4-42

\twolineshloka
{क्षणनाशिनि संसारे शरीरे क्षणभङ्गुरे}
{पञ्चभूतात्मके राजंश्चतुर्विंशतितत्त्वके} %4-43

\twolineshloka
{मलमांसास्थिदुर्गन्धभूयिष्ठेऽहङ्कृतालये}
{कैवास्था व्यतिरिक्तस्य काये तव जडात्मके} %4-44

\twolineshloka
{यत्कृते ब्रह्महत्यादिपातकानि कृतानि च}
{भोगभोक्ता तु यो देहः स देहोऽत्र पतिष्यति} %4-45

\twolineshloka
{पुण्यपापे समायातो जीवेन सुखदुःखयोः}
{कारणे देहयोगादिनाऽऽत्मनः कुरुतोऽनिशम्} %4-46

\twolineshloka
{यावद्देहोऽस्मि कर्ताऽस्मीत्यात्माऽहं कुरुतेऽवशः}
{अध्यासात्तावदेव स्याज्जन्मनाशादिसम्भवः} %4-47

\twolineshloka
{तस्मात्त्वं त्यज देहादावभिमानं महामते}
{आत्मातिऽनिर्मलः शुद्धो विज्ञानात्माऽचलोऽव्ययः} %4-48

\twolineshloka
{स्वाज्ञानवशतो बन्धं प्रतिपद्य विमुह्यति}
{तस्मात्त्वं शुद्धभावेन ज्ञात्वाऽऽत्मानं सदा स्मर} %4-49

\twolineshloka
{विरतिं भज सर्वत्र पुत्रदारगृहादिषु}
{निरयेष्वपि भोगः स्याच्छ्वशूकरतनावपि} %4-50

\twolineshloka
{देहं लब्ध्वा विवेकाढ्यं द्विजत्वं च विशेषतः}
{तत्रापि भारते वर्षे कर्मभूमौ सुदुर्लभम्} %4-51

\twolineshloka
{को विद्वानात्मसात्कृत्वा देहं भोगानुगो भवेत्}
{अतस्त्वं ब्राह्मणो भूत्वा पौलस्त्यतनयश्च सन्} %4-52

\twolineshloka
{अज्ञानीव सदा भोगाननुधावसि किं मुधा}
{इतः परं वा त्यक्त्वा त्वं सर्वसङ्गं समाश्रय} %4-53

\twolineshloka
{राममेव परात्मनं भक्तिभावेन सर्वदा}
{सीतां समर्प्य रामाय तत्पादानुचरो भव} %4-54

\threelineshloka
{विमुक्तः सर्वपापेभ्यो विष्णुलोकं प्रयास्यसि}
{नो चेद्गमिष्यसेऽधोऽधः पुनरावृत्तिवर्जितः}
{अङ्गीकुरुष्व मद्वाक्यं हितमेव वदामि ते} %4-55

\fourlineindentedshloka
{सत्सङ्गतिं कुरु भजस्व हरिं शरण्यम्}
{श्रीराघवं मरकतोपलकान्तिकान्तम्}
{सीतासमेतमनिशं धृतचापबाणम्}
{सुग्रीवलक्ष्मणविभीषणसेविताङ्घ्रिम्} %4-56

\iti{युद्धकाण्डे}{चतुर्थः}
%%%%%%%%%%%%%%%%%%%%



\sect{पञ्चमः सर्गः}

\uvacha{श्रीमहादेव उवाच}

\twolineshloka
{श्रुत्वा शुकमुखोद्गीतं वाक्यमज्ञाननाशनम्}
{रावणः क्रोधताम्राक्षो दहन्निव तमब्रवीत्} %5-1

\twolineshloka
{अनुजीव्य सुदुर्बुद्धे गुरुवद्भाषसे कथम्}
{शासिताऽहं त्रिजगतां त्वं मां शिक्षन्न लज्जसे} %5-2

\twolineshloka
{इदानीमेव हन्मि त्वां किन्तु पूर्वकृतं तव}
{स्मरामि तेन रक्षामि त्वां यद्यपि वधोचितम्} %5-3

\twolineshloka
{इतो गच्छ विमूढ त्वमेवं श्रोतुं न मे क्षमम्}
{महाप्रसाद इत्युक्त्वा वेपमानो गृहं ययौ} %5-4

\twolineshloka
{शुकोऽपि ब्राह्मणः पूर्वं ब्रह्मिष्ठो ब्रह्मवित्तमः}
{वानप्रस्थविधानेन वने तिष्ठन् स्वकर्मकृत्} %5-5

\twolineshloka
{देवानामभिवृद्ध्यर्थं विनाशाय सुरद्विषाम्}
{चकार यज्ञविततिमविच्छिन्नां महामतिः} %5-6

\twolineshloka
{राक्षसानां विरोधोऽभूच्छुको देवहितोद्यतः}
{वज्रदंष्ट्र इति ख्यातस्तत्रैको राक्षसो महान्} %5-7

\twolineshloka
{अन्तरं प्रेप्सुरातिष्ठच्छुकापकरणोद्यतः}
{कदाचिदागतोऽगस्त्यस्तस्याऽऽश्रमपदं मुनेः} %5-8

\twolineshloka
{तेन सम्पूजितोऽगस्त्यो भोजनार्थं निमन्त्रितः}
{गते स्नातुं मुनौ कुम्भसम्भवे प्राप्य चान्तरम्} %5-9

\twolineshloka
{अगस्त्यरूपधृक् सोऽपि राक्षसः शुकमब्रवीत्}
{यदि दास्यसि मे ब्रह्मन् भोजनं देहि सामिषम्} %5-10

\twolineshloka
{बहुकालं न भुक्तं मे मांसं छागाङ्गसम्भवम्}
{तथेति कारयामास मांसभोज्यं सविस्तरम्} %5-11

\twolineshloka
{उपविष्टे मुनौ भोक्तुं राक्षसोऽतीव सुन्दरम्}
{शुकभार्यावपुर्धृत्वा तां चान्तर्मोहयन् खलः} %5-12

\twolineshloka
{नरमांसं ददौ तस्मै सुपक्वं बहुविस्तरम्}
{दत्त्वैवान्तर्दधे रक्षस्ततो दृष्ट्वा चुकोप सः} %5-13

\twolineshloka
{अमेध्यं मानुषं मांसमगस्त्यः शुकमब्रवीत्}
{अभक्ष्यं मानुषं मांसं दत्तवानसि दुर्मते} %5-14

\twolineshloka
{मह्यं त्वं राक्षसो भूत्वा तिष्ठ त्वं मानुषाशनः}
{इति शप्तः शुको भीत्या प्राहागस्त्यं मुने त्वया} %5-15

\twolineshloka
{इदानीं भाषितं मेऽद्य मांसं देहीति विस्तरम्}
{तथैव दत्तं भो देव किं मे शापं प्रदास्यसि} %5-16

\twolineshloka
{श्रुत्वा शुकस्य वचनं मुहूर्तं ध्यानमास्थितः}
{ज्ञात्वा रक्षःकृतं सर्वं ततः प्राह शुकं सुधीः} %5-17

\twolineshloka
{तवापकारिणा सर्वं राक्षसेन कृतं त्विदम्}
{अविचार्यैव मे दत्तः शापस्ते मुनिसत्तम} %5-18

\twolineshloka
{तथाऽपि मे वचोऽमोघमेवमेव भविष्यति}
{राक्षसं वपुरास्थाय रावणस्य सहायकृत्} %5-19

\twolineshloka
{तिष्ठ तावद्यदा रामो दशाननवधाय हि}
{आगमिष्यति लङ्कायाः समीपं वानरैः सह} %5-20

\twolineshloka
{प्रेषितो रावणेन त्वं चारो भूत्वा रघूत्तमम्}
{दृष्ट्वा शापाद्विनिर्मुक्तो बोधयित्वा च रावणम्} %5-21

\twolineshloka
{तत्त्वज्ञानं ततो मुक्तः परं पदमवाप्स्यसि}
{इत्युक्तोऽगस्त्यमुनिना शुको ब्राह्मणसत्तमः} %5-22

\twolineshloka
{बभूव राक्षसः सद्यो रावणं प्राप्य संस्थितः}
{इदानीं चाररूपेण दृष्ट्वा रामं सहानुजम्} %5-23

\twolineshloka
{रावणं तत्त्वविज्ञानं बोधयित्वा पुनर्द्रुतम्}
{पूर्ववद्ब्राह्मणो भूत्वा स्थितो वैखानसैः सह} %5-24

\twolineshloka
{ततः समागमद्वृद्धो माल्यवान् राक्षसो महान्}
{बुद्धिमान्नीतिनिपुणो राज्ञो मातुः प्रियः पिता} %5-25

\twolineshloka
{प्राह तं राक्षसं वीरं प्रशान्तेनान्तरात्मना}
{शृणु राजन् वचो मेऽद्य श्रुत्वा कुरु यथेप्सितम्} %5-26

\twolineshloka
{यदा प्रविष्टा नगरीं जानकी रामवल्लभा}
{तदादि पुर्यां दृश्यन्ते निमित्तानि दशानन} %5-27

\twolineshloka
{घोराणि नाशहेतूनि तानि मे वदतः शृणु}
{खरस्तनितनिर्घोषा मेघा अतिभयङ्कराः} %5-28

\twolineshloka
{शोणितेनाभिवर्षन्ति लङ्कामुष्णेन सर्वदा}
{रुदन्ति देवलिङ्गानि स्विद्यन्ति प्रचलन्ति च} %5-29

\twolineshloka
{कालिका पाण्डुरैर्दन्तैः प्रहसत्यग्रतः स्थिता}
{खरा गोषु प्रजायन्ते मूषका नकुलैः सह} %5-30

\twolineshloka
{मार्जारेण तु युद्ध्यन्ति पन्नगा गरुडेन तु}
{करालो विकटो मुण्डः पुरुषः कृष्णपिङ्गलः} %5-31

\twolineshloka
{कालो गृहाणि सर्वेषां काले काले त्ववेक्षते}
{एतान्यन्यानि दृश्यन्ते निमित्तान्युद्भवन्ति च} %5-32

\twolineshloka
{अतः कुलस्य रक्षार्थं शान्तिं कुरु दशानन}
{सीतां सत्कृत्य सधनां रामायाऽऽशु प्रयच्छ भोः} %5-33

\twolineshloka
{रामं नारायणं विद्धि विद्वेषं त्यज राघवे}
{यत्पादपोतमाश्रित्य ज्ञानिनो भवसागरम्} %5-34

\twolineshloka
{तरन्ति भक्तिपूतान्तास्ततो रामो न मानुषः}
{भजस्व भक्तिभावेन रामं सर्वहृदालयम्} %5-35

\twolineshloka
{यद्यपि त्वं दुराचारो भक्त्या पूतो भविष्यसि}
{मद्वाक्यं कुरु राजेन्द्र कुलकौशलहेतवे} %5-36

\twolineshloka
{तत्तु माल्यवतो वाक्यं हितमुक्तं दशाननः}
{न मर्षयति दुष्टात्मा कालस्य वशमागतः} %5-37

\twolineshloka
{मानवं कृपणं राममेकं शाखामृगाश्रयम्}
{समर्थं मन्यसे केन हीनं पित्रा मुनिप्रियम्} %5-38

\twolineshloka
{रामेण प्रेषितो नूनं भाषसे त्वमनर्गलम्}
{गच्छ वृद्धोऽसि बन्धुस्त्वं सोढं सर्वं त्वयोदितम्} %5-39

\twolineshloka
{इतो मत्कर्णपदवीं दहत्येतद्वचस्तव}
{इत्युक्त्वा सर्वसचिवैः सहितः प्रस्थितस्तदा} %5-40

\twolineshloka
{प्रासादाग्रे समासीनः पश्यन् वानरसैनिकान्}
{युद्धायाऽऽयोजयत्सर्वराक्षसान् समुपस्थितान्} %5-41

\twolineshloka
{रामोऽपि धनुरादाय लक्ष्मणेन समाहृतम्}
{दृष्ट्वा रावणमासीनं कोपेन कलुषीकृतः} %5-42

\twolineshloka
{किरीटिनं समासीनं मन्त्रिभिः परिवेष्टितम्}
{शशाङ्कार्धनिभेनैव बाणेनैकेन राघवः} %5-43

\twolineshloka
{श्वेतच्छत्रसहस्राणि किरीटदशकं तथा}
{चिच्छेद निमिषार्धेन तदद्भुतमिवाभवत्} %5-44

\twolineshloka
{लज्जितो रावणस्तूर्णं विवेश भवनं स्वकम्}
{आहूय राक्षसान् सर्वान् प्रहस्तप्रमुखान् खलः} %5-45

\twolineshloka
{वानरैः सह युद्धाय नोदयामास सत्वरः}
{ततो भेरीमृदङ्गाद्यैः पणवानकगोमुखैः} %5-46

\twolineshloka
{महिषोष्ट्रैः खरैः सिंहैर्द्वीपिभिः कृतवाहनाः}
{खड्गशूलधनुःपाशयष्टितोमरशक्तिभिः} %5-47

\twolineshloka
{लक्षिताः सर्वतो लङ्कां प्रतिद्वारमुपाययुः}
{तत्पूर्वमेव रामेण नोदिता वानरर्षभाः} %5-48

\twolineshloka
{उद्यम्य गिरिशृङ्गाणि शिखराणि महान्ति च}
{तरूंश्चोत्पाट्य विविधान् युद्धाय हरियूथपाः} %5-49

\twolineshloka
{प्रेक्षमाणा रावणस्य तान्यनीकानि भागशः}
{राघवप्रियकामार्थं लङ्कामारुरुहुस्तदा} %5-50

\twolineshloka
{ते द्रुमैः पर्वताग्रैश्च मुष्टिभिश्च प्लवङ्गमाः}
{ततः सहस्रयूथाश्च कोटियूथाश्च यूथपाः} %5-51

\twolineshloka
{कोटीशतयुताश्चान्ये रुरुधुर्नगरं भृशम्}
{आप्लवन्तः प्लवन्तश्च गर्जन्तश्च प्लवङ्गमाः} %5-52

\twolineshloka
{रामो जयत्यतिबलो लक्ष्मणश्च महाबलः}
{राजा जयति सुग्रीवो राघवेणानुपालितः} %5-53

\twolineshloka
{इत्येवं घोषयन्तश्च समं युयुधिरेऽरिभिः}
{हनूमानङ्गदश्चैव कुमुदो नील एव च} %5-54

\twolineshloka
{नलश्च शरभश्चैव मैन्दो द्विविद एव च}
{जाम्बवान् दधिवक्त्रश्च केसरी तार एव च} %5-55

\threelineshloka
{अन्ये च बलिनः सर्वे यूथपाश्च प्लवङ्गमाः}
{द्वाराण्युत्प्लुत्य लङ्कायाः सर्वतो रुरुधुर्भृशम्}
{तदा वृक्षैर्महाकायाः पर्वताग्रैश्च वानराः} %5-56

\twolineshloka
{निजघ्नुस्तानि रक्षांसि नखैर्दन्तैश्च वेगिताः}
{राक्षसाश्च तदा भीमा द्वारेभ्यः सर्वतो रुषा} %5-57

\twolineshloka
{निर्गत्य भिन्दिपालैश्च खड्गैः शूलैः परश्वधैः}
{निजघ्नुर्वानरानीकं महाकाया महाबलाः} %5-58

\twolineshloka
{राक्षसांश्च तथा जघ्नुर्वानरा जितकाशिनः}
{तदा बभूव समरो मांसशोणितकर्दमः} %5-59

\twolineshloka
{रक्षसां वानराणां च सम्बभूवाद्भुतोपमः}
{ते हयैश्च गजैश्चैव रथैः काञ्चनसन्निभैः} %5-60

\twolineshloka
{रक्षोव्याघ्रा युयुधिरे नादयन्तो दिशो दश}
{राक्षसाश्च कपीन्द्राश्च परस्परजयैषिणः} %5-61

\twolineshloka
{राक्षसान् वानरा जघ्नुर्वानरांश्चैव राक्षसाः}
{रामेण विष्णुना दृष्टा हरयो दिविजांशजाः} %5-62

\twolineshloka
{बभूवुर्बलिनो हृष्टास्तदा पीतामृता इव}
{सीताभिमर्शपापेन रावणेनाभिपालितान्} %5-63

\twolineshloka
{हतश्रीकान् हतबलान् राक्षसान् जघ्नुरोजसा}
{चतुर्थांशावशेषेण निहतं राक्षसं बलम्} %5-64

\twolineshloka
{स्वसैन्यं निहतं दृष्ट्वा मेघनादोऽथ दुष्टधीः}
{ब्रह्मदत्तवरः श्रीमानन्तर्धानं गतोऽसुरः} %5-65

\twolineshloka
{सर्वास्त्रकुशलो व्योम्नि ब्रह्मास्त्रेण समन्ततः}
{नानाविधानि शस्त्राणि वानरानीकमर्दयन्} %5-66

\twolineshloka
{ववर्ष शरजालानि तदद्भुतमिवाभवत्}
{रामोऽपि मानयन् ब्राह्ममस्त्रमस्त्रविदां वरः} %5-67

\twolineshloka
{क्षणं तूष्णीमुवासाथ ददर्श पतितं बलम्}
{वानराणां रघुश्रेष्ठश्चुकोपानलसन्निभः} %5-68

\twolineshloka
{चापमानय सौमित्रे ब्रह्मास्त्रेणासुरं क्षणात्}
{भस्मीकरोमि मे पश्य बलमद्य रघूत्तम} %5-69

\twolineshloka
{मेघनादोऽपि तच्छ्रुत्वा रामवाक्यमतन्द्रितः}
{तूर्णं जगाम नगरं मायया मायिकोऽसुरः} %5-70

\twolineshloka
{पतितं वानरानीकं दृष्ट्वा रामोऽतिदुःखितः}
{उवाच मारुतिं शीघ्रं गत्वा क्षीरमहोदधिम्} %5-71

\twolineshloka
{तत्र द्रोणगिरिर्नाम दिव्यौषधिसमुद्भवः}
{तमानय द्रुतं गत्वा सञ्जीवय महामते} %5-72

\twolineshloka
{वानरौघान् महासत्त्वान् कीर्तिस्ते सुस्थिरा भवेत्}
{आज्ञाप्रमाणमित्युक्त्वा जगामानिलनन्दनः} %5-73

\twolineshloka
{आनीय च गिरिं सर्वान् वानरान् वानरर्षभः}
{जीवयित्वा पुनस्तत्र स्थापयित्वाऽऽययौ द्रुतम्} %5-74

\twolineshloka
{पूर्ववद्भैरवं नादं वानराणां बलौघतः}
{श्रुत्वा विस्मयमापन्नो रावणो वाक्यमब्रवीत्} %5-75

\twolineshloka
{राघवो मे महान् शत्रुः प्राप्तो देवविनिर्मितः}
{हन्तुं तं समरे शीघ्रं गच्छन्तु मम यूथपाः} %5-76

\twolineshloka
{मन्त्रिणो बान्धवाः शूरा ये च मत्प्रियकाङ्क्षिणः}
{सर्वे गच्छन्तु युद्धाय त्वरितं मम शासनात्} %5-77

\twolineshloka
{ये न गच्छन्ति युद्धाय भीरवः प्राणविप्लवात्}
{तान् हनिष्याम्यहं सर्वान् मच्छासनपराङ्मुखान्} %5-78

\twolineshloka
{तच्छ्रुत्वा भयसन्त्रस्ता निर्जग्मू रणकोविदाः}
{अतिकायः प्रहस्तश्च महानादमहोदरौ} %5-79

\twolineshloka
{देवशत्रुर्निकुम्भश्च देवान्तकनरान्तकौ}
{अपरे बलिनः सर्वे ययुर्युद्धाय वानरैः} %5-80

\twolineshloka
{एते चान्ये च बहवः शूराः शतसहस्रशः}
{प्रविश्य वानरं सैन्यं ममन्थुर्बलदर्पिताः} %5-81

\twolineshloka
{भुशुण्डीभिन्दिपालैश्च बाणैः खड्गैः परश्वधैः}
{अन्यैश्च विविधैरस्त्रैर्निजघ्नुर्हरियूथपान्} %5-82

\twolineshloka
{ते पादपैः पर्वताग्रैर्नखदंष्ट्रैश्च मुष्टिभिः}
{प्राणैर्विमोचयामासुः सर्वराक्षसयूथपान्} %5-83

\threelineshloka
{रामेण निहताः केचित्सुग्रीवेण तथाऽपरे}
{हनूमता चाङ्गदेन लक्ष्मणेन महात्मना}
{यूथपैर्वानराणां ते निहताः सर्वराक्षसाः} %5-84

\twolineshloka
{रामतेजः समाविश्य वानरा बलिनोऽभवन्}
{रामशक्तिविहीनानामेवं शक्तिः कुतो भवेत्} %5-85

\fourlineindentedshloka
{सर्वेश्वरः सर्वमयो विधाता}
{मायामनुष्यत्वविडम्बनेन}
{सदा चिदानन्दमयोऽपि रामो}
{युद्धादिलीलां वितनोति मायाम्} %5-86

\iti{युद्धकाण्डे}{पञ्चमः}
%%%%%%%%%%%%%%%%%%%%



\sect{षष्ठः सर्गः}

\uvacha{श्रीमहादेव उवाच}

\twolineshloka
{श्रुत्वा युद्धे बलं नष्टमतिकायमुखं महत्}
{रावणो दुःखसन्तप्तः क्रोधेन महताऽऽवृतः} %6-1

\twolineshloka
{निधायेन्द्रजितं लङ्कारक्षणार्थं महाद्युतिः}
{स्वयं जगाम युद्धाय रामेण सह राक्षसः} %6-2

\twolineshloka
{दिव्यं स्यन्दनमारुह्य सर्वशस्त्रास्त्रसंयुतम्}
{राममेवाभिदुद्राव राक्षसेन्द्रो महाबलः} %6-3

\twolineshloka
{वानरान् बहुशो हत्वा बाणैराशीविषोपमैः}
{पातयामास सुग्रीवप्रमुखान् यूथनायकान्} %6-4

\twolineshloka
{गदापाणिं महासत्त्वं तत्र दृष्ट्वा विभीषणम्}
{उत्ससर्ज महाशक्तिं मयदत्तां विभीषणे} %6-5

\twolineshloka
{तामापतन्तीमालोक्य विभीषणविघातिनीम्}
{दत्ताभयोऽयं रामेण वधार्हो नायमासुरः} %6-6

\twolineshloka
{इत्युक्त्वा लक्ष्मणो भीमं चापमादाय वीर्यवान्}
{विभीषणस्य पुरतः स्थितोऽकम्प इवाचलः} %6-7

\twolineshloka
{सा शक्तिर्लक्ष्मणतनुं विवेशामोघशक्तितः}
{यावन्त्यः शक्तयो लोके मायायाः सम्भवन्ति हि} %6-8

\twolineshloka
{तासामाधारभूतस्य लक्ष्मणस्य महात्मनः}
{मायाशक्त्या भवेत्किं वा शेषांशस्य हरेस्तनोः} %6-9

\twolineshloka
{तथाऽपि मानुषं भावमापन्नस्तदनुव्रतः}
{मूर्च्छितः पतितो भूमौ तमादातुं दशाननः} %6-10

\twolineshloka
{हस्तैस्तोलयितुं शक्तो न बभूवातिविस्मितः}
{सर्वस्य जगतः सारं विराजं परमेश्वरम्} %6-11

\twolineshloka
{कथं लोकाश्रयं विष्णुं तोलयेल्लघुराक्षसः}
{ग्रहीतुकामं सौमित्रिं रावणं वीक्ष्य मारुतिः} %6-12

\twolineshloka
{आजघानोरसि क्रुद्धो वज्रकल्पेन मुष्टिना}
{तेन मुष्टिप्रहारेण जानुभ्यामपतद्भुवि} %6-13

\twolineshloka
{आस्यैश्च नेत्रश्रवणैरुद्वमन् रुधिरं बहु}
{विघूर्णमाननयनो रथोपस्थ उपाविशत्} %6-14

\twolineshloka
{अथ लक्ष्मणमादाय हनूमान् रावणार्दितम्}
{आनयद्रामसामीप्यं बाहुभ्यां परिगृह्य तम्} %6-15

\twolineshloka
{हनूमतः सुहृत्त्वेन भक्त्या च परमेश्वरः}
{लघुत्वमगमद्देवो गुरूणां गुरुरप्यजः} %6-16

\twolineshloka
{सा शक्तिरपि तं त्यक्त्वा ज्ञात्वा नारायणांशजम्}
{रावणस्य रथं प्रागाद्रावणोऽपि शनैस्ततः} %6-17

\twolineshloka
{संज्ञामवाप्य जग्राह बाणासनमथो रुषा}
{राममेवाभिदुद्राव दृष्ट्वा रामोऽपि तं क्रुधा} %6-18

\twolineshloka
{आरुह्य जगतां नाथो हनूमन्तं महाबलम्}
{रथस्थं रावणं दृष्ट्वा अभिदुद्राव राघवः} %6-19

\twolineshloka
{ज्याशब्दमकरोत्तीव्रं वज्रनिष्पेषनिष्ठुरम्}
{रामो गम्भीरया वाचा राक्षसेन्द्रमुवाच ह} %6-20

\twolineshloka
{राक्षसाधम तिष्ठाद्य क्व गमिष्यसि मे पुरः}
{कृत्वाऽपराधमेवं मे सर्वत्र समदर्शिनः} %6-21

\twolineshloka
{येन बाणेन निहता राक्षसास्ते जनालये}
{तेनैव त्वां हनिष्यामि तिष्ठाद्य मम गोचरे} %6-22

\twolineshloka
{श्रीरामस्य वचः श्रुत्वा रावणो मारुतात्मजम्}
{वहन्तं राघवं सङ्ख्ये शरैस्तीक्ष्णैरताडयत्} %6-23

\twolineshloka
{हतस्यापि शरैस्तीक्ष्णैर्वायुसूनोः स्वतेजसा}
{व्यवर्धत पुनस्तेजो ननर्द च महाकपिः} %6-24

\twolineshloka
{ततो दृष्ट्वा हनूमन्तं सव्रणं रघुसत्तमः}
{क्रोधमाहारयामास कालरुद्र इवापरः} %6-25

\twolineshloka
{साश्वं रथं ध्वजं सूतं शस्त्रौघं धनुरञ्जसा}
{छत्रं पताकां तरसा चिच्छेद शितसायकैः} %6-26

\twolineshloka
{ततो महाशरेणाशु रावणं रघुसत्तमः}
{विव्याध वज्रकल्पेन पाकारिरिव पर्वतम्} %6-27

\twolineshloka
{रामबाणहतो वीरश्चचाल च मुमोह च}
{हस्तान्निपतितश्चापस्तं समीक्ष्य रघूत्तमः} %6-28

\twolineshloka
{अर्धचन्द्रेण चिच्छेद तत्किरीटं रविप्रभम्}
{अनुजानामि गच्छ त्वमिदानीं बाणपीडितः} %6-29

\twolineshloka
{प्रविश्य लङ्कामाश्वास्य श्वः पश्यसि बलं मम}
{रामबाणेन संविद्धो हतदर्पोऽथ रावणः} %6-30

\twolineshloka
{महत्या लज्जया युक्तो लङ्कां प्राविशदातुरः}
{रामोऽपि लक्ष्मणं दृष्ट्वा मूर्च्छितं पतितं भुवि} %6-31

\twolineshloka
{मानुषत्वमुपाश्रित्य लीलयाऽनुशुशोच ह}
{ततः प्राह हनूमन्तं वत्स जीवय लक्ष्मणम्} %6-32

\twolineshloka
{महौषधीः समानीय पूर्ववद्वानरानपि}
{तथेति रघवेणोक्तो जगामाऽऽशु महाकपिः} %6-33

\twolineshloka
{हनूमान् वायुवेगेन क्षणात्तीर्त्वा महोदधिम्}
{एतस्मिन्नन्तरे चारा रावणाय न्यवेदयन्} %6-34

\twolineshloka
{रामेण प्रेषितो देव हनूमान् क्षीरसागरम्}
{गतो नेतुं लक्ष्मणस्य जीवनार्थं महौषधीः} %6-35

\twolineshloka
{श्रुत्वा तच्चारवचनं राजा चिन्तापरोऽभवत्}
{जगाम रात्रावेकाकी कालनेमिगृहं क्षणात्} %6-36

\threelineshloka
{गृहागतं समालोक्य रावणं विस्मयान्वितः}
{कालनेमिरुवाचेदं प्राञ्जलिर्भयविह्वलः}
{अर्घ्यादिकं ततः कृत्वा रावणस्याग्रतः स्थितः} %6-37

\twolineshloka
{किं ते करोमि राजेन्द्र किमागमनकारणम्}
{कालनेमिमुवाचेदं रावणो दुःखपीडितः} %6-38

\twolineshloka
{ममापि कालवशतः कष्टमेतदुपस्थितम्}
{मया शक्त्या हतो वीरो लक्ष्मणः पतितो भुवि} %6-39

\twolineshloka
{तं जीवयितुमानेतुमोषधीर्हनुमान् गतः}
{यथा तस्य भवेद्विघ्नस्तथा कुरु महामते} %6-40

\twolineshloka
{मायया मुनिवेषेण मोहयस्व महाकपिम्}
{कालात्ययो यथा भूयात्तथा कृत्वैहि मन्दिरे} %6-41

\twolineshloka
{रावणस्य वचः श्रुत्वा कालनेमिरुवाच तम्}
{रावणेश वचो मेऽद्य शृणु धारय तत्त्वतः} %6-42

\twolineshloka
{प्रियं ते करवाण्येव न प्राणान् धारयाम्यहम्}
{मारीचस्य यथाऽरण्ये पुराऽभून्मृगरूपिणः} %6-43

\twolineshloka
{तथैव मे न सन्देहो भविष्यति दशानन}
{हताः पुत्राश्च पौत्राश्च बान्धवा राक्षसाश्च ते} %6-44

\twolineshloka
{घातयित्वाऽसुरकुलं जीवितेनापि किं तव}
{राज्येन वा सीतया वा किं देहेन जडात्मना} %6-45

\twolineshloka
{सीतां प्रयच्छ रामाय राज्यं देहि विभीषणे}
{वनं याहि महाबाहो रम्यं मुनिगणाश्रयम्} %6-46

\twolineshloka
{स्नात्वा प्रातः शुभजले कृत्वा सन्ध्यादिकाः क्रियाः}
{तत एकान्तमाश्रित्य सुखासनपरिग्रहः} %6-47

\twolineshloka
{विसृज्य सर्वतः सङ्गमितरान् विषयान् बहिः}
{बहिःप्रवृत्ताक्षगणं शनैः प्रत्यक् प्रवाहय} %6-48

\twolineshloka
{प्रकृतेर्भिन्नमात्मानं विचारय सदाऽनघ}
{चराचरं जगत्कृत्स्नं देहबुद्धीन्द्रियादिकम्} %6-49

\twolineshloka
{आब्रह्मस्तम्बपर्यन्तं दृश्यते श्रूयते च यत्}
{सैषा प्रकृतिरित्युक्ता सैव मायेति कीर्तिता} %6-50

\twolineshloka
{सर्गस्थितिविनाशानां जगद्वृक्षस्य कारणम्}
{लोहितश्वेतकृष्णादि प्रजाः सृजति सर्वदा} %6-51

\twolineshloka
{कामक्रोधादिपुत्राद्यान् हिंसातृष्णादिकन्यकाः}
{मोहयन्त्यनिशं देवमात्मानं स्वैर्गुणैर्विभुम्} %6-52

\twolineshloka
{कर्तृत्वभोक्तृत्वमुखान् स्वगुणानात्मनीश्वरे}
{आरोप्य स्ववशं कृत्वा तेन क्रीडति सर्वदा} %6-53

\twolineshloka
{शुद्धोऽप्यात्मा यया युक्तः पश्यतीव सदा बहिः}
{विस्मृत्य च स्वमात्मानं मायागुणविमोहितः} %6-54

\twolineshloka
{यदा सद्गुरुणा युक्तो बोध्यते बोधरूपिणा}
{निवृत्तदृष्टिरात्मानं पश्यत्येव सदा स्फुटम्} %6-55

\twolineshloka
{जीवन्मुक्तः सदा देही मुच्यते प्राकृतैर्गुणैः}
{त्वमप्येवं सदाऽऽत्मानं विचार्य नियतेन्द्रियः} %6-56

\twolineshloka
{प्रकृतेरन्यमात्मानं ज्ञात्वा मुक्तो भविष्यसि}
{ध्यातुं यद्यसमर्थोऽसि सगुणं देवमाश्रय} %6-57

\twolineshloka
{हृत्पद्मकर्णिके स्वर्णपीठे मणिगणान्विते}
{मृदुश्लक्ष्णतरे तत्र जानक्या सह संस्थितम्} %6-58

\twolineshloka
{वीरासनं विशालाक्षं विद्युत्पुञ्जनिभाम्बरम्}
{किरीटहारकेयूरकौस्तुभादिभिरन्वितम्} %6-59

\twolineshloka
{नूपुरैः कटकैर्भान्तं तथैव वनमालया}
{लक्ष्मणेन धनुर्द्वन्द्वकरेण परिसेवितम्} %6-60

\twolineshloka
{एवं ध्यात्वा सदाऽऽत्मानं रामं सर्वहृदि स्थितम्}
{भक्त्या परमया युक्तो मुच्यते नात्र संशयः} %6-61

\threelineshloka
{शृणु वै चरितं तस्य भक्तैर्नित्यमनन्यधीः}
{एवं चेत्कृतपूर्वाणि पापानि च महान्त्यपि}
{क्षणादेव विनश्यन्ति यथाऽग्नेस्तूलराशयः} %6-62

\fourlineindentedshloka
{भजस्व रामं परिपूर्णमेकम्}
{विहाय वैरं निजभक्तियुक्तः}
{हृदा सदा भावितभावरूपम्}
{अनामरूपं पुरुषं पुराणम्} %6-63

\iti{युद्धकाण्डे}{षष्ठः}
%%%%%%%%%%%%%%%%%%%%



\sect{सप्तमः सर्गः}

\uvacha{श्रीमहादेव उवाच}

\twolineshloka
{कालनेमिवचः श्रुत्वा रावणोऽमृतसन्निभम्}
{जज्वाल क्रोधताम्राक्षः सर्पिरद्भिरिवाग्निमत्} %7-1

\twolineshloka
{निहन्मि त्वां दुरात्मानं मच्छासनपराङ्मुखम्}
{परैः किञ्चिद्गृहीत्वा त्वं भाषसे रामकिङ्करः} %7-2

\twolineshloka
{कालनेमिरुवाचेदं रावणं देव किं क्रुधा}
{न रोचते मे वचनं यदि गत्वा करोमि तत्} %7-3

\twolineshloka
{इत्युक्त्वा प्रययौ शीघ्रं कालनेमिर्महासुरः}
{नोदितो रावणेनैव हनूमद्विघ्नकारणात्} %7-4

\twolineshloka
{स गत्वा हिमवत्पार्श्वं तपोवनमकल्पयत्}
{तत्र शिष्यैः परिवृतो मुनिवेषधरः खलः} %7-5

\twolineshloka
{गच्छतो मार्गमासाद्य वायुसूनोर्महात्मनः}
{ततो गत्वा ददर्शाथ हनूमानाश्रमं शुभम्} %7-6

\twolineshloka
{चिन्तयामास मनसा श्रीमान् पवननन्दनः}
{पुरा न दृष्टमेतन्मे मुनिमण्डलमुत्तमम्} %7-7

\twolineshloka
{मार्गो विभ्रंशितो वा मे भ्रमो वा चित्तसम्भवः}
{यद्वाऽऽविश्याऽऽश्रमपदं दृष्ट्वा मुनिमशेषतः} %7-8

\twolineshloka
{पीत्वा जलं ततो यामि द्रोणाचलमनुत्तमम्}
{इत्युक्त्वा प्रविवेशाथ सर्वतो योजनायतम्} %7-9

\twolineshloka
{आश्रमं कदलीशालखर्जूरपनसादिभिः}
{समावृतं पक्वफलैर्नम्रशाखैश्च पादपैः} %7-10

\twolineshloka
{वैरभावविनिर्मुक्तं शुद्धं निर्मललक्षणम्}
{तस्मिन्महाश्रमे रम्ये कालनेमिः स राक्षसः} %7-11

\twolineshloka
{इन्द्रयोगं समास्थाय चकार शिवपूजनम्}
{हनूमानभिवाद्याऽऽह गौरवेण महासुरम्} %7-12

\twolineshloka
{भगवन् रामदूतोऽहं हनूमान्नाम नामतः}
{रामकार्येण महता क्षीराब्धिं गन्तुमुद्यतः} %7-13

\twolineshloka
{तृषा मां बाधते ब्रह्मन्नुदकं कुत्र विद्यते}
{यथेच्छं पातुमिच्छामि कथ्यतां मे मुनीश्वर} %7-14

\twolineshloka
{तच्छ्रुत्वा मारुतेर्वाक्यं कालनेमिस्तमब्रवीत्}
{कमण्डलुगतं तोयं मम त्वं पातुमर्हसि} %7-15

\twolineshloka
{भुङ्क्ष्व चेमानि पक्वानि फलानि तदनन्तरम्}
{निवसस्व सुखेनात्र निद्रामेहि त्वरास्तु मा} %7-16

\twolineshloka
{भूतं भव्यं भविष्यं च जानामि तपसा स्वयम्}
{उत्थितो लक्ष्मणः सर्वे वानरा रामवीक्षिताः} %7-17

\twolineshloka
{तच्छ्रुत्वा हनुमानाह कमण्डलुजलेन मे}
{न शाम्यत्यधिका तृष्णा ततो दर्शय मे जलम्} %7-18

\twolineshloka
{तथेत्याज्ञापयामास वटुं मायाविकल्पितम्}
{वटो दर्शय विस्तीर्णं वायुसूनोर्जलाशयम्} %7-19

\twolineshloka
{निमील्य चाक्षिणी तोयं पीत्वाऽऽगच्छ ममान्तिकम्}
{उपदेक्ष्यामि ते मन्त्रं येन द्रक्ष्यसि चौषधीः} %7-20

\twolineshloka
{तथेति दर्शितं शीघ्रं वटुना सलिलाशयम्}
{प्रविश्य हनुमांस्तोयमपिबन्मीलितेक्षणः} %7-21

\twolineshloka
{ततश्चाऽऽगत्य मकरी महामाया महाकपिम्}
{अग्रसत्तं महावेगान्मारुतिं घोररूपिणी} %7-22

\twolineshloka
{ततो ददर्श हनुमान् ग्रसन्तीं मकरीं रुषा}
{दारयामास हस्ताभ्यां वदनं सा ममार ह} %7-23

\twolineshloka
{ततोऽन्तरिक्षे ददृशे दिव्यरूपधराङ्गना}
{धान्यमालीति विख्याता हनूमन्तमथाब्रवीत्} %7-24

\twolineshloka
{त्वत्प्रसादादहं शापाद्विमुक्ताऽस्मि कपीश्वर}
{शप्ताऽहं मुनिना पूर्वमप्सरा कारणान्तरे} %7-25

\twolineshloka
{आश्रमे यस्तु ते दृष्टः कालनेमिर्महासुरः}
{रावणप्रहितो मार्गे विघ्नं कर्तुं तवानघ} %7-26

\twolineshloka
{मुनिवेषधरो नासौ मुनिर्विप्रविहिंसकः}
{जहि दुष्टं गच्छ शीघ्रं द्रोणाचलमनुत्तमम्} %7-27

\twolineshloka
{गच्छाम्यहं ब्रह्मलोकं त्वत्स्पर्शाद्धतकल्मषा}
{इत्युक्त्वा सा ययौ स्वर्गं हनूमानप्यथाऽऽश्रमम्} %7-28

\twolineshloka
{आगतं तं समालोक्य कालनेमिरभाषत}
{किं विलम्बेन महता तव वानरसत्तम} %7-29

\twolineshloka
{गृहाण मत्तो मन्त्रांस्त्वं देहि मे गुरुदक्षिणाम्}
{इत्युक्तो हनुमान्मुष्टिं दृढं बद्ध्वाऽऽह राक्षसम्} %7-30

\twolineshloka
{गृहाण दक्षिणामेतामित्युक्त्वा निजघान तम्}
{विसृज्य मुनिवेषं स कालनेमिर्महासुरः} %7-31

\twolineshloka
{युयुधे वायुपुत्रेण नानामायाविधानतः}
{महामायिकदूतोऽसौ हनूमान्मायिनां रिपुः} %7-32

\twolineshloka
{जघान मुष्टिना शीर्ष्णि भग्नमूर्धा ममार सः}
{ततः क्षीरनिधिं गत्वा दृष्ट्वा द्रोणं महागिरिम्} %7-33

\twolineshloka
{अदृष्ट्वा चौषधीस्तत्र गिरिमुत्पाट्य सत्वरः}
{गृहीत्वा वायुवेगेन गत्वा रामस्य सन्निधिम्} %7-34

\twolineshloka
{उवाच हनुमान् राममानीतोऽयं महागिरिः}
{यद्युक्तं कुरु देवेश विलम्बो नात्र युज्यते} %7-35

\twolineshloka
{श्रुत्वा हनूमतो वाक्यं रामः सन्तुष्टमानसः}
{गृहीत्वा चौषधीः शीघ्रं सुषेणेन महामतिः} %7-36

\twolineshloka
{चिकित्सां कारयामास लक्ष्मणाय महात्मने}
{ततः सुप्तोत्थित इव बुद्ध्वा प्रोवाच लक्ष्मणः} %7-37

\twolineshloka
{तिष्ठ तिष्ठ क्व गन्तासि हन्मीदानीं दशानन}
{इति ब्रुवन्तमालोक्य मूर्ध्न्यवघ्राय राघवः} %7-38

\twolineshloka
{मारुतिं प्राह वत्साद्य त्वत्प्रसादान्महाकपे}
{निरामयं प्रपश्यामि लक्ष्मणं भ्रातरं मम} %7-39

\twolineshloka
{इत्युक्त्वा वानरैः सार्धं सुग्रीवेण समन्वितः}
{विभीषणमतेनैव युद्धाय समवस्थितः} %7-40

\twolineshloka
{पाषाणैः पादपैश्चैव पर्वताग्रैश्च वानराः}
{युद्धायाभिमुखा भूत्वा ययुः सर्वे युयुत्सवः} %7-41

\twolineshloka
{रावणो विव्यथे रामबाणैर्विद्धो महासुरः}
{मातङ्ग इव सिंहेन गरुडेनेव पन्नगः} %7-42

\twolineshloka
{अभिभूतोऽगमद्राजा राघवेण महात्मना}
{सिंहासने समाविश्य राक्षसानिदमब्रवीत्} %7-43

\twolineshloka
{मानुषेणैव मे मृत्युमाह पूर्वं पितामहः}
{मानुषो हि न मां हन्तुं शक्तोऽस्ति भुवि कश्चन} %7-44

\twolineshloka
{ततो नारायणः साक्षान्मानुषोऽभून्न संशयः}
{रामो दाशरथिर्भूत्वा मां हन्तुं समुपस्थितः} %7-45

\twolineshloka
{अनरण्येन यत्पूर्वं शप्तोऽहं राक्षसेश्वर}
{उत्पत्स्यते च मद्वंशे परमात्मा सनातनः} %7-46

\twolineshloka
{तेन त्वं पुत्रपौत्रैश्च बान्धवैश्च समन्वितः}
{हनिष्यसे न सन्देह इत्युक्त्वा मां दिवं गतः} %7-47

\twolineshloka
{स एव रामः सञ्जातो मदर्थे मां हनिष्यति}
{कुम्भकर्णस्तु मूढात्मा सदा निद्रावशं गतः} %7-48

\twolineshloka
{तं विबोध्य महासत्त्वमानयन्तु ममान्तिकम्}
{इत्युक्तास्ते महाकायास्तूर्णं गत्वा तु यत्नतः} %7-49

\twolineshloka
{विबोध्य कुम्भश्रवणं निन्यू रावणसन्निधिम्}
{नमस्कृत्य स राजानमासनोपरि संस्थितः} %7-50

\twolineshloka
{तमाह रावणो राजा भ्रातरं दीनया गिरा}
{कुम्भकर्ण निबोध त्वं महत्कष्टमुपस्थितम्} %7-51

\twolineshloka
{रामेण निहताः शूराः पुत्राः पौत्राश्च बान्धवाः}
{किं कर्तव्यमिदानीं मे मृत्युकाल उपस्थिते} %7-52

\twolineshloka
{एष दाशरथी रामः सुग्रीवसहितो बली}
{समुद्रं सबलस्तीर्त्वा मूलं नः परिकृन्तति} %7-53

\twolineshloka
{ये राक्षसा मुख्यतमास्ते हता वानरैर्युधि}
{वानराणां क्षयं युद्धे न पश्यामि कदाचन} %7-54

\twolineshloka
{नाशयस्व महाबाहो यदर्थं परिबोधितः}
{भ्रातुरर्थे महासत्त्व कुरु कर्म सुदुष्करम्} %7-55

\twolineshloka
{श्रुत्वा तद्रावणेन्द्रस्य वचनं परिदेवितम्}
{कुम्भकर्णो जहासोच्चैर्वचनं चेदमब्रवीत्} %7-56

\twolineshloka
{पुरा मन्त्रविचारे ते गदितं यन्मया नृप}
{तदद्य त्वामुपगतं फलं पापस्य कर्मणः} %7-57

\twolineshloka
{पूर्वमेव मया प्रोक्तो रामो नारायणः परः}
{सीता च योगमायेति बोधितोऽपि न बुध्यसे} %7-58

\twolineshloka
{एकदाऽहं वने सानौ विशालायां स्थितो निशि}
{दृष्टो मया मुनिः साक्षान्नारदो दिव्यदर्शनः} %7-59

\twolineshloka
{तमब्रवं महाभाग कुतो गन्तासि मे वद}
{इत्युक्तो नारदः प्राह देवानां मन्त्रणे स्थितः} %7-60

\twolineshloka
{तत्रोत्पन्नमुदन्तं ते वक्ष्यामि शृणु तत्त्वतः}
{युवाभ्यां पीडिता देवाः सर्वे विष्णुमुपागताः} %7-61

\twolineshloka
{ऊचुस्ते देवदेवेशं स्तुत्वा भक्त्या समाहिताः}
{जहि रावणमक्षोभ्यं देव त्रैलोक्यकण्टकम्} %7-62

\twolineshloka
{मानुषेण मृतिस्तस्य कल्पिता ब्रह्मणा पुरा}
{अतस्त्वं मानुषो भूत्वा जहि रावणकण्टकम्} %7-63

\twolineshloka
{तथेत्याह महाविष्णुः सत्यसङ्कल्प ईश्वरः}
{जातो रघुकुले देवो राम इत्यभिविश्रुतः} %7-64

\twolineshloka
{स हनिष्यति वः सर्वानित्युक्त्वा प्रययौ मुनिः}
{अतो जानीहि रामं त्वं परं ब्रह्म सनातनम्} %7-65

\twolineshloka
{त्यज वैरं भजस्वाद्य मायामानुषविग्रहम्}
{भजतो भक्तिभावेन प्रसीदति रघूत्तमः} %7-66

\twolineshloka
{भक्तिर्जनित्री ज्ञानस्य भक्तिर्मोक्षप्रदायिनी}
{भक्तिहीनेन यत्किञ्चित्कृतं सर्वमसत्समम्} %7-67

\twolineshloka
{अवताराः सुबहवो विष्णोर्लीलानुकारिणः}
{तेषां सहस्रसदृशो रामो ज्ञानमयः शिवः} %7-68

\twolineshloka
{रामं भजन्ति निपुणा मनसा वचसाऽनिशम्}
{अनायासेन संसारं तीर्त्वा यान्ति हरेः पदम्} %7-69

\fourlineindentedshloka
{ये राममेव सततं भुवि शुद्धसत्त्वा}
{ध्यायन्ति तस्य चरितानि पठन्ति सन्तः}
{मुक्तास्त एव भवभोगमहाहिपाशैः}
{सीतापतेः पदमनन्तसुखं प्रयान्ति} %7-70

\iti{युद्धकाण्डे}{सप्तमः}
%%%%%%%%%%%%%%%%%%%%



\sect{अष्टमः सर्गः}

\uvacha{श्रीमहादेव उवाच}

\twolineshloka
{कुम्भकर्णवचः श्रुत्वा भ्रुकुटीविकटाननः}
{दशग्रीवो जगादेदमासनादुत्पतन्निव} %8-1

\twolineshloka
{त्वमानीतो न मे ज्ञानबोधनाय सुबुद्धिमान्}
{मया कृतं समीकृत्य युध्यस्व यदि रोचते} %8-2

\twolineshloka
{नो चेद्गच्छ सुषुप्त्यर्थं निद्रा त्वां बाधतेऽधुना}
{रावणस्य वचः श्रुत्वा कुम्भकर्णो महाबलः} %8-3

\twolineshloka
{रुष्टोऽयमिति विज्ञाय तूर्णं युद्धाय निर्ययौ}
{स लङ्घयित्वा प्राकारं महापर्वतसन्निभः} %8-4

\twolineshloka
{निर्ययौ नगरात्तूर्णं भीषयन् हरिसैनिकान्}
{स ननाद महानादं समुद्रमभिनादयन्} %8-5

\twolineshloka
{वानरान् कालयामास बाहुभ्यां भक्षयन् रुषा}
{कुम्भकर्णं तदा दृष्ट्वा सपक्षमिव पर्वतम्} %8-6

\twolineshloka
{दुद्रुवुर्वानराः सर्वे कालान्तकमिवाखिलाः}
{भ्रमन्तं हरिवाहिन्यां मुद्गरेण महाबलम्} %8-7

\twolineshloka
{कालयन्तं हरीन् वेगाद्भक्षयन्तं समन्ततः}
{चूर्णयन्तं मुद्गरेण पाणिपादैरनेकधा} %8-8

\twolineshloka
{कुम्भकर्णं तदा दृष्ट्वा गदापाणिर्विभीषणः}
{ननाम चरणं तस्य भ्रातुर्ज्येष्ठस्य बुद्धिमान्} %8-9

\twolineshloka
{विभीषणोऽहं भ्रातुर्मे दयां कुरु महामते}
{रावणस्तु मया भ्रातर्बहुधा परिबोधितः} %8-10

\twolineshloka
{सीतां देहीति रामाय रामः साक्षाज्जनार्दनः}
{न शृणोति च मां हन्तुं खड्गमुद्यम्य चोक्तवान्} %8-11

\twolineshloka
{धिक् त्वां गच्छेति मां हत्वा पदा पापिभिरावृतः}
{चतुर्भिर्मन्त्रिभिः सार्धं रामं शरणमागतः} %8-12

\twolineshloka
{तच्छ्रुत्वा कुम्भकर्णोऽपि ज्ञात्वा भ्रातरमागतम्}
{समालिङ्ग्य च वत्स त्वं जीव रामपदाश्रयात्} %8-13

\twolineshloka
{कुलसंरक्षणार्थाय राक्षसानां हिताय च}
{महाभागवतोऽसि त्वं पुरा मे नारदाच्छ्रुतम्} %8-14

\twolineshloka
{गच्छ तात ममेदानीं दृश्यते न च किञ्चन}
{मदीयो वा परो वाऽपि मदमत्तविलोचनः} %8-15

\twolineshloka
{इत्युक्तोऽश्रुमुखो भ्रातुश्चरणावभिवन्द्य सः}
{रामपार्श्वमुपागत्य चिन्तापर उपस्थितः} %8-16

\twolineshloka
{कुम्भकर्णोऽपि हस्ताभ्यां पादाभ्यां पेषयन् हरीन्}
{चचार वानरीं सेनां कालयन् गन्धहस्तिवत्} %8-17

\twolineshloka
{दृष्ट्वा तं राघवः क्रुद्धो वायव्यं शस्त्रमादरात्}
{चिक्षेप कुम्भकर्णाय तेन चिच्छेद रक्षसः} %8-18

\twolineshloka
{समुद्गरं दक्षहस्तं तेन घोरं ननाद सः}
{स हस्तः पतितो भूमावनेकानर्दयन् कपीन्} %8-19

\twolineshloka
{पर्यन्तमाश्रिताः सर्वे वानरा भयवेपिताः}
{रामराक्षसयोर्युद्धं पश्यन्तः पर्यवस्थिताः} %8-20

\twolineshloka
{कुम्भकर्णश्छिन्नहस्तः शालमुद्यम्य वेगतः}
{समरे राघवं हन्तुं दुद्राव तमथोऽच्छिनत्} %8-21

\twolineshloka
{शालेन सहितं वामहस्तमैन्द्रेण राघवः}
{छिन्नबाहुमथायान्तं नर्दन्तं वीक्ष्य राघवः} %8-22

\twolineshloka
{द्वावर्धचन्द्रौ निशितावादायास्य पदद्वयम्}
{चिच्छेद पतितौ पादौ लङ्काद्वारि महास्वनौ} %8-23

\twolineshloka
{निकृत्तपाणिपादोऽपि कुम्भकर्णोऽतिभीषणः}
{वडवामुखवद्वक्त्रं व्यादाय रघुनन्दनम्} %8-24

\twolineshloka
{अभिदुद्राव निनदन् राहुश्चन्द्रमसं यथा}
{अपूरयच्छिताग्रैश्च सायकैस्तद्रघूत्तमः} %8-25

\twolineshloka
{शरपूरितवक्त्रोऽसौ चुक्रोशातिभयङ्करः}
{अथ सूर्यप्रतीकाशमैन्द्रं शरमनुत्तमम्} %8-26

\twolineshloka
{वज्राशनिसमं रामश्चिक्षेपासुरमृत्यवे}
{स तत्पर्वतसङ्काशं स्फुरत्कुण्डलदंष्ट्रकम्} %8-27

\twolineshloka
{चकर्त रक्षोऽधिपतेः शिरो वृत्रमिवाशनिः}
{तच्छिरः पतितं लङ्काद्वारि कायो महोदधौ} %8-28

\twolineshloka
{शिरोऽस्य रोधयद्द्वारं कायो नक्राद्यचूर्णयत्}
{ततो देवाः सऋषयो गन्धर्वाः पन्नगाः खगाः} %8-29

\twolineshloka
{सिद्धा यक्षा गुह्यकाश्च अप्सरोभिश्च राघवम्}
{ईडिरे कुसुमासारैर्वर्षन्तश्चाभिनन्दिताः} %8-30

\twolineshloka
{आजगाम तदा रामं द्रष्टुं देवमुनीश्वरः}
{नारदो गगनात्तुर्णं स्वभासा भासयन् दिशः} %8-31

\twolineshloka
{राममिन्दीवरश्याममुदाराङ्गं धनुर्धरम्}
{ईषत्ताम्रविशालाक्षमैन्द्रास्त्राञ्चितबाहुकम्} %8-32

\twolineshloka
{दयार्द्रदृष्ट्या पश्यन्तं वानराञ्छरपीडितान्}
{दृष्ट्वा गद्गदया वाचा भक्त्या स्तोतुं प्रचक्रमे} %8-33

\uvacha{नारद उवाच}

\twolineshloka
{देवदेव जगन्नाथ परमात्मन् सनातन}
{नारायणाखिलाधार विश्वसाक्षिन्नमोऽस्तु ते} %8-34

\twolineshloka
{विशुद्धज्ञानरूपोऽपि त्वं लोकानतिवञ्चयन्}
{मायया मनुजाकारः सुखदुःखादिमानिव} %8-35

\twolineshloka
{त्वं मायया गुह्यमानः सर्वेषां हृदि संस्थितः}
{स्वयञ्ज्योतिः स्वभावस्त्वं व्यक्त एवामलात्मनाम्} %8-36

\twolineshloka
{उन्मीलयन् सृजस्येतन्नेत्रे राम जगत्त्रयम्}
{उपसंह्रियते सर्वं त्वया चक्षुर्निमीलनात्} %8-37

\twolineshloka
{यस्मिन् सर्वमिदं भाति यतश्चैतच्चराचरम्}
{यस्मान्न किञ्चिल्लोकेऽस्मिंस्तस्मै ते ब्रह्मणे नमः} %8-38

\twolineshloka
{प्रकृतिं पुरुषं कालं व्यक्ताव्यक्तस्वरूपिणम्}
{यं जानन्ति मुनिश्रेष्ठास्तस्मै रामाय ते नमः} %8-39

\twolineshloka
{विकाररहितं शुद्धं ज्ञानरूपं श्रुतिर्जगौ}
{त्वां सर्वजगदाकारमूर्तिं चाप्याह सा श्रुतिः} %8-40

\twolineshloka
{विरोधो दृश्यते देव वैदिको वेदवादिनाम्}
{निश्चयं नाधिगच्छन्ति त्वत्प्रसादं विना बुधाः} %8-41

\twolineshloka
{मायया क्रीडतो देव न विरोधो मनागपि}
{रश्मिजालं रवेर्यद्वद्दृश्यते जलवद्\mbox{}भ्रमात्} %8-42

\twolineshloka
{भ्रान्तिज्ञानात्तथा राम त्वयि सर्वं प्रकल्प्यते}
{मनसोऽविषयो देव रूपं ते निर्गुणं परम्} %8-43

\twolineshloka
{कथं दृश्यं भवेद्देव दृश्याभावे भजेत्कथम्}
{अतस्तवावतारेषु रूपाणि निपुणा भुवि} %8-44

\twolineshloka
{भजन्ति बुद्धिसम्पन्नास्तरन्त्येव भवार्णवम्}
{कामक्रोधादयस्तत्र बहवः परिपन्थिनः} %8-45

\twolineshloka
{भीषयन्ति सदा चेतो मार्जारा मूषकं यथा}
{त्वन्नाम स्मरतां नित्यं त्वद्रूपमपि मानसे} %8-46

\twolineshloka
{त्वत्पूजानिरतानां ते कथामृतपरात्मनाम्}
{त्वद्भक्तसङ्गिनां राम संसारो गोष्पदायते} %8-47

\twolineshloka
{अतस्ते सगुणं रूपं ध्यात्वाऽहं सर्वदा हृदि}
{मुक्तश्चरामि लोकेषु पूज्योऽहं सर्वदैवतैः} %8-48

\twolineshloka
{राम त्वया महत्कार्यं कृतं देवहितेच्छया}
{कुम्भकर्णवधेनाद्य भूभारोऽयं गतः प्रभो} %8-49

\twolineshloka
{श्वो हनिष्यति सौमित्रिरिन्द्रजेतारमाहवे}
{हनिष्यसेऽथ राम त्वं परश्वो दशकन्धरम्} %8-50

\twolineshloka
{पश्यामि सर्वं देवेश सिद्धैः सह नभोगतः}
{अनुगृह्णीष्व मां देव गमिष्यामि सुरालयम्} %8-51

\twolineshloka
{इत्युक्त्वा राममामन्त्र्य नारदो भगवानृषिः}
{ययौ देवैः पूज्यमानो ब्रह्मलोकमकल्मषम्} %8-52

\twolineshloka
{भ्रातरं निहतं श्रुत्वा कुम्भकर्णं महाबलम्}
{रावणः शोकसन्तप्तो रामेणाक्लिष्टकर्मणा} %8-53

\twolineshloka
{मूर्च्छितः पतितो भूमावुत्थाय विललाप ह}
{पितृव्यं निहतं श्रुत्वा पितरं चातिविह्वलम्} %8-54

\twolineshloka
{इन्द्रजित्प्राह शोकार्तं त्यज शोकं महामते}
{व्येतु ते दुःखमखिलं स्वस्थो भव महीपते} %8-56

\twolineshloka
{सर्वं समीकरिष्यामि हनिष्यामि च वै रिपून्}
{गत्वा निकुम्भिलां सद्यस्तर्पयित्वा हुताशनम्} %8-57

\twolineshloka
{लब्ध्वा रथादिकं तस्मादजेयोऽहं भवाम्यरेः}
{इत्युक्त्वा त्वरितं गत्वा निर्दिष्टं हवनस्थलम्} %8-58

\twolineshloka
{रक्तमाल्याम्बरधरो रक्तगन्धानुलेपनः}
{निकुम्भिलास्थले मौनी हवनायोपचक्रमे} %8-59

\twolineshloka
{विभीषणोऽथ तच्छ्रुत्वा मेघनादस्य चेष्टितम्}
{प्राह रामाय सकलं होमारम्भं दुरात्मनः} %8-60

\twolineshloka
{समाप्यते चेद्धोमोऽयं मेघनादस्य दुर्मतेः}
{तदाऽजेयो भवेद्राम मेघनादः सुरासुरैः} %8-61

\threelineshloka
{अतः शीघ्रं लक्ष्मणेन घातयिष्यामि रावणिम्}
{आज्ञापय मया सार्धं लक्ष्मणं बलिनां वरम्}
{हनिष्यति न सन्देहो मेघनादं तवानुजः} %8-62

\uvacha{श्रीरामचन्द्र उवाच}

\twolineshloka
{अहमेवागमिष्यामि हन्तुमिन्द्रजितं रिपुम्}
{आग्नेयेन महास्त्रेण सर्वराक्षसघातिना} %8-63

\twolineshloka
{विभीषणोऽपि तं प्राह नासावन्यैर्निहन्यते}
{यस्तु द्वादश वर्षाणि निद्राहारविवर्जितः} %8-64

\twolineshloka
{तेनैव मृत्युर्निर्दिष्टो ब्रह्मणाऽस्य दुरात्मनः}
{लक्ष्मणस्तु अयोध्याया निर्गम्यायात्त्वया सह} %8-65

\twolineshloka
{तदादि निद्राहारादीन्न जानाति रघूत्तम}
{सेवार्थं तव राजेन्द्र ज्ञातं सर्वमिदं मया} %8-66

\twolineshloka
{तदाज्ञापय देवेश लक्ष्मणं त्वरया मया}
{हनिष्यति न सन्देहः शेषः साक्षाद्धराधरः} %8-67

\fourlineindentedshloka
{त्वमेव साक्षाज्जगतामधीशो}
{नारायणो लक्ष्मण एव शेषः}
{युवां धराभारनिवारणार्थम्}
{जातौ जगन्नाटकसूत्रधारौ} %8-68

\iti{युद्धकाण्डे}{अष्टमः}
%%%%%%%%%%%%%%%%%%%%



\sect{नवमः सर्गः}

\uvacha{श्रीमहादेव उवाच}

\twolineshloka
{विभीषणवचः श्रुत्वा रामो वाक्यमथाब्रवीत्}
{जानामि तस्य रौद्रस्य मायां कृत्स्नां विभीषण} %9-1

\twolineshloka
{स हि ब्रह्मास्त्रविच्छूरो मायावी च महाबलः}
{जानामि लक्ष्मणस्यापि स्वरूपं मम सेवनम्} %9-2

\twolineshloka
{ज्ञात्वैवासमहं तूष्णीं भविष्यत्कार्यगौरवात्}
{इत्युक्त्वा लक्ष्मणं प्राह रामो ज्ञानवतां वरः} %9-3

\twolineshloka
{गच्छ लक्ष्मण सैन्येन महता जहि रावणिम्}
{हनूमत्प्रमुखैः सर्वैर्यूथपैः सह लक्ष्मण} %9-4

\twolineshloka
{जाम्बवानृक्षराजोऽयं सह सैन्येन संवृतः}
{विभीषणश्च सचिवैः सह त्वामभियास्यति} %9-5

\twolineshloka
{अभिज्ञस्तस्य देशस्य जानाति विवराणि सः}
{रामस्य वचनं श्रुत्वा लक्ष्मणः सविभीषणः} %9-6

\twolineshloka
{जग्राह कार्मुकं श्रेष्ठमन्यद्भीमपराक्रमः}
{रामपादाम्बुजं स्पृष्ट्वा हृष्टः सौमित्रिरब्रवीत्} %9-7

\twolineshloka
{अद्य मत्कार्मुकान्मुक्ताः शरा निर्भिद्य रावणिम्}
{गमिष्यन्ति हि पातालं स्नातुं भोगवतीजले} %9-8

\twolineshloka
{एवमुक्त्वा स सौमित्रिः परिक्रम्य प्रणम्य तम्}
{इन्द्रजिन्निधनाकाङ्क्षी ययौ त्वरितविक्रमः} %9-9

\twolineshloka
{वानरैर्बहुसाहस्रैर्हनूमान् पृष्ठतोऽन्वगात्}
{विभीषणश्च सहितो मन्त्रिभिस्त्वरितं ययौ} %9-10

\twolineshloka
{जाम्बवत्प्रमुखा ऋक्षाः सौमित्रिं त्वरयान्वयुः}
{गत्वा निकुम्भिलादेशं लक्ष्मणो वानरैः सह} %9-11

\twolineshloka
{अपश्यद्बलसङ्घातं दूराद्राक्षससङ्कुलम्}
{धनुरायम्य सौमित्रिर्यत्तोऽभूद्भूरिविक्रमः} %9-12

\twolineshloka
{अङ्गदेन च वीरेण जाम्बवान् राक्षसाधिपः}
{तदा विभीषणः प्राह सौमित्रिं पश्य राक्षसान्} %9-13

\twolineshloka
{यदेतद्राक्षसानीकं मेघश्यामं विलोक्यते}
{अस्यानीकस्य महतो भेदने यत्नवान् भव} %9-14

\twolineshloka
{राक्षसेन्द्रसुतोऽप्यस्मिन् भिन्ने दृश्यो भविष्यति}
{अभिद्रवाऽऽशु यावद्वै नैतत्कर्म समाप्यते} %9-15

\twolineshloka
{जहि वीर दुरात्मानं हिंसापरमधार्मिकम्}
{विभीषणवचः श्रुत्वा लक्ष्मणः शुभलक्ष्मणः} %9-16

\twolineshloka
{ववर्ष शरवर्षाणि राक्षसेन्द्रसुतं प्रति}
{पाषाणैः पर्वताग्रैश्च वृक्षैश्च हरियूथपाः} %9-17

\twolineshloka
{निर्जघ्नुः सर्वतो दैत्यांस्तेऽपि वानरयूथपान्}
{परश्वधैः शितैर्बाणैरसिभिर्यष्टितोमरैः} %9-18

\twolineshloka
{निर्जघ्नुर्वानरानीकं तदा शब्दो महानभूत्}
{स सम्प्रहारस्तुमुलः सञ्जज्ञे हरिरक्षसाम्} %9-19

\twolineshloka
{इन्द्रजित्स्वबलं सर्वमर्द्यमानं विलोक्य सः}
{निकुम्भिलां च होमं च त्यक्त्वा शीघ्रं विनिर्गतः} %9-20

\twolineshloka
{रथमारुह्य सधनुः क्रोधेन महताऽऽगमत्}
{समाह्वयन् स सौमित्रिं युद्धाय रणमूर्धनि} %9-21

\twolineshloka
{सौमित्रे मेघनादोऽहं मया जीवन्न मोक्ष्यसे}
{तत्र दृष्ट्वा पितृव्यं स प्राह निष्ठुरभाषणम्} %9-22

\twolineshloka
{इहैव जातः संवृद्धः साक्षाद्\mbox{}भ्राता पितुर्मम}
{यस्त्वं स्वजनमुत्सृज्य परभृत्यत्वमागतः} %9-23

\twolineshloka
{कथं द्रुह्यसि पुत्राय पापीयानसि दुर्मतिः}
{इत्युक्त्वा लक्ष्मणं दृष्ट्वा हनूमत्पृष्ठतः स्थितम्} %9-24

\twolineshloka
{उद्यदायुधनिस्त्रिंशे रथे महति संस्थितः}
{महाप्रमाणमुद्यम्य घोरं विस्फारयन् धनुः} %9-25

\twolineshloka
{अद्य वो मामका बाणाः प्राणान् पास्यन्ति वानराः}
{ततः शरं दाशरथिः सन्धायामित्रकर्षणः} %9-26

\twolineshloka
{ससर्ज राक्षसेन्द्राय क्रुद्धः सर्प इव श्वसन्}
{इन्द्रजिद्रक्तनयनो लक्ष्मणं समुदैक्षत} %9-27

\twolineshloka
{शक्राशनिसमस्पर्शैर्लक्ष्मणेनाहतः शरैः}
{मुहूर्तमभवन्मूढः पुनः प्रत्याहृतेन्द्रियः} %9-28

\twolineshloka
{ददर्शावस्थितं वीरं वीरो दशरथात्मजम्}
{सोऽभिचक्राम सौमित्रिं क्रोधसंरक्तलोचनः} %9-29

\twolineshloka
{शरान् धनुषि सन्धाय लक्ष्मणं चेदमब्रवीत्}
{यदि ते प्रथमे युद्धे न दृष्टो मे पराक्रमः} %9-30

\twolineshloka
{अद्य त्वां दर्शयिष्यामि तिष्ठेदानीं व्यवस्थितः}
{इत्युक्त्वा सप्तभिर्बाणैरभिविव्याध लक्ष्मणम्} %9-31

\twolineshloka
{दशभिश्च हनूमन्तं तीक्ष्णधारैः शरोत्तमैः}
{ततः शरशतेनैव सम्प्रयुक्तेन वीर्यवान्} %9-32

\twolineshloka
{क्रोधद्विगुणसंरब्धो निर्बिभेद विभीषणम्}
{लक्ष्मणोऽपि तथा शत्रुं शरवर्षैरवाकिरत्} %9-33

\twolineshloka
{तस्य बाणैः सुसंविद्धं कवचं काञ्चनप्रभम्}
{व्यशीर्यत रथोपस्थे तिलशः पतितं भुवि} %9-34

\twolineshloka
{ततः शरसहस्रेण सङ्क्रुद्धो रावणात्मजः}
{बिभेद समरे वीरं लक्ष्मणं भीमविक्रमम्} %9-35

\twolineshloka
{व्यशीर्यतापतद्दिव्यं कवचं लक्ष्मणस्य च}
{कृतप्रतिकृतान्योन्यं बभूवतुरभिद्रुतौ} %9-36

\twolineshloka
{अभीक्ष्णं निःश्वसन्तौ तौ युध्येतां तुमुलं पुनः}
{शरसंवृतसर्वाङ्गौ सर्वतो रुधिरोक्षितौ} %9-37

\twolineshloka
{सुदीर्घकालं तौ वीरावन्योन्यं निशितैः शरैः}
{अयुध्येतां महासत्त्वौ जयाजयविवर्जितौ} %9-38

\twolineshloka
{एतस्मिन्नन्तरे वीरो लक्ष्मणः पञ्चभिः शरैः}
{रावणेः सारथिं साश्वं रथं च समचूर्णयत्} %9-39

\twolineshloka
{चिच्छेद कार्मुकं तस्य दर्शयन् हस्तलाघवम्}
{सोऽन्यत्तु कार्मुकं भद्रं सज्यं चक्रे त्वरान्वितः} %9-40

\twolineshloka
{तच्चापमपि चिच्छेद लक्ष्मणस्त्रिभिराशुगैः}
{तमेव छिन्नधन्वानं विव्याधानेकसायकैः} %9-41

\twolineshloka
{पुनरन्यत्समादाय कार्मुकं भीमविक्रमः}
{इन्द्रजिल्लक्ष्मणं बाणैः शितैरादित्यसन्निभैः} %9-42

\twolineshloka
{बिभेद वानरान् सर्वान् बाणैरापूरयन् दिशः}
{तत ऐन्द्रं समादाय लक्ष्मणो रावणिं प्रति} %9-43

\twolineshloka
{सन्धायाकृष्य कर्णान्तं कार्मुकं दृढनिष्ठुरम्}
{उवाच लक्ष्मणो वीरः स्मरन् रामपदाम्बुजम्} %9-44

\twolineshloka
{धर्मात्मा सत्यसन्धश्च रामो दाशरथिर्यदि}
{त्रिलोक्यामप्रतिद्वन्द्वस्तदेनं जहि रावणिम्} %9-45

\twolineshloka
{इत्युक्त्वा बाणमाकर्णाद्विकृष्य तमजिह्मगम्}
{लक्ष्मणः समरे वीरः ससर्जेन्द्रजितं प्रति} %9-46

\twolineshloka
{स शरः सशिरस्त्राणं श्रीमज्ज्वलितकुण्डलम्}
{प्रमथ्येन्द्रजितः कायात्पातयामास भूतले} %9-47

\twolineshloka
{ततः प्रमुदिता देवाः कीर्तयन्तो रघूत्तमम्}
{ववर्षुः पुष्पवर्षाणि स्तुवन्तश्च मुहुर्मुहुः} %9-48

\twolineshloka
{जहर्ष शक्रो भगवान् सह देवैर्महर्षिभिः}
{आकाशेऽपि च देवानां शुश्रुवे दुन्दुभिस्वनः} %9-49

\twolineshloka
{विमलं गगनं चाऽऽसीत्स्थिराऽभूद्विश्वधारिणी}
{निहतं रावणिं दृष्ट्वा जयजल्पसमन्वितः} %9-50

\twolineshloka
{गतश्रमः स सौमित्रिः शङ्खमापूरयद्रणे}
{सिंहनादं ततः कृत्वा ज्याशब्दमकरोद्विभुः} %9-51

\twolineshloka
{तेन नादेन संहृष्टा वानराश्च गतश्रमाः}
{वानरेन्द्रैश्च सहितः स्तुवद्भिर्हृष्टमानसैः} %9-52

\twolineshloka
{लक्ष्मणः परितुष्टात्मा ददर्शाभ्येत्य राघवम्}
{हनूमद्राक्षसाभ्यां च सहितो विनयान्वितः} %9-53

\twolineshloka
{ववन्दे भ्रातरं रामं ज्येष्ठं नारायणं विभुम्}
{त्वत्प्रसादाद्रघुश्रेष्ठ हतो रावणिराहवे} %9-54

\twolineshloka
{श्रुत्वा तल्लक्ष्मणाद्भक्त्या तमालिङ्ग्य रघूत्तमः}
{मूर्ध्न्यवघ्राय मुदितः सस्नेहमिदमब्रवीत्} %9-55

\twolineshloka
{साधु लक्ष्मण तुष्टोऽस्मि कर्म ते दुष्करं कृतम्}
{मेघनादस्य निधने जितं सर्वमरिन्दम} %9-56

\twolineshloka
{अहोरात्रैस्त्रिभिर्वीरः कथञ्चिद्विनिपातितः}
{निःसपत्नः कृतोऽस्म्यद्य निर्यास्यति हि रावणः} %9-57

{पुत्रशोकान्मया योद्धुं तं हनिष्यामि रावणम्} %9-58


\threelineshloka
{मेघनादं हतं श्रुत्वा लक्ष्मणेन महाबलम्}
{रावणः पतितो भुमौ मूर्च्छितः पुनरुत्थितः}
{विललापातिदीनात्मा पुत्रशोकेन रावणः} %9-59

\twolineshloka
{पुत्रस्य गुणकर्माणि संस्मरन् पर्यदेवयत्}
{अद्य देवगणाः सर्वे लोकपाला महर्षयः} %9-60

\twolineshloka
{हतमिन्द्रजितं ज्ञात्वा सुखं स्वप्स्यन्ति निर्भयाः}
{इत्यादि बहुशः पुत्रलालसो विललाप ह} %9-61

\twolineshloka
{ततः परमसङ्क्रुद्धो रावणो राक्षसाधिपः}
{उवाच राक्षसान् सर्वान्निनाशयिषुराहवे} %9-62

\twolineshloka
{स पुत्रवधसन्तप्तः शूरः क्रोधवशं गतः}
{संवीक्ष्य रावणो बुद्ध्या हन्तुं सीतां प्रदुद्रुवे} %9-63

\twolineshloka
{खड्गपाणिमथायान्तं क्रुद्धं दृष्ट्वा दशाननम्}
{राक्षसीमध्यगा सीता भयशोकाकुलाभवत्} %9-64

\twolineshloka
{एतस्मिन्नन्तरे तस्य सचिवो बुद्धिमान् शुचिः}
{सुपार्श्वो नाम मेधावी रावणं वाक्यमब्रवीत्} %9-65

\twolineshloka
{ननु नाम दशग्रीव साक्षाद्वैश्रवणानुजः}
{वेदविद्याव्रतस्नातः स्वकर्मपरिनिष्ठितः} %9-66

\threelineshloka
{अनेकगुणसम्पन्नः कथं स्त्रीवधमिच्छसि}
{अस्माभिः सहितो युद्धे हत्वा रामं च लक्ष्मणम्}
{प्राप्स्यसे जानकीं शीघ्रमित्युक्तः स न्यवर्तत} %9-67

\fourlineindentedshloka
{ततो दुरात्मा सुहृदा निवेदितम्}
{वचः सुधर्म्यं प्रतिगृह्य रावणः}
{गृहं जगामाऽऽशु शुचा विमूढधीः}
{पुनः सभां च प्रययौ सुहृद्वृतः} %9-68

\iti{युद्धकाण्डे}{नवमः}
%%%%%%%%%%%%%%%%%%%%



\sect{दशमः सर्गः}

\uvacha{श्रीमहादेव उवाच}

\twolineshloka
{स विचार्य सभामध्ये राक्षसैः सह मन्त्रिभिः}
{निर्ययौ येऽवशिष्टास्तै राक्षसैः सह राघवम्} %10-1

\twolineshloka
{शलभः शलभैर्युक्तः प्रज्वलन्तमिवानलम्}
{ततो रामेण निहताः सर्वे ते राक्षसा युधि} %10-2

\twolineshloka
{स्वयं रामेण निहतस्तीक्ष्णबाणेन वक्षसि}
{व्यथितस्त्वरितं लङ्कां प्रविवेश दशाननः} %10-3

\twolineshloka
{दृष्ट्वा रामस्य बहुशः पौरुषं चाप्यमानुषम्}
{रावणो मारुतेश्चैव शीघ्रं शुक्रान्तिकं ययौ} %10-4

\twolineshloka
{नमस्कृत्य दशग्रीवः शुक्रं प्राञ्जलिरब्रवीत्}
{भगवन् राघवेणैवं लङ्का राक्षसयूथपैः} %10-5

\twolineshloka
{विनाशिता महादैत्या निहताः पुत्रबान्धवाः}
{कथं मे दुःखसन्दोहस्त्वयि तिष्ठति सद्गुरौ} %10-6

\twolineshloka
{इति विज्ञापितो दैत्यगुरुः प्राह दशाननम्}
{होमं कुरु प्रयत्नेन रहसि त्वं दशानन} %10-7

{यदि विघ्नो न चेद्धोमे तर्हि होमानलोत्थितः} %10-8


\twolineshloka
{महान् रथश्च वाहाश्च चापतूणीरसायकाः}
{सम्भविष्यन्ति तैर्युक्तस्त्वमजेयो भविष्यसि} %10-9

\twolineshloka
{गृहाण मन्त्रान् मद्दत्तान् गच्छ होमं कुरु द्रुतम्}
{इत्युक्तस्त्वरितं गत्वा रावणो राक्षसाधिपः} %10-10

\twolineshloka
{गुहां पातालसदृशीं मन्दिरे स्वे चकार ह}
{लङ्काद्वारकपाटादि बद्ध्वा सर्वत्र यत्नतः} %10-11

\twolineshloka
{होमद्रव्याणि सम्पाद्य यान्युक्तान्याभिचारिके}
{गुहां प्रविश्य चैकान्ते मौनी होमं प्रचक्रमे} %10-12

\twolineshloka
{उत्थितं धूममालोक्य महान्तं रावणानुजः}
{रामाय दर्शयामास होमधूमं भयाकुलः} %10-13

\twolineshloka
{पश्य राम दशग्रीवो होमं कर्तुं समारभत्}
{यदि होमः समाप्तः स्यात्तदाऽजेयो भविष्यति} %10-14

\twolineshloka
{अतो विघ्नाय होमस्य प्रेषयाऽऽशु हरीश्वरान्}
{तथेति रामः सुग्रीवसम्मतेनाङ्गदं कपिम्} %10-15

\twolineshloka
{हनूमत्प्रमुखान् वीरानादिदेश महाबलान्}
{प्राकारं लङ्घयित्वा ते गत्वा रावणमन्दिरम्} %10-16

\twolineshloka
{दशकोट्यः प्लवङ्गानां गत्वा मन्दिररक्षकान्}
{चूर्णयामासुरश्वांश्च गजांश्च न्यहनन् क्षणात्} %10-17

\twolineshloka
{ततश्च सरमा नाम प्रभाते हस्तसंज्ञया}
{विभीषणस्य भार्या सा होमस्थानमसूचयत्} %10-18

\twolineshloka
{गुहापिधानपाषाणमङ्गदः पादघट्टनैः}
{चूर्णयित्वा महासत्त्वः प्रविवेश महागुहाम्} %10-19

\twolineshloka
{दृष्ट्वा दशाननं तत्र मीलिताक्षं दृढासनम्}
{ततोऽङ्गदाज्ञया सर्वे वानरा विविशुर्द्रुतम्} %10-20

\twolineshloka
{तत्र कोलाहलं चक्रुस्ताडयन्तश्च सेवकान्}
{सम्भारांश्चिक्षिपुस्तस्य होमकुण्डे समन्ततः} %10-21

\twolineshloka
{स्रुवमाच्छिद्य हस्ताच्च रावणस्य बलाद्रुषा}
{तेनैव सञ्जघानाशु हनूमान् प्लवगाग्रणीः} %10-22

\twolineshloka
{घ्नन्ति दन्तैश्च काष्ठैश्च वानरास्तमितस्ततः}
{न जहौ रावणो ध्यानं हतोऽपि विजिगीषया} %10-23

\twolineshloka
{प्रविश्यान्तःपुरे वेश्मन्यङ्गदो वेगवत्तरः}
{समानयत्केशबन्धे धृत्वा मन्दोदरीं शुभाम्} %10-24

\twolineshloka
{रावणस्यैव पुरतो विलपन्तीमनाथवत्}
{विददाराङ्गदस्तस्याः कञ्जुकं रत्नभूषितम्} %10-25

\twolineshloka
{मुक्ता विमुक्ताः पतिताः समन्ताद्रत्नसञ्चयैः}
{श्रोणिसूत्रं निपतितं त्रुटितं रत्नचित्रितम्} %10-26

\twolineshloka
{कटिप्रदेशाद्विस्रस्ता नीवी तस्यैव पश्यतः}
{भूषणानि च सर्वाणि पतितानि समन्ततः} %10-27

\twolineshloka
{देवगन्धर्वकन्याश्च नीता हृष्टैः प्लवङ्गमैः}
{मन्दोदरी रुरोदाथ रावणस्याग्रतो भृशम्} %10-28

\twolineshloka
{क्रोशन्ती करुणं दीना जगाद दशकन्धरम्}
{निर्लज्जोऽसि परैरेवं केशपाशे विकृष्यते} %10-29

\twolineshloka
{भार्या तवैव पुरतः किं जुहोषि न लज्जसे}
{हन्यते पश्यतो यस्य भार्या पापैश्च शत्रुभिः} %10-30

\twolineshloka
{मर्तव्यं तेन तत्रैव जीवितान्मरणं वरम्}
{हा मेघनाद ते माता क्लिश्यते बत वानरैः} %10-31

\twolineshloka
{त्वयि जीवति मे दुःखमीदृशं च कथं भवेत्}
{भार्या लज्जा च सन्त्यक्ता भर्त्रा मे जीविताशया} %10-32

\twolineshloka
{श्रुत्वा तद्देवितं राजा मन्दोदर्या दशाननः}
{उत्तस्थौ खड्गमादाय त्यज देवीमिति ब्रुवन्} %10-33

\twolineshloka
{जघानाङ्गदमव्यग्रः कटिदेशे दशाननः}
{तदोत्सृज्य ययुः सर्वे विध्वंस्य हवनं महत्} %10-34

{रामपार्श्वमुपागम्य तस्थुः सर्वे प्रहर्षिताः} %10-35


\threelineshloka
{रावणस्तु ततो भार्यामुवाच परिसान्त्वयन्}
{दैवाधीनमिदं भद्रे जीवता किं न दृश्यते}
{त्यज शोकं विशालाक्षि ज्ञानमालम्ब्य निश्चितम्} %10-36

\twolineshloka
{अज्ञानप्रभवः शोकः शोको ज्ञानविनाशकृत्}
{अज्ञानप्रभवाहन्धीः शरीरादिष्वनात्मसु} %10-37

\twolineshloka
{तन्मूलः पुत्रदारादिसम्बन्धः संसृतिस्ततः}
{हर्षशोकभयक्रोधलोभमोहस्पृहादयः} %10-38

\twolineshloka
{अज्ञानप्रभवा ह्येते जन्ममृत्युजरादयः}
{आत्मा तु केवलं शुद्धो व्यतिरिक्तो ह्यलेपकः} %10-39

\twolineshloka
{आनन्दरूपो ज्ञानात्मा सर्वभावविवर्जितः}
{न संयोगो वियोगो वा विद्यते केनचित्सतः} %10-40

\twolineshloka
{एवं ज्ञात्वा स्वमात्मानं त्यज शोकमनिन्दिते}
{इदानीमेव गच्छामि हत्वा रामं सलक्ष्मणम्} %10-41

\twolineshloka
{आगमिष्यामि नो चेन्मां दारयिष्यति सायकैः}
{श्रीरामो वज्रकल्पैश्च ततो गच्छामि तत्पदम्} %10-42

\twolineshloka
{तदा त्वया मे कर्तव्या क्रिया मच्छासनात्प्रिये}
{सीतां हत्वा मया सार्धं त्वं प्रवेक्ष्यसि पावकम्} %10-43

\twolineshloka
{एवं श्रुत्वा वचस्तस्य रावणस्यातिदुःखिता}
{उवाच नाथ मे वाक्यं शृणु सत्यं तथा कुरु} %10-44

\twolineshloka
{शक्यो न राघवो जेतुं त्वया चान्यैः कदाचन}
{रामो देववरः साक्षात्प्रधानपुरुषेश्वरः} %10-45

\twolineshloka
{मत्स्यो भूत्वा पुरा कल्पे मनुं वैवस्वतं प्रभुः}
{ररक्ष सकलापद्भ्यो राघवो भक्तवत्सलः} %10-46

\twolineshloka
{रामः कूर्मोऽभवत्पूर्वं लक्षयोजनविस्तृतः}
{समुद्रमथने पृष्ठे दधार कनकाचलम्} %10-47

\twolineshloka
{हिरण्याक्षोऽतिदुर्वृत्तो हतोऽनेन महात्मना}
{क्रोडरूपेण वपुषा क्षोणीमुद्धरता क्वचित्} %10-48

\twolineshloka
{त्रिलोककण्टकं दैत्यं हिरण्यकशिपुं पुरा}
{हतवान्नारसिंहेन वपुषा रघुनन्दनः} %10-49

\twolineshloka
{विक्रमैस्त्रिभिरेवासौ बलिं बद्ध्वा जगत्त्रयम्}
{आक्रम्यादात्सुरेन्द्राय भृत्याय रघुसत्तमः} %10-50

\twolineshloka
{राक्षसाः क्षत्रियाकारा जाता भूमेर्भरावहाः}
{तान् हत्वा बहुशो रामो भुवं जित्वा ह्यदान्मुनेः} %10-51

\twolineshloka
{स एव साम्प्रतं जातो रघुवंशे परात्परः}
{भवदर्थे रघुश्रेष्ठो मानुषत्वमुपागतः} %10-52

\twolineshloka
{तस्य भार्या किमर्थं वा हृता सीता वनाद्बलात्}
{मम पुत्रविनाशार्थं स्वस्यापि निधनाय च} %10-53

\twolineshloka
{इतः परं वा वैदेहीं प्रेषयस्व रघूत्तमे}
{विभीषणाय राज्यं तु दत्त्वा गच्छामहे वनम्} %10-54

\twolineshloka
{मन्दोदरीवचः श्रुत्वा रावणो वाक्यमब्रवीत्}
{कथं भद्रे रणे पुत्रान् भ्रातॄन् राक्षसमण्डलम्} %10-55

\twolineshloka
{घातयित्वा राघवेण जीवामि वनगोचरः}
{रामेण सह योत्स्यामि रामबाणैः सुशीघ्रगैः} %10-56

\threelineshloka
{विदार्यमाणो यास्यामि तद्विष्णोः परमं पदम्}
{जानामि राघवं विष्णुं लक्ष्मीं जानामि जानकीम्}
{ज्ञात्वैव जानकी सीता मयाऽऽनीता वनाद्बलात्} %10-57

\twolineshloka
{रामेण निधनं प्राप्य यास्यामीति परं पदम्}
{विमुच्य त्वां तु संसाराद्गमिष्यामि सह प्रिये} %10-58

\twolineshloka
{परानन्दमयी शुद्धा सेव्यते या मुमुक्षुभिः}
{तां गतिं तु गमिष्यामि हतो रामेण संयुगे} %10-59

\onelineshloka
{प्रक्षाल्य कल्मषाणीह मुक्तिं यास्यामि दुर्लभाम्} %10-60

\fourlineindentedshloka
{क्लेशादिपञ्चकतरङ्गयुतं भ्रमाढ्यम्}
{दारात्मजाप्तधनबन्धुझषाभियुक्तम्}
{और्वानलाभनिजरोषमनङ्गजालम्}
{संसारसागरमतीत्य हरिं व्रजामि} %10-61

\iti{युद्धकाण्डे}{दशमः}
%%%%%%%%%%%%%%%%%%%%



\sect{एकादशः सर्गः}

\uvacha{श्रीमहादेव उवाच}

\twolineshloka
{इत्युक्त्वा वचनं प्रेम्णा राज्ञीं मन्दोदरीं तदा}
{रावणः प्रययौ योद्धुं रामेण सह संयुगे} %11-1

\twolineshloka
{दृढं स्यन्दनमास्थाय वृतो घोरैर्निशाचरैः}
{चक्रैः षोडशभिर्युक्तं सवरूथं सकूबरम्} %11-2

\twolineshloka
{पिशाचवदनैर्घोरैः खरैर्युक्तं भयावहम्}
{सर्वास्त्रशस्त्रसहितं सर्वोपस्करसंयुतम्} %11-3

\twolineshloka
{निश्चक्रामाथ सहसा रावणो भीषणाकृतिः}
{आयान्तं रावणं दृष्ट्वा भीषणं रणकर्कशम्} %11-4

{सन्त्रस्ताऽभूत्तदा सेना वानरी रामपालिता} %11-5


\twolineshloka
{हनूमानथ चोत्प्लुत्य रावणं योद्धुमाययौ}
{आगत्य हनुमान् रक्षोवक्षस्यतुलविक्रमः} %11-6

\twolineshloka
{मुष्टिबन्धं दृढं बद्ध्वा ताडयामास वेगतः}
{तेन मुष्टिप्रहारेण जानुभ्यामपतद्रथे} %11-7

\twolineshloka
{मूर्च्छितोऽथ मुहूर्तेन रावणः पुनरुत्थितः}
{उवाच च हनूमन्तं शूरोऽसि मम सम्मतः} %11-8

\twolineshloka
{हनूमानाह तं धिङ्मां यस्त्वं जीवसि रावण}
{त्वं तावन्मुष्टिना वक्षो मम ताडय रावण} %11-9

\twolineshloka
{पश्चान्मया हतः प्राणान्मोक्ष्यसे नात्र संशयः}
{तथेति मुष्टिना वक्षो रावणेनापि ताडितः} %11-10

\twolineshloka
{विघूर्णमाननयनः किञ्चित्कश्मलमाययौ}
{संज्ञामवाप्य कपिराड् रावणं हन्तुमुद्यतः} %11-11

\twolineshloka
{ततोऽन्यत्र गतो भीत्या रावणो राक्षसाधिपः}
{हनूमानङ्गदश्चैव नलो नीलस्तथैव च} %11-12

\twolineshloka
{चत्वारः समवेत्याग्रे दृष्ट्वा राक्षसपुङ्गवान्}
{अग्निवर्णं तथा सर्परोमाणं खड्गरोमकम्} %11-13

\threelineshloka
{तथा वृश्चिकरोमाणं निर्जघ्नुः क्रमशोऽसुरान्}
{चत्वारश्चतुरो हत्वा राक्षसान् भीमविक्रमान्}
{सिंहनादं पृथक् कृत्वा रामपार्श्वमुपागताः} %11-14

\onelineshloka
{ततः क्रुद्धो दशग्रीवः सन्दश्य दशनच्छदम्} %11-15

\twolineshloka
{विवृत्य नयने क्रूरो राममेवान्वधावत}
{दशग्रीवो रथस्थस्तु रामं वज्रोपमैः शरैः} %11-16

\twolineshloka
{आजघान महाघोरैर्धाराभिरिव तोयदः}
{रामस्य पुरतः सर्वान् वानरानपि विव्यधे} %11-17

\twolineshloka
{ततः पावकसङ्काशैः शरैः काञ्चनभूषणैः}
{अभ्यवर्षद्रणे रामो दशग्रीवं समाहितः} %11-18

\twolineshloka
{रथस्थं रावणं दृष्ट्वा भूमिष्ठं रघुनन्दनम्}
{आहूय मातलिं शक्रो वचनं चेदमब्रवीत्} %11-19

\twolineshloka
{रथेन मम भूमिष्ठं शीघ्रं याहि रघूत्तमम्}
{त्वरितं भूतलं गत्वा कुरु कार्यं ममानघ} %11-20

\twolineshloka
{एवमुक्तोऽथ तं नत्वा मातलिर्देवसारथिः}
{ततो हयैश्च संयोज्य हरितैः स्यन्दनोत्तमम्} %11-21

\twolineshloka
{स्वर्गाज्जयार्थं रामस्य ह्युपचक्राम मातलिः}
{प्राञ्जलिर्देवराजेन प्रेषितोऽस्मि रघूत्तम} %11-22

\twolineshloka
{रथोऽयं देवराजस्य विजयाय तव प्रभो}
{प्रेषितश्च महाराज धनुरैन्द्रं च भूषितम्} %11-23

\twolineshloka
{अभेद्यं कवचं खड्गं दिव्यतूणीयुगं तथा}
{आरुह्य च रथं राम रावणं जहि राक्षसम्} %11-24

\twolineshloka
{मया सारथिना देव वृत्रं देवपतिर्यथा}
{इत्युक्तस्तं परिक्रम्य नमस्कृत्य रथोत्तमम्} %11-25

\twolineshloka
{आरुरोह रथं रामो लोकाँल्लक्ष्म्या नियोजयन्}
{ततोऽभवन्महायुद्धं भैरवं रोमहर्षणम्} %11-26

\twolineshloka
{महात्मनो राघवस्य रावणस्य च धीमतः}
{आग्नेयेन च आग्नेयं दैवं दैवेन राघवः} %11-27

\threelineshloka
{अस्त्रं राक्षसराजस्य जघान परमास्त्रवित्}
{ततस्तु ससृजे घोरं राक्षसं चास्त्रमस्त्रवित्}
{क्रोधेन महताऽऽविष्टो रामस्योपरि रावणः} %11-28

\twolineshloka
{रावणस्य धनुर्मुक्ताः सर्पा भूत्वा महाविषाः}
{शराः काञ्चनपुङ्खाभा राघवं परितोऽपतन्} %11-29

\twolineshloka
{तैः शरैः सर्पवदनैर्वमद्भिरनलं मुखैः}
{दिशश्च विदिशश्चैव व्याप्तास्तत्र तदाऽभवन्} %11-30

\twolineshloka
{रामः सर्पांस्ततो दृष्ट्वा समन्तात्परिपूरितान्}
{सौपर्णमस्त्रं तद्\mbox{}घोरं पुरः प्रावर्तयद्रणे} %11-31

\twolineshloka
{रामेण मुक्तास्ते बाणा भूत्वा गरुडरूपिणः}
{चिच्छिदुः सर्पबाणांस्तान् समन्तात् सर्पशत्रवः} %11-32

\twolineshloka
{अस्त्रे प्रतिहते युद्धे रामेण दशकन्धरः}
{अभ्यवर्षत्ततो रामं घोराभिः शरवृष्टिभिः} %11-33

\twolineshloka
{ततः पुनः शरानीकै राममक्लिष्टकारिणम्}
{अर्दयित्वा तु घोरेण मातलिं प्रत्यविध्यत} %11-34

\twolineshloka
{पातयित्वा रथोपस्थे रथकेतुं च काञ्चनम्}
{ऐन्द्रानश्वानभ्यहनद्रावणः क्रोधमूर्च्छितः} %11-35

\twolineshloka
{विषेदुर्देवगन्धर्वाश्चारणाः पितरस्तथा}
{आर्त्ताकारं हरिं दृष्ट्वा व्यथिताश्च महर्षयः} %11-36

\twolineshloka
{व्यथिता वानरेन्द्राश्च बभूवुः सविभीषणाः}
{दशास्यो विंशतिभुजः प्रगृहीतशरासनः} %11-37

\twolineshloka
{ददृशे रावणस्तत्र मैनाक इव पर्वतः}
{रामस्तु भ्रुकुटिं बद्ध्वा क्रोधसंरक्तलोचनः} %11-38

\twolineshloka
{कोपं चकार सदृशं निर्दहन्निव राक्षसम्}
{धनुरादाय देवेन्द्रधनुराकारमद्भुतम्} %11-39

\twolineshloka
{गृहीत्वा पाणिना बाणं कालानलसमप्रभम्}
{निर्दहन्निव चक्षुर्भ्यां ददृशे रिपुमन्तिके} %11-40

\twolineshloka
{पराक्रमं दर्शयितुं तेजसा प्रज्वलन्निव}
{प्रचक्रमे कालरूपी सर्वलोकस्य पश्यतः} %11-41

\twolineshloka
{विकृष्य चापं रामस्तु रावणं प्रतिविध्य च}
{हर्षयन् वानरानीकं कालान्तक इवाबभौ} %11-43

\twolineshloka
{क्रुद्धं रामस्य वदनं दृष्ट्वा शत्रुं प्रधावतः}
{तत्रसुः सर्वभूतानि चचाल च वसुन्धरा} %11-43

\twolineshloka
{रामं दृष्ट्वा महारौद्रमुत्पातांश्च सुदारुणान्}
{त्रस्तानि सर्वभूतानि रावणं चाविशद्भयम्} %11-44

\threelineshloka
{विमानस्था सुरगणाः सिद्धगन्धर्वकिन्नराः}
{ददृशुः सुमहायुद्धं लोकसंवर्तकोपमम्}
{ऐन्द्रमस्त्रं समादाय रावणस्य शिरोऽच्छिनत्} %11-45

\twolineshloka
{मूर्धानो रावणस्याथ बहवो रुधिरोक्षिताः}
{गगनात्प्रपतन्ति स्म तालादिव फलानि हि} %11-46

\twolineshloka
{न दिनं न च वै रात्रिर्न सन्ध्यां न दिशोऽपि वा}
{प्रकाशन्ते न तद्रूपं दृश्यते तत्र सङ्गरे} %11-47

\twolineshloka
{ततो रामो बभूवाथ विस्मयाविष्टमानसः}
{शतमेकोत्तरं छिन्नं शिरसां चैकवर्चसाम्} %11-48

\twolineshloka
{न चैव रावणः शान्तो दृश्यते जीवितक्षयात्}
{ततः सर्वास्त्रविद्धीरः कौसल्यानन्दवर्धनः} %11-49

\twolineshloka
{अस्त्रैश्च बहुभिर्युक्तश्चिन्तयामास राघवः}
{यैर्यैर्बाणैर्हता दैत्या महासत्त्वपराक्रमाः} %11-50

\twolineshloka
{त एते निष्फलं याता रावणस्य निपातने}
{इति चिन्ताकुले रामे समीपस्थो विभीषणः} %11-51

\twolineshloka
{उवाच राघवं वाक्यं ब्रह्मदत्तवरो ह्यसौ}
{विच्छिन्ना बाहवोऽप्यस्य विच्छिन्नानि शिरांसि च} %11-52

\twolineshloka
{उत्पत्स्यन्ति पुनः शीघ्रमित्याह भगवानजः}
{नाभिदेशेऽमृतं तस्य कुण्डलाकारसंस्थितम्} %11-53

\twolineshloka
{तच्छोषयानलास्त्रेण तस्य मृत्युस्ततो भवेत्}
{विभीषणवचः श्रुत्वा रामः शीघ्रपराक्रमः} %11-54

\twolineshloka
{पावकास्त्रेण संयोज्य नाभिं विव्याध रक्षसः}
{अनन्तरं च चिच्छेद शिरांसि च महाबलः} %11-55

\twolineshloka
{बाहूनपि च संरब्धो रावणस्य रघूत्तमः}
{ततो घोरां महाशक्तिमादाय दशकन्धरः} %11-56

\twolineshloka
{विभीषणवधार्थाय चिक्षेप क्रोधविह्वलः}
{चिच्छेद राघवो बाणैस्तां शितैर्हेमभूषितैः} %11-57

\twolineshloka
{दशग्रीवशिरश्छेदात्तदा तेजो विनिर्गतम्}
{म्लानरूपो बभूवाथ छिन्नैः शीर्षैर्भयङ्करैः} %11-58

\twolineshloka
{एकेन मुख्यशिरसा बाहुभ्यां रावणो बभौ}
{रावणस्तु पुनः क्रुद्धो नानाशस्त्रास्त्रवृष्टिभिः} %11-59

\twolineshloka
{ववर्ष रामं तं रामस्तथा बाणैर्ववर्ष च}
{ततो युद्धमभूद्\mbox{}घोरं तुमुलं लोमहर्षणम्} %11-60

\twolineshloka
{अथ संस्मारयामास मातली राघवं तदा}
{विसृजास्त्रं वधायास्य ब्राह्मं शीघ्रं रघूत्तम} %11-61

\twolineshloka
{विनाशकालः प्रथितो यः सुरैः सोऽद्य वर्तते}
{उत्तमाङ्गं न चैतस्य छेत्तव्यं राघव त्वया} %11-62

\twolineshloka
{नैव शीर्ष्णि प्रभो वध्यो वध्य एव हि मर्मणि}
{ततः संस्मारितो रामस्तेन वाक्येन मातलेः} %11-63

\twolineshloka
{जग्राह स शरं दीप्तं निःश्वसन्तमिवोरगम्}
{यस्य पार्श्वे तु पवनः फले भास्करपावकौ} %11-64

\twolineshloka
{शरीरमाकाशमयं गौरवे मेरुमन्दरौ}
{पर्वस्वपि च विन्यस्ता लोकपाला महौजसः} %11-65

\twolineshloka
{जाज्वल्यमानं वपुषा भातं भास्करवर्चसा}
{तमुग्रमस्त्रं लोकानां भयनाशनमद्भुतम्} %11-66

\twolineshloka
{अभिमन्त्र्य ततो रामस्तं महेषुं महाभुजः}
{वेदप्रोक्तेन विधिना सन्दधे कार्मुके बली} %11-67

\twolineshloka
{तस्मिन् सन्धीयमाने तु राघवेण शरोत्तमे}
{सर्वभूतानि वित्रेसुश्चचाल च वसुन्धरा} %11-68

\twolineshloka
{स रावणाय सङ्क्रुद्धो भृशमानम्य कार्मुकम्}
{चिक्षेप परमायत्तस्तमस्त्रं मर्मघातिनम्} %11-69

\twolineshloka
{स वज्र इव दुर्धर्षो वज्रपाणिविसर्जितः}
{कृतान्त इव घोरास्यो न्यपतद्रावणोरसि} %11-70

\twolineshloka
{स निमग्नो महाघोरः शरीरान्तकरः परः}
{बिभेद हृदयं तूर्णं रावणस्य महात्मनः} %11-71

\twolineshloka
{रावणस्याहरत्प्राणान् विवेश धरणीतले}
{स शरो रावणं हत्वा रामतूणीरमाविशत्} %11-72

\twolineshloka
{तस्य हस्तात्पपाताशु सशरं कार्मुकं महत्}
{गतासुर्भ्रमिवेगेन राक्षसेन्द्रोऽपतद्भुवि} %11-73

\twolineshloka
{तं दृष्ट्वा पतितं भूमौ हतशेषाश्च राक्षसाः}
{हतनाथा भयत्रस्ता दुद्रुवुः सर्वतोदिशम्} %11-74

\twolineshloka
{दशग्रीवस्य निधनं विजयं राघवस्य च}
{ततो विनेदुः संहृष्टा वानरा जितकाशिनः} %11-75

\twolineshloka
{वदन्तो रामविजयं रावणस्य च तद्वधम्}
{अथान्तरिक्षे व्यनदत्सौम्यस्त्रिदशदुन्दुभिः} %11-76

\twolineshloka
{पपात पुष्पवृष्टिश्च समन्ताद्राघवोपरि}
{तुष्टुवुर्मुनयः सिद्धाश्चारणाश्च दिवौकसः} %11-77

\twolineshloka
{अथान्तरिक्षे ननृतुः सर्वतोऽप्सरसो मुदा}
{रावणस्य च देहोत्थं ज्योतिरादित्यवत्स्फुरत्} %11-78

\twolineshloka
{प्रविवेश रघुश्रेष्ठं देवानां पश्यतां सताम्}
{देवा ऊचुरहो भाग्यं रावणस्य महात्मनः} %11-79

\twolineshloka
{वयं तु सात्त्विका देवा विष्णोः कारुण्यभाजनाः}
{भयदुःखादिभिर्व्याप्ताः संसारे परिवर्तिनः} %11-80

\twolineshloka
{अयं तु राक्षसः क्रूरो ब्रह्महाऽतीव तामसः}
{परदाररतो विष्णुद्वेषी तापसहिंसकः} %11-81

\twolineshloka
{पश्यत्सु सर्वभूतेषु राममेव प्रविष्टवान्}
{एवं ब्रुवत्सु देवेषु नारदः प्राह सुस्मितः} %11-82

\twolineshloka
{शृणुतात्र सुरा यूयं धर्मतत्त्वविचक्षणाः}
{रावणो राघवद्वेषादनिशं हृदि भावयन्} %11-83

\twolineshloka
{भृत्यैः सह सदा रामचरितं द्वेषसंयुतः}
{श्रुत्वा रामात्स्वनिधनं भयात्सर्वत्र राघवम्} %11-84

\twolineshloka
{पश्यन्ननुदिनं स्वप्ने राममेवानुपश्यति}
{क्रोधोऽपि रावणस्याऽऽशु गुरुबोधाधिकोऽभवत्} %11-85

\twolineshloka
{रामेण निहतश्चान्ते निर्धूताशेषकल्मषः}
{रामसायुज्यमेवाऽऽप रावणो मुक्तबन्धनः} %11-86

\fourlineindentedshloka
{पापिष्ठो वा दुरात्मा परधनपरदारेषु सक्तो यदि स्यान्-}
{नित्यं स्नेहाद्भयाद्वा रघुकुलतिलकं भावयन् सम्परेतः}
{भूत्वा शुद्धान्तरङ्गो भवशतजनितानेकदोषैर्विमुक्तः}
{सद्यो रामस्य विष्णोः सुरवरविनुतं याति वैकुण्ठमाद्यम्} %11-87

\fourlineindentedshloka
{हत्वा युद्धे दशास्यं त्रिभुवनविषमं वामहस्तेन चापम्}
{भुमौ विष्टभ्य तिष्ठन्नितरकरधृतं भ्रामयन् बाणमेकम्}
{आरक्तोपान्तनेत्रः शरदलितवपुः सूर्यकोटिप्रकाशो}
{वीरश्रीबन्धुराङ्गस्त्रिदशपतिनुतः पातु मां वीररामः} %11-88

\iti{युद्धकाण्डे}{एकादशः}
%%%%%%%%%%%%%%%%%%%%



\sect{द्वादशः सर्गः}

\uvacha{श्रीमहादेव उवाच}

\twolineshloka
{रामो विभीषणं दृष्ट्वा हनूमन्तं तथाऽङ्गदम्}
{लक्ष्मणं कपिराजं च जाम्बवन्तं तथा परान्} %12-1

\twolineshloka
{परितुष्टेन मनसा सर्वानेवाब्रवीद्वचः}
{भवतां बाहुवीर्येण निहतो रावणो मया} %12-2

\twolineshloka
{कीर्तिः स्थास्यति वः पुण्या यावच्चन्द्रदिवाकरौ}
{कीर्तयिष्यन्ति भवतां कथां त्रैलोक्यपावनीम्} %12-3

\twolineshloka
{मयोपेतां कलिहरां यास्यन्ति परमां गतिम्}
{एतस्मिन्नन्तरे दृष्ट्वा रावणं पतितं भुवि} %12-4

\twolineshloka
{मन्दोदरीमुखाः सर्वाः स्त्रियो रावणपालिताः}
{पतिता रावणस्याग्रे शोचन्त्यः पर्यदेवयन्} %12-5

\twolineshloka
{विभीषणः शुशोचार्तः शोकेन महताऽऽवृतः}
{पतितो रावणस्याग्रे बहुधा पर्यदेवयत्} %12-6

\twolineshloka
{रामस्तु लक्ष्मणं प्राह बोधयस्व विभीषणम्}
{करोतु भ्रातृसंस्कारं किं विलम्बेन मानद} %12-7

\twolineshloka
{स्त्रियो मन्दोदरीमुख्याः पतिता विलपन्ति च}
{निवारयतु ताः सर्वा राक्षसी रावणप्रियाः} %12-8

\twolineshloka
{एवमुक्तोऽथ रामेण लक्ष्मणोऽगाद्विभीषणम्}
{उवाच मृतकोपान्ते पतितं मृतकोपमम्} %12-9

\twolineshloka
{शोकेन महताऽऽविष्टं सौमित्रिरिदमब्रवीत्}
{यं शोचसि त्वं दुःखेन कोऽयं तव विभीषण} %12-10

\twolineshloka
{त्वं वास्य कतमः सृष्टेः पुरेदानीमतः परम्}
{यद्वत्तोयौघपतिताः सिकता यान्ति तद्वशाः} %12-11

\twolineshloka
{संयुज्यन्ते वियुज्यन्ते तथा कालेन देहिनः}
{यथा धानासु वै धाना भवन्ति न भवन्ति च} %12-12

\twolineshloka
{एवं भूतेषु भूतानि प्रेरितानीशमायया}
{त्वं चेमे वयमन्ये च तुल्याः कालवशोद्भवाः} %12-13

\twolineshloka
{जन्ममृत्यू यदा यस्मात्तदा तस्माद्भविष्यतः}
{ईश्वरः सर्वभूतानि भूतैः सृजति हन्त्यजः} %12-14

\twolineshloka
{आत्मसृष्टैरस्वतन्त्रैर्निरपेक्षोऽपि बालवत्}
{देहेन देहिनो जीवा देहाद्देहोऽभिजायते} %12-15

\twolineshloka
{बीजादेव यथा बीजं देहान्य इव शाश्वतः}
{देहिदेहविभागोऽयमविवेककृतः पुरा} %12-16

\twolineshloka
{नानात्वं जन्म नाशश्च क्षयो वृद्धिः क्रियाफलम्}
{द्रष्टुराभान्त्यतद्धर्मा यथाग्नेर्दारुविक्रियाः} %12-17

\twolineshloka
{त इमे देहसंयोगादात्मना भान्त्यसद्ग्रहात्}
{यथा यथा तथा चान्यद्ध्यायतोऽसत्सदाग्रहात्} %12-18

\twolineshloka
{प्रसुप्तस्यानहम्भावात्तदा भाति न संसृतिः}
{जीवतोऽपि तथा तद्वद्विमुक्तस्यानहङ्कृतेः} %12-19

\twolineshloka
{तस्मान्मायामनोधर्मं जह्यहम्ममताभ्रमम्}
{रामभद्रे भगवति मनो धेह्यात्मनीश्वरे} %12-20

\twolineshloka
{सर्वभूतात्मनि परे मायामानुषरूपिणि}
{बाह्येन्द्रियार्थसम्बन्धात्त्याजयित्वा मनः शनैः} %12-21

\twolineshloka
{तत्र दोषान् दर्शयित्वा रामानन्दे नियोजय}
{देहबुद्ध्या भवेद्भ्राता पिता माता सुहृत्प्रियः} %12-22

\twolineshloka
{विलक्षणं यदा देहाज्जानात्यात्मानमात्मना}
{तदा कः कस्य वा बन्धुर्भ्राता माता पिता सुहृत्} %12-23

\twolineshloka
{मिथ्याज्ञानवशाज्जाता दारागारादयः सदा}
{शब्दादयश्च विषया विविधाश्चैव सम्पदः} %12-24

\twolineshloka
{बलं कोशो भृत्यवर्गो राज्यं भूमिः सुतादयः}
{अज्ञानजत्वात्सर्वे ते क्षणसङ्गमभङ्गुराः} %12-25

\twolineshloka
{अथोत्तिष्ठ हृदा रामं भावयन् भक्तिभावितम्}
{अनुवर्तस्व राज्यादि भुञ्जन् प्रारब्धमन्वहम्} %12-26

\twolineshloka
{भूतं भविष्यदभजन् वर्तमानमथाचरन्}
{विहरस्व यथान्यायं भवदोषैर्न लिप्यसे} %12-27

\twolineshloka
{आज्ञापयति रामस्त्वां यद्भ्रातुः साम्परायिकम्}
{तत्कुरुष्व यथाशास्त्रं रुदतीश्चापि योषितः} %12-28

\twolineshloka
{निवारय महाबुद्धे लङ्कां गच्छन्तु मा चिरम्}
{श्रुत्वा यथावद्वचनं लक्ष्मणस्य विभीषणः} %12-29

\twolineshloka
{त्यक्त्वा शोकं च मोहं च रामपार्श्वमुपागमत्}
{विमृश्य बुद्ध्या धर्मज्ञो धर्मार्थसहितं वचः} %12-30

\twolineshloka
{रामस्यैवानुवृत्त्यर्थमुत्तरं पर्यभाषत}
{नृशंसमनृतं क्रूरं त्यक्तधर्मव्रतं प्रभो} %12-31

\twolineshloka
{नार्होऽस्मि देव संस्कर्तुं परदाराभिमर्शिनम्}
{श्रुत्वा तद्वचनं प्रीतो रामो वचनमब्रवीत्} %12-32

\twolineshloka
{मरणान्तानि वैराणि निवृत्तं नः प्रयोजनम्}
{क्रियतामस्य संस्कारो ममाप्येष यथा तव} %12-33

\twolineshloka
{रामाज्ञां शिरसा धृत्वा शीघ्रमेव विभीषणः}
{सान्त्ववाक्यैर्महाबुद्धिं राज्ञीं मन्दोदरीं तदा} %12-34

\twolineshloka
{सान्त्वयामास धर्मात्मा धर्मबुद्धिर्विभीषणः}
{त्वरयामास धर्मज्ञः संस्कारार्थं स्वबान्धवान्} %12-35

\twolineshloka
{चित्यां निवेश्य विधिवत्पितृमेधविधानतः}
{आहिताग्नेर्यथा कार्यं रावणस्य विभीषणः} %12-36

\twolineshloka
{तथैव सर्वमकरोद्बन्धुभिः सह मन्त्रिभिः}
{ददौ च पावकं तस्य विधियुक्तं विभीषणः} %12-37

\twolineshloka
{स्नात्वा चैवार्द्रवस्त्रेण तिलान् दर्भाभिमिश्रितान्}
{उदकेन च सम्मिश्रान् प्रदाय विधिपूर्वकम्} %12-38

\twolineshloka
{प्रदाय चोदकं तस्मै मूर्ध्ना चैनं प्रणम्य च}
{ताः स्त्रियोऽनुनयामास सान्त्वमुक्त्वा पुनः पुनः} %12-39

\twolineshloka
{गम्यतामिति ताः सर्वा विविशुर्नगरं तदा}
{प्रविष्टासु च सर्वासु राक्षसीषु विभीषणः} %12-40

\twolineshloka
{रामपार्श्वमुपागत्य तदाऽतिष्ठद्विनीतवत्}
{रामोऽपि सह सैन्येन ससुग्रीवः सलक्ष्मणः} %12-41

\twolineshloka
{हर्षं लेभे रिपून् हत्वा यथा वृत्रं शतक्रतुः}
{मातलिश्च तदा रामं परिक्रम्याभिवन्द्य च} %12-42

\twolineshloka
{अनुज्ञातश्च रामेण ययौ स्वर्गं विहायसा}
{ततो हृष्टमना रामो लक्ष्मणं चेदमब्रवीत्} %12-43

\twolineshloka
{विभीषणाय मे लङ्काराज्यं दत्तं पुरैव हि}
{इदानीमपि गत्वा त्वं लङ्कामध्ये विभीषणम्} %12-44

\twolineshloka
{अभिषेचय विप्रैश्च मन्त्रवद्विधिपूर्वकम्}
{इत्युक्तो लक्ष्मणस्तूर्णं जगाम सह वानरैः} %12-45

\twolineshloka
{लङ्कां सुवर्णकलशैः समुद्रजलसंयुतैः}
{अभिषेकं शुभं चक्रे राक्षसेन्द्रस्य धीमतः} %12-46

\twolineshloka
{ततः पौरजनैः सार्धं नानोपायनपाणिभिः}
{विभीषणः ससौमित्रिरुपायनपुरस्कृतः} %12-47

\twolineshloka
{दण्डप्रणाममकरोद्रामस्याक्लिष्टकर्मणः}
{रामो विभीषणं दृष्ट्वा प्राप्तराज्यं मुदान्वितः} %12-48

\twolineshloka
{कृतकृत्यमिवात्मानममन्यत सहानुजः}
{सुग्रीवं च समालिङ्ग्य रामो वाक्यमथाब्रवीत्} %12-49

\twolineshloka
{सहायेन त्वया वीर जितो मे रावणो महान्}
{विभीषणोऽपि लङ्कायामभिषिक्तो मयाऽनघ} %12-50

\twolineshloka
{ततः प्राह हनूमन्तं पार्श्वस्थं विनयान्वितम्}
{विभीषणस्यानुमतेर्गच्छ त्वं रावणालयम्} %12-51

\twolineshloka
{जानक्यै सर्वमाख्याहि रावणस्य वधादिकम्}
{जानक्याः प्रतिवाक्यं मे शीघ्रमेव निवेदय} %12-52

\twolineshloka
{एवमाज्ञापितो धीमान् रामेण पवनात्मजः}
{प्रविवेश पुरीं लङ्कां पूज्यमानो निशाचरैः} %12-53

\twolineshloka
{प्रविश्य रावणगृहं शिंशपामूलमाश्रिताम्}
{ददर्श जानकीं तत्र कृशां दीनामनिन्दिताम्} %12-54

\twolineshloka
{राक्षसीभिः परिवृतां ध्यायन्तीं राममेव हि}
{विनयावनतो भूत्वा प्रणम्य पवनात्मजः} %12-55

\twolineshloka
{कृताञ्जलिपुटो भूत्वा प्रह्वो भक्त्याऽग्रतः स्थितः}
{तं दृष्ट्वा जानकी तूष्णीं स्थित्वा पूर्वस्मृतिं ययौ} %12-56

\threelineshloka
{ज्ञात्वा तं रामदूतं सा हर्षात्सौम्यमुखी बभौ}
{स तां सौम्यमुखीं दृष्ट्वा तस्यै पवननन्दनः}
{रामस्य भाषितं सर्वमाख्यातुमुपचक्रमे} %12-57

\twolineshloka
{देवि रामः ससुग्रीवो विभीषणसहायवान्}
{कुशली वानराणां च सैन्यैश्च सहलक्ष्मणः} %12-58

\twolineshloka
{रावणं ससुतं हत्वा सबलं सह मन्त्रिभिः}
{त्वामाह कुशलं रामो राज्ये कृत्वा विभीषणम्} %12-59

\twolineshloka
{श्रुत्वा भर्तुः प्रियं वाक्यं हर्षगद्गदया गिरा}
{किं ते प्रियं करोम्यद्य न पश्यामि जगत्त्रये} %12-60

\twolineshloka
{समं ते प्रियवाक्यस्य रत्नान्याभरणानि च}
{एवमुक्तस्तु वैदेह्या प्रत्युवाच प्लवङ्गमः} %12-61

\twolineshloka
{रत्नौघाद्विविधाद्वाऽपि देवराज्याद्विशिष्यते}
{हतशत्रुं विजयिनं रामं पश्यामि सुस्थिरम्} %12-62

\twolineshloka
{तस्य तद्वचनं श्रुत्वा मैथिली प्राह मारुतिम्}
{सर्वे सौम्या गुणा सौम्य त्वय्येव परिनिष्ठिताः} %12-63

\twolineshloka
{रामं द्रक्ष्यामि शीघ्रं मामाज्ञापयतु राघवः}
{तथेति तां नमस्कृत्य ययौ द्रष्टुं रघूत्तमम्} %12-64

\twolineshloka
{जानक्या भाषितं सर्वं रामस्याग्रे न्यवेदयत्}
{यन्निमित्तोऽयमारम्भः कर्मणां च फलोदयः} %12-65

\twolineshloka
{तां देवीं शोकसन्तप्तां द्रष्टुमर्हसि मैथिलीम्}
{एवमुक्तो हनुमता रामो ज्ञानवतां वरः} %12-66

\twolineshloka
{मायासीतां परित्यक्तुं जानकीमनले स्थिताम्}
{आदातुं मनसा ध्यात्वा रामः प्राह विभीषणम्} %12-67

\twolineshloka
{गच्छ राजन् जनकजामानयाऽऽशु ममान्तिकम्}
{स्नातां विरजवस्त्राढ्यां सर्वाभरणभूषिताम्} %12-68

\twolineshloka
{विभीषणोऽपि तच्छ्रुत्वा जगाम सहमारुतिः}
{राक्षसीभिः सुवृद्धाभिः स्नापयित्वा तु मैथिलीम्} %12-69

\twolineshloka
{सर्वाभरणसम्पन्नामारोप्य शिबिकोत्तमे}
{याष्टीकैर्बहुभिर्गुप्तां कञ्चुकोष्णीषिभिः शुभाम्} %12-70

\twolineshloka
{तां द्रष्टुमागताः सर्वे वानरा जनकात्मजाम्}
{तान् वारयन्तो बहवः सर्वतो वेत्रपाणयः} %12-71

\twolineshloka
{कोलाहलं प्रकुर्वन्तो रामपार्श्वमुपाययुः}
{दृष्ट्वा तां शिबिकारूढां दूरादथ रघूत्तमः} %12-72

\twolineshloka
{विभीषण किमर्थं ते वानरान् वारयन्ति हि}
{पश्यन्तु वानराः सर्वे मैथिलीं मातरं यथा} %12-73

\twolineshloka
{पादचारेण साऽऽयातु जानकी मम सन्निधिम्}
{श्रुत्वा तद्रामवचनं शिबिकादवरुह्य सा} %12-74

\twolineshloka
{पादचारेण शनकैरागता रामसन्निधिम्}
{रामोऽपि दृष्ट्वा तां मायासीतां कार्यार्थनिर्मिताम्} %12-75

\twolineshloka
{अवाच्यवादान् बहुशः प्राह तां रघुनन्दनः}
{अमृष्यमाणा सा सीता वचनं राघवोदितम्} %12-76

\twolineshloka
{लक्ष्मणं प्राह मे शीघ्रं प्रज्वालय हुताशनम्}
{विश्वासार्थं हि रामस्य लोकानां प्रत्ययाय च} %12-77

\twolineshloka
{राघवस्य मतं ज्ञात्वा लक्ष्मणोऽपि तदैव हि}
{महाकाष्ठचयं कृत्वा ज्वालयित्वा हुताशनम्} %12-78

{रामपार्श्वमुपागम्य तस्थौ तूष्णीमरिन्दमः} %12-83

\twolineshloka
{ततः सीता परिक्रम्य राघवं भक्तिसंयुता} %12-79
{पश्यतां सर्वलोकानां देवराक्षसयोषिताम्}

\twolineshloka
{प्रणम्य देवताभ्यश्च ब्राह्मणेभ्यश्च मैथिली} %12-80
{बद्धाञ्जलिपुटा चेदमुवाचाग्निसमीपगा}

\twolineshloka
{यथा मे हृदयं नित्यं नापसर्पति राघवात्} %12-81
{तथा लोकस्य साक्षी मां सर्वतः पातु पावकः}

\twolineshloka
{एवमुक्त्वा तदा सीता परिक्रम्य हुताशनम्} %12-82
{विवेश ज्वलनं दीप्तं निर्भयेन हृदा सती} %12-83


\fourlineindentedshloka
{दृष्ट्वा ततो भूतगणाः ससिद्धाः}
{सीतां महावह्निगतां भृशार्ताः}
{परस्परं प्राहुरहो स सीताम्}
{रामः श्रियं स्वां कथमत्यजज्ज्ञः} %12-84

\iti{युद्धकाण्डे}{द्वादशः}
%%%%%%%%%%%%%%%%%%%%



\sect{त्रयोदशः सर्गः}

\uvacha{श्रीमहादेव उवाच}

\twolineshloka
{ततः शक्रः सहस्राक्षो यमश्च वरुणस्तथा}
{कुबेरश्च महातेजाः पिनाकी वृषवाहनः} %13-1

\twolineshloka
{ब्रह्मा ब्रह्मविदां श्रेष्ठो मुनिभिः सिद्धचारणैः}
{ऋषयः पितरः साध्या गन्धर्वाप्सरसोरगाः} %13-2

\twolineshloka
{एते चान्ये विमानाग्र्यैराजग्मुर्यत्र राघवः}
{अब्रुवन् परमात्मानं रामं प्राञ्जलयश्च ते} %13-3

\twolineshloka
{कर्ता त्वं सर्वलोकानां साक्षी विज्ञानविग्रहः}
{वसूनामष्टमोऽसि त्वं रुद्राणां शङ्करो भवान्} %13-4

\twolineshloka
{आदिकर्ताऽसि लोकानां ब्रह्मा त्वं चतुराननः}
{अश्विनौ घ्राणभूतौ ते चक्षुषी चन्द्रभास्करौ} %13-5

\twolineshloka
{लोकानामादिरन्तोऽसि नित्य एकः सदोदितः}
{सदा शुद्धः सदा बुद्धः सदा मुक्तोऽगुणोऽद्वयः} %13-6

\twolineshloka
{त्वन्मायासंवृतानां त्वं भासि मानुषविग्रहः}
{त्वन्नाम स्मरतां राम सदा भासि चिदात्मकः} %13-7

\twolineshloka
{रावणेन हृतं स्थानमस्माकं तेजसा सह}
{त्वयाऽद्य निहतो दुष्टः पुनः प्राप्तं पदं स्वकम्} %13-8

\twolineshloka
{एवं स्तुवत्सु देवेषु ब्रह्मा साक्षात्पितामहः}
{अब्रवीत्प्रणतो भूत्वा रामं सत्यपथे स्थितम्} %13-9


\uvacha{ब्रह्मोवाच}

\fourlineindentedshloka
{वन्दे देवं विष्णुमशेषस्थितिहेतुम्}
{त्वामध्यात्मज्ञानिभिरन्तर्हृदि भाव्यम्}
{हेयाहेयद्वन्द्वविहीनं परमेकम्}
{सत्तामात्रं सर्वहृदिस्थं दृशिरूपम्} %13-10

\fourlineindentedshloka
{प्राणापानौ निश्चयबुद्ध्या हृदि रुद्ध्वा}
{छित्वा सर्वं संशयबन्धं विषयौघान्}
{पश्यन्तीशं यं गतमोहा यतयस्तम्}
{वन्दे रामं रत्नकिरीटं रविभासम्} %13-11

\fourlineindentedshloka
{मायातीतं माधवमाद्यं जगदादिम्}
{मानातीतं मोहविनाशं मुनिवन्द्यम्}
{योगिध्येयं योगविधानं परिपूर्णम्}
{वन्दे रामं रञ्जितलोकं रमणीयम्} %13-12

\fourlineindentedshloka
{भावाभावप्रत्ययहीनं भवमुख्यैः}
{योगासक्तैरर्चितपादाम्बुजयुग्मम्}
{नित्यं शुद्धं बुद्धमनन्तं प्रणवाख्यम्}
{वन्दे रामं वीरमशेषासुरदावम्} %13-13

\fourlineindentedshloka
{त्वं मे नाथो नाथितकार्याखिलकारी}
{मानातीतो माधवरूपोऽखिलधारी}
{भक्त्या गम्यो भावितरूपो भवहारी}
{योगाभ्यासैर्भावितचेतःसहचारी} %13-14

\fourlineindentedshloka
{त्वामाद्यन्तं लोकततीनां परमीशम्}
{लोकानां नो लौकिकमानैरधिगम्यम्}
{भक्तिश्रद्धाभावसमेतैर्भजनीयम्}
{वन्दे रामं सुन्दरमिन्दीवरनीलम्} %13-15

\fourlineindentedshloka
{को वा ज्ञातुं त्वामतिमानं गतमानम्}
{मायासक्तो माधव शक्तो मुनिमान्यम्}
{वृन्दारण्ये वन्दितवृन्दारकवृन्दम्}
{वन्दे रामं भवमुखवन्द्यं सुखकन्दम्} %13-16

\fourlineindentedshloka
{नानाशास्त्रैर्वेदकदम्बैः प्रतिपाद्यम्}
{नित्यानन्दं निर्विषयज्ञानमनादिम्}
{मत्सेवार्थं मानुषभावं प्रतिपन्नम्}
{वन्दे रामं मरकतवर्णं मथुरेशम्} %13-17

\fourlineindentedshloka
{श्रद्धायुक्तो यः पठतीमं स्तवमाद्यम्}
{ब्राह्मं ब्रह्मज्ञानविधानं भुवि मर्त्यः}
{रामं श्यामं कामितकामप्रदमीशम्}
{ध्यात्वा ध्याता पातकजालैर्विगतः स्यात्} %13-18

\fourlineindentedshloka
{श्रुत्वा स्तुतिं लोकगुरोर्विभावसुः}
{स्वाङ्के समादाय विदेहपुत्रिकाम्}
{विभ्राजमानां विमलारुणद्युतिम्}
{रक्ताम्बरां दिव्यविभूषणान्विताम्} %13-19

\fourlineindentedshloka
{प्रोवाच साक्षी जगतां रघूत्तमम्}
{प्रपन्नसर्वार्तिहरं हुताशनः}
{गृहाण देवीं रघुनाथ जानकीम्}
{पुरा त्वया मय्यवरोपितां वने} %13-20

\fourlineindentedshloka
{विधाय मायाजनकात्मजां हरे}
{दशाननप्राणविनाशनाय च}
{हतो दशास्यः सह पुत्रबान्धवैः}
{निराकृतोऽनेन भरो भुवः प्रभो} %13-21

\fourlineindentedshloka
{तिरोहिता सा प्रतिबिम्बरूपिणी}
{कृता यदर्थं कृतकृत्यतां गता}
{ततोऽतिहृष्टां परिगृह्य जानकीम्}
{रामः प्रहृष्टः प्रतिपूज्य पावकम्} %13-22

\begin{minipage}{\linewidth}
\centering%major manual alignment to make it look good!
{\hspace{-7ex}स्वाङ्के समावेश्य सदाऽनपायिनीम्}\\
{\hspace{-1ex}श्रियं त्रिलोकीजननीं श्रियः पतिः।}
\fourlineindentedshloka
{दृष्ट्वाऽथ\hspace{1.3ex}रामं\hspace{1.3ex}जनकात्मजायुतम्}
{श्रिया स्फुरन्तं सुरनायको मुदा}
{भक्त्या गिरा गद्गदया समेत्य}
{कृताञ्जलिः \hspace{1.5ex}स्तोतुमथोपचक्रमे} %13-23
\end{minipage}

\begin{minipage}{\linewidth}
\uvacha{इन्द्र उवाच}

\fourlineindentedshloka
{भजेऽहं सदा राममिन्दीवराभम्}
{भवारण्यदावानलाभाभिधानम्}
{भवानीहृदा भावितानन्दरूपम्}
{भवाभावहेतुं भवादिप्रपन्नम्} %13-24
\end{minipage}

\fourlineindentedshloka
{सुरानीकदुःखौघनाशैकहेतुम्}
{नराकारदेहं निराकारमीड्यम्}
{परेशं परानन्दरूपं वरेण्यम्}
{हरिं राममीशं भजे भारनाशम्} %13-25

\fourlineindentedshloka
{प्रपन्नाखिलानन्ददोहं प्रपन्नम्}
{प्रपन्नार्तिनिःशेषनाशाभिधानम्}
{तपोयोगयोगीशभावाभिभाव्यम्}
{कपीशादिमित्रं भजे राममित्रम्} %13-26

\fourlineindentedshloka
{सदा भोगभाजां सुदूरे विभान्तम्}
{सदा योगभाजामदूरे विभान्तम्}
{चिदानन्दकन्दं सदा राघवेशम्}
{विदेहात्मजानन्दरूपं प्रपद्ये} %13-27

\fourlineindentedshloka
{महायोगमायाविशेषानुयुक्तो}
{विभासीश लीलानराकारवृत्तिः}
{त्वदानन्दलीलाकथापूर्णकर्णाः}
{सदानन्दरूपा भवन्तीह लोके} %13-28

\fourlineindentedshloka
{अहं मानपानाभिमत्तप्रमत्तो}
{न वेदाखिलेशाभिमानाभिमानः}
{इदानीं भवत्पादपद्मप्रसादात्}
{त्रिलोकाधिपत्याभिमानो विनष्टः} %13-29

\fourlineindentedshloka
{स्फुरद्रत्नकेयूरहाराभिरामम्}
{धराभारभूतासुरानीकदावम्}
{शरच्चन्द्रवक्त्रं लसत्पद्मनेत्रम्}
{दुरावारपारं भजे राघवेशम्} %13-30

\fourlineindentedshloka
{सुराधीशनीलाभ्रनीलाङ्गकान्तिम्}
{विराधादिरक्षोवधाल्लोकशान्तिम्}
{किरीटादिशोभं पुरारातिलाभम्}
{भजे रामचन्द्रं रघूणामधीशम्} %13-31

\fourlineindentedshloka
{लसच्चन्द्रकोटिप्रकाशादिपीठे}
{समासीनमङ्के समाधाय सीताम्}
{स्फुरद्धेमवर्णां तडित्पुञ्जभासाम्}
{भजे रामचन्द्रं निवृत्तार्तितन्द्रम्} %13-32

\twolineshloka
{ततः प्रोवाच भगवान् भवान्या सहितो भवः}
{रामं कमलपत्राक्षं विमानस्थो नभःस्थले} %13-33

\twolineshloka
{आगमिष्याम्ययोध्यायां द्रष्टुं त्वां राज्यसत्कृतम्}
{इदानीं पश्य पितरमस्य देहस्य राघव} %13-34

\twolineshloka
{ततोऽपश्यद्विमानस्थं रामो दशरथं पुरः}
{ननाम शिरसा पादौ मुदा भक्त्या सहानुजः} %13-35

\twolineshloka
{आलिङ्ग्य मूर्ध्न्यवघ्राय रामं दशरथोऽब्रवीत्}
{तारितोऽस्मि त्वया वत्स संसाराद्दुःखसागरात्} %13-36

\twolineshloka
{इत्युक्त्वा पुनरालिङ्ग्य ययौ रामेण पूजितः}
{रामोऽपि देवराजं तं दृष्ट्वा प्राह कृताञ्जलिम्} %13-37

\twolineshloka
{मत्कृते निहतान् सङ्ख्ये वानरान् पतितान् भुवि}
{जीवयाऽऽशु सुधावृष्ट्या सहस्राक्ष ममाऽऽज्ञया} %13-38

\threelineshloka
{तथेत्यमृतवृष्ट्या तान् जीवयामास वानरान्}
{ये ये मृता मृधे पूर्वं ते ते सुप्तोत्थिता इव}
{पूर्ववद्बलिनो हृष्टा रामपार्श्वमुपाययुः} %13-39

\twolineshloka
{नोत्थिता राक्षसास्तत्र पीयूषस्पर्शनादपि}
{विभीषणस्तु साष्टाङ्गं प्रणिपत्याब्रवीद्वचः} %13-40

\twolineshloka
{देव मामनुगृह्णीष्व मयि भक्तिर्यदा तव}
{मङ्गलस्नानमद्य त्वं कुरु सीतासमन्वितः} %13-41

\twolineshloka
{अलङ्कृत्य सह भ्रात्रा श्वो गमिष्यामहे वयम्}
{विभीषणवचः श्रुत्वा प्रत्युवाच रघूत्तमः} %13-42

\twolineshloka
{सुकुमारोऽतिभक्तो मे भरतो मामवेक्षते}
{जटावल्कलधारी स शब्दब्रह्मसमाहितः} %13-43

\twolineshloka
{कथं तेन विना स्नानमलङ्कारादिकं मम}
{अतः सुग्रीवमुख्यांस्त्वं पूजयाऽऽशु विशेषतः} %13-44

\twolineshloka
{पूजितेषु कपीन्द्रेषु पूजितोऽहं न संशयः}
{इत्युक्तो राघवेणाशु स्वर्णरत्नाम्बराणि च} %13-45

\twolineshloka
{ववर्ष राक्षसश्रेष्ठो यथाकामं यथारुचि}
{ततस्तान् पूजितान् दृष्ट्वा रामो रत्नैश्च यूथपान्} %13-46

\twolineshloka
{अभिनन्द्य यथान्यायं विससर्ज हरीश्वरान्}
{विभीषणसमानीतं पुष्पकं सूर्यवर्चसम्} %13-47

\twolineshloka
{आरुरोह ततो रामस्तद्विमानमनुत्तमम्}
{अङ्के निधाय वैदेहीं लज्जमानां यशस्विनीम्} %13-48

\twolineshloka
{लक्ष्मणेन सह भ्रात्रा विक्रान्तेन धनुष्मता}
{अब्रवीच्च विमानस्थः श्रीरामः सर्ववानरान्} %13-49

\twolineshloka
{सुग्रीवं हरिराजं च अङ्गदं च विभीषणम्}
{मित्रकार्यं कृतं सर्वं भवद्भिः सह वानरैः} %13-50

\twolineshloka
{अनुज्ञाता मया सर्वे यथेष्टं गन्तुमर्हथ}
{सुग्रीव प्रतियाह्याशु किष्किन्धां सर्वसैनिकैः} %13-51

\twolineshloka
{स्वराज्ये वस लङ्कायां मम भक्तो विभीषण}
{न त्वां धर्षयितुं शक्ताः सेन्द्रा अपि दिवौकसः} %13-52

\twolineshloka
{अयोध्यां गन्तुमिच्छामि राजधानीं पितुर्मम}
{एवमुक्तास्तु रामेण वानरास्ते महाबलाः} %13-53

\twolineshloka
{ऊचुः प्राञ्जलयः सर्वे राक्षसश्च विभीषणः}
{अयोध्यां गन्तुमिच्छामस्त्वया सह रघूत्तम} %13-54

\twolineshloka
{दृष्ट्वा त्वामभिषिक्तं तु कौसल्यामभिवाद्य च}
{पश्चाद्वृणीमहे राज्यमनुज्ञां देहि नः प्रभो} %13-55

\twolineshloka
{रामस्तथेति सुग्रीव वानरैः सविभीषणः}
{पुष्पकं सहनूमांश्च शीघ्रमारोह साम्प्रतम्} %13-56

\twolineshloka
{ततस्तु पुष्पकं दिव्यं सुग्रीवः सह सेनया}
{विभीषणश्च सामात्यः सर्वे चारुरुहुर्द्रुतम्} %13-57

\twolineshloka
{तेष्वारूढेषु सर्वेषु कौबेरं परमासनम्}
{राघवेणाभ्यनुज्ञातमुत्पपात विहायसा} %13-58

\twolineshloka
{बभौ तेन विमानेन हंसयुक्तेन भास्वता}
{प्रहृष्टश्च तदा रामश्चतुर्मुख इवापरः} %13-59

\fourlineindentedshloka
{ततो बभौ भास्करबिम्बतुल्यम्}
{कुबेरयानं तपसानुलब्धम्}
{रामेण शोभां नितरां प्रपेदे}
{सीतासमेतेन सहानुजेन} %13-60

\iti{युद्धकाण्डे}{त्रयोदशः}
%%%%%%%%%%%%%%%%%%%%



\sect{चतुर्दशः सर्गः}

\uvacha{श्रीमहादेव उवाच}

\twolineshloka
{पातयित्वा ततश्चक्षुः सर्वतो रघुनन्दनः}
{अब्रवीन्मैथिलीं सीतां रामः शशिनिभाननाम्} %14-1

\twolineshloka
{त्रिकूटशिखराग्रस्थां पश्य लङ्कां महाप्रभाम्}
{एतां रणभुवं पश्य मांसकर्दमपङ्किलाम्} %14-2

\twolineshloka
{असुराणां प्लवङ्गानामत्र वैशसनं महत्}
{अत्र मे निहतः शेते रावणो राक्षसेश्वरः} %14-3

\twolineshloka
{कुम्भकर्णेन्द्रजिन्मुख्याः सर्वे चात्र निपातिताः}
{एष सेतुर्मया बद्धः सागरे सलिलाशये} %14-4

\twolineshloka
{एतच्च दृश्यते तीर्थं सागरस्य महात्मनः}
{सेतुबन्धमिति ख्यातं त्रैलोक्येन च पूजितम्} %14-5

\twolineshloka
{एतत्पवित्रं परमं दर्शनात्पातकापहम्}
{अत्र रामेश्वरो देवो मया शम्भुः प्रतिष्ठितः} %14-6

\twolineshloka
{अत्र मां शरणं प्राप्तो मन्त्रिभिश्च विभीषणः}
{एषा सुग्रीवनगरी किष्किन्धा चित्रकानना} %14-7

\twolineshloka
{तत्र रामाज्ञया ताराप्रमुखा हरियोषितः}
{आनयामास सुग्रीवः सीतायाः प्रियकाम्यया} %14-8

\twolineshloka
{ताभिः सहोत्थितं शीघ्रं विमानं प्रेक्ष्य राघवः}
{प्राह चाद्रिमृष्यमूकं पश्य वाल्यत्र मे हतः} %14-9

\twolineshloka
{एषा पञ्चवटी नाम राक्षसा यत्र मे हताः}
{अगस्त्यस्य सुतीक्ष्णस्य पश्याश्रमपदे शुभे} %14-10

\twolineshloka
{एते ते तापसाः सर्वे दृश्यन्ते वरवर्णिनि}
{असौ शैलवरो देवि चित्रकूटः प्रकाशते} %14-11

\twolineshloka
{अत्र मां कैकयीपुत्रः प्रसादयितुमागतः}
{भरद्वाजाश्रमं पश्य दृश्यते यमुनातटे} %14-12

\twolineshloka
{एषा भागीरथी गङ्गा दृश्यते लोकपावनी}
{एषा सा दृश्यते सीते सरयूयूपमालिनी} %14-13

\twolineshloka
{एषा सा दृश्यतेऽयोध्या प्रणामं कुरु भामिनि}
{एवं क्रमेण सम्प्राप्तो भरद्वाजाश्रमं हरिः} %14-14

\twolineshloka
{पूर्णे चतुर्दशे वर्षे पञ्चम्यां रघुनन्दनः}
{भरद्वाजं मुनिं दृष्ट्वा ववन्दे सानुजः प्रभुः} %14-15

\twolineshloka
{पप्रच्छ मुनिमासीनं विनयेन रघूत्तमः}
{शृणोषि कच्चिद्भरतः कुशल्यास्ते सहानुजः} %14-16

\twolineshloka
{सुभिक्षा वर्ततेऽयोध्या जीवन्ति च हि मातरः}
{श्रुत्वा रामस्य वचनं भरद्वाजः प्रहृष्टधीः} %14-17

\twolineshloka
{प्राह सर्वे कुशलिनो भरतस्तु महामनाः}
{फलमूलकृताहारो जटावल्कलधारकः} %14-18

\twolineshloka
{पादुके सकलं न्यस्य राज्यं त्वां सुप्रतीक्षते}
{यद्यत्कृतं त्वया कर्म दण्डके रघुनन्दन} %14-19

\twolineshloka
{राक्षसानां विनाशं च सीताहरणपूर्वकम्}
{सर्वं ज्ञातं मया राम तपसा ते प्रसादतः} %14-20

\twolineshloka
{त्वं ब्रह्म परमं साक्षादादिमध्यान्तवर्जितः}
{त्वमग्रे सलिलं सृष्ट्वा तत्र सुप्तोऽसि भूतकृत्} %14-21

\twolineshloka
{नारायणोऽसि विश्वात्मन्नराणामन्तरात्मकः}
{त्वन्नाभिकमलोत्पन्नो ब्रह्मा लोकपितामहः} %14-22

\twolineshloka
{अतस्त्वं जगतामीशः सर्वलोकनमस्कृतः}
{त्वं विष्णुर्जानकी लक्ष्मीः शेषोऽयं लक्ष्मणाभिधः} %14-23

\twolineshloka
{आत्मना सृजसीदं त्वमात्मन्येवाऽऽत्ममायया}
{न सज्जसे नभोवत्त्वं चिच्छक्त्या सर्वसाक्षिकः} %14-24

\twolineshloka
{बहिरन्तश्च भूतानां त्वमेव रघुनन्दन}
{पूर्णोऽपि मूढदृष्टीनां विच्छिन्न इव लक्ष्यसे} %14-25

\twolineshloka
{जगत्त्वं जगदाधारस्त्वमेव परिपालकः}
{त्वमेव सर्वभूतानां भोक्ता भोज्यं जगत्पते} %14-26

\twolineshloka
{दृश्यते श्रूयते यद्यत्स्मर्यते वा रघूत्तम}
{त्वमेव सर्वमखिलं त्वद्विनाऽन्यन्न किञ्चन} %14-27

\twolineshloka
{माया सृजति लोकांश्च स्वगुणैरहमादिभिः}
{त्वच्छक्तिप्रेरिता राम तस्मात्त्वय्युपचर्यते} %14-28

\twolineshloka
{यथा चुम्बकसान्निध्याच्चलन्त्येवायसादयः}
{जडास्तथा त्वया दृष्टा माया सृजति वै जगत्} %14-29

\twolineshloka
{देहद्वयमदेहस्य तव विश्वं रिरक्षिषोः}
{विराट् स्थूलं शरीरं ते सूत्रं सूक्ष्ममुदाहृतम्} %14-30

\twolineshloka
{विराजः सम्भवन्त्येते अवताराः सहस्रशः}
{कार्यान्ते प्रविशन्त्येव विराजं रघुनन्दन} %14-31

\twolineshloka
{अवतारकथां लोके ये गायन्ति गृणन्ति च}
{अनन्यमनसो मुक्तिस्तेषामेव रघूत्तम} %14-32

\twolineshloka
{त्वं ब्रह्मणा पुरा भूमेर्भारहाराय राघव}
{प्रार्थितस्तपसा तुष्टस्त्वं जातोऽसि रघोः कुले} %14-33

\twolineshloka
{देवकार्यमशेषेण कृतं ते राम दुष्करम्}
{बहुवर्षसहस्राणि मानुषं देहमाश्रितः} %14-34

\twolineshloka
{कुर्वन् दुष्करकर्माणि लोकद्वयहिताय च}
{पापहारीणि भुवनं यशसा पूरयिष्यसि} %14-35

\twolineshloka
{प्रार्थयामि जगन्नाथ पवित्रं कुरु मे गृहम्}
{स्थित्वाऽद्य भुक्त्वा सबलः श्वो गमिष्यसि पत्तनम्} %14-36

\twolineshloka
{तथेति राघवोऽतिष्ठत्तस्मिन्नाश्रम उत्तमे}
{ससैन्यः पूजितस्तेन सीतया लक्ष्मणेन च} %14-37

\twolineshloka
{ततो रामश्चिन्तयित्वा मुहूर्तं प्राह मारुतिम्}
{इतो गच्छ हनूमंस्त्वमयोध्यां प्रति सत्वरः} %14-38

\twolineshloka
{जानीहि कुशली कच्चिज्जनो नृपतिमन्दिरे}
{शृङ्गवेरपुरं गत्वा ब्रूहि मित्रं गुहं मम} %14-39

\twolineshloka
{जानकीलक्ष्मणोपेतमागतं मां निवेदय}
{नन्दिग्रामं ततो गत्वा भ्रातरं भरतं मम} %14-40

\twolineshloka
{दृष्ट्वा ब्रूहि सभार्यस्य सभ्रातुः कुशलं मम}
{सीतापहरणादीनि रावणस्य वधादिकम्} %14-41

\twolineshloka
{ब्रूहि क्रमेण मे भ्रातुः सर्वं तत्र विचेष्टितम्}
{हत्वा शत्रुगणान् सर्वान् सभार्यः सहलक्ष्मणः} %14-42

\twolineshloka
{उपयाति समृद्धार्थः सह ऋक्षहरीश्वरैः}
{इत्युक्त्वा तत्र वृत्तान्तं भरतस्य विचेष्टितम्} %14-43

\twolineshloka
{सर्वं ज्ञात्वा पुनः शीघ्रमागच्छ मम सन्निधिम्}
{तथेति हनुमांस्तत्र मानुषं वपुरास्थितः} %14-44

\twolineshloka
{नन्दिग्रामं ययौ तूर्णं वायुवेगेन मारुतिः}
{गरुत्मानिव वेगेन जिघृक्षन् भुजगोत्तमम्} %14-45

\twolineshloka
{शृङ्गवेरपुरं प्राप्य गुहमासाद्य मारुतिः}
{उवाचा मधुरं वाक्यं प्रहृष्टेनान्तरात्मना} %14-46

\twolineshloka
{रामो दाशरथिः श्रीमान् सखा ते सह सीतया}
{सलक्ष्मणस्त्वां धर्मात्मा क्षेमी कुशलमब्रवीत्} %14-47

\twolineshloka
{अनुज्ञातोऽद्य मुनिना भरद्वाजेन राघवः}
{आगमिष्यति तं देवं द्रक्ष्यसि त्वं रघूत्तमम्} %14-48

\twolineshloka
{एवमुक्त्वा महातेजाः सम्प्रहृष्टतनूरुहम्}
{उत्पपात महावेगो वायुवेगेन मारुतिः} %14-49

\twolineshloka
{सोऽपश्यद्रामतीर्थं च सरयूं च महानदीम्}
{तामतिक्रम्य हनुमान्नन्दिग्रामं ययौ मुदा} %14-50

\twolineshloka
{क्रोशमात्रे त्वयोध्यायाश्चीरकृष्णाजिनाम्बरम्}
{ददर्श भरतं दीनं कृशमाश्रमवासिनम्} %14-51

\twolineshloka
{मलपङ्कविदिग्धाङ्गं जटिलं वल्कलाम्बरम्}
{फलमूलकृताहारं रामचिन्तापरायणम्} %14-52

\twolineshloka
{पादुके ते पुरस्कृत्य शासयन्तं वसुन्धराम्}
{मन्त्रिभिः पौरमुख्यैश्च काषायाम्बरधारिभिः} %14-53

\twolineshloka
{वृतदेहं मूर्तिमन्तं साक्षाद्धर्ममिव स्थितम्}
{उवाच प्राञ्जलिर्वाक्यं हनूमान्मारुतात्मजः} %14-54

\twolineshloka
{यं त्वं चिन्तयसे रामं तापसं दण्डके स्थितम्}
{अनुशोचसि काकुत्स्थः स त्वां कुशलमब्रवीत्} %14-55

\twolineshloka
{प्रियमाख्यामि ते देव शोकं त्यज सुदारुणम्}
{अस्मिन्मुहूर्ते भ्रात्रा त्वं रामेण सह सङ्गतः} %14-56

\twolineshloka
{समरे रावणं हत्वा रामः सीतामवाप्य च}
{उपयाति समृद्धार्थः ससीतः सहलक्ष्मणः} %14-57

\twolineshloka
{एवमुक्तो महातेजा भरतो हर्षमूर्च्छितः}
{पपात भुवि चास्वस्थः कैकयीप्रियनन्दनः} %14-58

\twolineshloka
{आलिङ्ग्य भरतः शीघ्रं मारुतिं प्रियवादिनम्}
{आनन्दजैरश्रुजलैः सिषेच भरतः कपिम्} %14-59

\twolineshloka
{देवो वा मानुषो वा त्वमनुक्रोशादिहागतः}
{प्रियाख्यानस्य ते सौम्य ददामि ब्रुवतः प्रियम्} %14-60

\twolineshloka
{गवां शतसहस्रं च ग्रामाणां च शतं वरम्}
{सर्वाभरणसम्पन्ना मुग्धाः कन्यास्तु षोडश} %14-61

\twolineshloka
{एवमुक्त्वा पुनः प्राह भरतो मारुतात्मजम्}
{बहूनीमानि वर्षाणि गतस्य सुमहद्वनम्} %14-62

\twolineshloka
{शृणोम्यहं प्रीतिकरं मम नाथस्य कीर्तनम्}
{कल्याणी बत गाथेयं लौकिकी प्रतिभाति मे} %14-63

\twolineshloka
{एति जीवन्तमानन्दो नरं वर्षशतादपि}
{राघवस्य हरीणां च कथमासीत्समागमः} %14-64

\twolineshloka
{तत्त्वमाख्याहि भद्रं ते विश्वसेयं वचस्तव}
{एवमुक्तोऽथ हनुमान् भरतेन महात्मना} %14-65

\twolineshloka
{आचचक्षेऽथ रामस्य चरितं कृत्स्नशः क्रमात्}
{श्रुत्वा तु परमानन्दं भरतो मारुतात्मजात्} %14-66

\twolineshloka
{आज्ञापयच्छत्रुहणं मुदा युक्तं मुदान्वितः}
{दैवतानि च यावन्ति नगरे रघुनन्दन} %14-67

\twolineshloka
{नानोपहारबलिभिः पूजयन्तु महाधियः}
{सूता वैतालिकाश्चैव वन्दिनः स्तुतिपाठकाः} %14-68

\twolineshloka
{वारमुख्याश्च शतशो निर्यान्त्वद्यैव सङ्घशः}
{राजदारास्तथाऽमात्याः सेना हस्त्यश्वपत्तयः} %14-69

\twolineshloka
{ब्राह्मणाश्च तथा पौरा राजानो ये समागताः}
{निर्यान्तु राघवस्याद्य द्रष्टुं शशिनिभाननम्} %14-70

\twolineshloka
{भरतस्य वचः श्रुत्वा शत्रुघ्नपरिचोदिताः}
{अलञ्चक्रुश्च नगरीं मुक्तारत्नमयोज्ज्वलैः} %14-71

\twolineshloka
{तोरणैश्च पताकाभिर्विचित्राभिरनेकधा}
{अलङ्कुर्वन्ति वेश्मानि नानाबलिविचक्षणाः} %14-72

\twolineshloka
{निर्यान्ति वृन्दशः सर्वे रामदर्शनलालसाः}
{हयानां शतसाहस्रं गजानामयुतं तथा} %14-73

\twolineshloka
{रथानां दशसाहस्रं स्वर्णसूत्रविभूषितम्}
{पारमेष्ठीन्युपादाय द्रव्याण्युच्चावचानि च} %14-74

\twolineshloka
{ततस्तु शिबिकारूढा निर्ययू राजयोषितः}
{भरतः पादुके न्यस्य शिरस्येव कृताञ्जलिः} %14-75

\twolineshloka
{शत्रुघ्नसहितो रामं पादचारेण निर्ययौ}
{तदैव दृश्यते दूराद्विमानं चन्द्रसन्निभम्} %14-76

\twolineshloka
{पुष्पकं सूर्यसङ्काशं मनसा ब्रह्मनिर्मितम्}
{एतस्मिन् भ्रातरौ वीरौ वैदेह्या रामलक्ष्मणौ} %14-77

\twolineshloka
{सुग्रीवश्च कपिश्रेष्ठो मन्त्रिभिश्च विभीषणः}
{दृश्यते पश्यत जना इत्याह पवनात्मजः} %14-78

\twolineshloka
{ततो हर्षसमुद्भूतो निःस्वनो दिवमस्पृशत्}
{स्त्रीबालयुववृद्धानां रामोऽयमिति कीर्तनात्} %14-79

\twolineshloka
{रथकुञ्जरवाजिस्था अवतीर्य महीं गताः}
{ददृशुस्ते विमानस्थं जनाः सोममिवाम्बरे} %14-80

\twolineshloka
{प्राञ्जलिर्भरतो भूत्वा प्रहृष्टो राघवोन्मुखः}
{ततो विमानाग्रगतं भरतो राघवं मुदा} %14-81

\twolineshloka
{ववन्दे प्रणतो रामं मेरुस्थमिव भास्करम्}
{ततो रामाभ्यनुज्ञातं विमानमपतद्भुवि} %14-82

\twolineshloka
{आरोपितो विमानं तद्भरतः सानुजस्तदा}
{राममासाद्य मुदितः पुनरेवाभ्यवादयत्} %14-83

\twolineshloka
{समुत्थाय चिराद्दृष्टं भरतं रघुनन्दनः}
{भ्रातरं स्वाङ्कमारोप्य मुदा तं परिषस्वजे} %14-84

\twolineshloka
{ततो लक्ष्मणमासाद्य वैदेहीं नाम कीर्तयन्}
{अभ्यवादयत प्रीतो भरतः प्रेमविह्वलः} %14-85

\twolineshloka
{सुग्रीवं जाम्बवन्तं च युवराजं तथाऽङ्गदम्}
{मैन्दद्विविदनीलांश्च ऋषभं चैव सस्वजे} %14-86

\twolineshloka
{सुषेणं च नलं चैव गवाक्षं गन्धमादनम्}
{शरभं पनसं चैव भरतः परिषस्वजे} %14-87

\twolineshloka
{सर्वे ते मानुषं रूपं कृत्वा भरतमादृताः}
{पप्रच्छुः कुशलं सौम्याः प्रहृष्टाश्च प्लवङ्गमाः} %14-88

\twolineshloka
{ततः सुग्रीवमालिङ्ग्य भरतः प्राह भक्तितः}
{त्वत्सहायेन रामस्य जयोऽभूद्रावणो हतः} %14-89

\twolineshloka
{त्वमस्माकं चतुर्णां तु भ्राता सुग्रीव पञ्चमः}
{शत्रुघ्नश्च तदा राममभिवाद्य सलक्ष्मणम्} %14-90

\twolineshloka
{सीतायाश्चरणौ पश्चाद्ववन्दे विनयान्वितः}
{रामो मातरमासाद्य विवर्णां शोकविह्वलाम्} %14-91

\twolineshloka
{जग्राह प्रणतः पादौ मनो मातुः प्रसादयन्}
{कैकेयीं च सुमित्रां च ननामेतरमातरौ} %14-92

\twolineshloka
{भरतः पादुके ते तु राघवस्य सुपूजिते}
{योजयामास रामस्य पादयोर्भक्तिसंयुतः} %14-93

\twolineshloka
{राज्यमेतन्न्यासभूतं मया निर्यातितं तव}
{अद्य मे सफलं जन्म फलितो मे मनोरथः} %14-94

\twolineshloka
{यत्पश्यामि समायातमयोध्यां त्वामहं प्रभो}
{कोष्ठागारं बलं कोशं कृतं दशगुणं मया} %14-95

\twolineshloka
{त्वत्तेजसा जगन्नाथ पालयस्व पुरं स्वकम्}
{इति ब्रुवाणं भरतं दृष्ट्वा सर्वे कपीश्वराः} %14-96

\twolineshloka
{मुमुचुर्नेत्रजं तोयं प्रशशंसुर्मुदान्विताः}
{ततो रामः प्रहृष्टात्मा भरतं स्वाङ्कगं मुदा} %14-97

\twolineshloka
{ययौ तेन विमानेन भरतस्याश्रमं तदा}
{अवरुह्य तदा रामो विमानाग्र्यान्महीतलम्} %14-98

\twolineshloka
{अब्रवीत्पुष्पकं देवो गच्छ वैश्रवणं वह}
{अनुगच्छानुजानामि कुबेरं धनपालकम्} %14-99

\fourlineindentedshloka
{रामो वसिष्ठस्य गुरोः पदाम्बुजम्}
{नत्वा यथा देवगुरोः शतक्रतुः}
{दत्त्वा महार्हासनमुत्तमं गुरो-}
{रुपाविवेशाथ गुरोः समीपतः} %14-100

\iti{युद्धकाण्डे}{चतुर्दशः}
%%%%%%%%%%%%%%%%%%%%



\sect{पञ्चदशः सर्गः}

\uvacha{श्रीमहादेव उवाच}

\twolineshloka
{ततस्तु कैकयीपुत्रो भरतो भक्तिसंयुतः}
{शिरस्यञ्जलिमाधाय ज्येष्ठं भ्रातरमब्रवीत्} %15-1

\twolineshloka
{माता मे सत्कृता राम दत्तं राज्यं त्वया मम}
{ददामि तत्ते च पुनर्यथा त्वमददा मम} %15-2

\twolineshloka
{इत्युक्त्वा पादयोर्भक्त्या साष्टाङ्गं प्रणिपत्य च}
{बहुधा प्रार्थयामास कैकेय्या गुरुणा सह} %15-3

\twolineshloka
{तथेति प्रतिजग्राह भरताद्राज्यमीश्वरः}
{मायामाश्रित्य सकलां नरचेष्टामुपागतः} %15-4

\twolineshloka
{स्वाराज्यानुभवो यस्य सुखज्ञानैकरूपिणः}
{निरस्तातिशयानन्दरूपिणः परमात्मनः} %15-5

\twolineshloka
{मानुषेण तु राज्येन किं तस्य जगदीशितुः}
{यस्य भ्रूभङ्गमात्रेण त्रिलोकी नश्यति क्षणात्} %15-6

\twolineshloka
{यस्यानुग्रहमात्रेण भवन्त्याखण्डलश्रियः}
{लीलासृष्टमहासृष्टेः कियदेतद्रमापतेः} %15-7

\twolineshloka
{तथाऽपि भजतां नित्यं कामपूरविधित्सया}
{लीलामानुषदेहेन सर्वमप्यनुवर्तते} %15-8

\twolineshloka
{ततः शत्रुघ्नवचनान्निपुणः श्मश्रुकृन्तकः}
{सम्भाराश्चाभिषेकार्थमानीता राघवस्य हि} %15-9

\twolineshloka
{पूर्वं तु भरते स्नाते लक्ष्मणे च महात्मनि}
{सुग्रीवे वानरेन्द्रे च राक्षसेन्द्रे विभीषणे} %15-10

\twolineshloka
{विशोधितजटः स्नातश्चित्रमाल्यानुलेपनः}
{महार्हवसनोपेतस्तस्थौ तत्र श्रिया ज्वलन्} %15-11

\twolineshloka
{प्रतिकर्म च रामस्य लक्ष्मणश्च महामतिः}
{कारयामास भरतः सीताया राजयोषितः} %15-12

\twolineshloka
{महार्हवस्त्राभरणैरलञ्चक्रुः सुमध्यमाम्}
{ततो वानरपत्नीनां सर्वासामेव शोभना} %15-13

\twolineshloka
{अकारयत कौसल्या प्रहृष्टा पुत्रवत्सला}
{ततः स्यन्दनमादाय शत्रुघ्नवचनात्सुधीः} %15-14

\twolineshloka
{सुमन्त्रः सूर्यसङ्काशं योजयित्वाऽग्रतः स्थितः}
{आरुरोह रथं रामः सत्यधर्मपरायणः} %15-15

\twolineshloka
{सुग्रीवो युवराजश्च हनुमांश्च विभीषणः}
{स्नात्वा दिव्याम्बरधरा दिव्याभरणभूषिताः} %15-16

\twolineshloka
{राममन्वीयुरग्रे च रथाश्वगजवाहनाः}
{सुग्रीवपत्न्यः सीता च ययुर्यानैः पुरं महत्} %15-17

\twolineshloka
{वज्रपाणिर्यथा देवैर्हरिताश्वरथे स्थितः}
{प्रययौ रथमास्थाय तथा रामो महत्पुरम्} %15-18

\twolineshloka
{सारथ्यं भरतश्चक्रे रत्नदण्डं महाद्युतिः}
{श्वेतातपत्रं शत्रुघ्नो लक्ष्मणो व्यजनं दधे} %15-19

\twolineshloka
{चामरं च समीपस्थो न्यवीजयदरिन्दमः}
{शशिप्रकाशं त्वपरं जग्राहासुरनायकः} %15-20

\twolineshloka
{दिविजैः सिद्धसङ्घैश्च ऋषिभिर्दिव्यदर्शनैः}
{स्तूयमानस्य रामस्य शुश्रुवे मधुरध्वनिः} %15-21

\twolineshloka
{मानुषं रूपमास्थाय वानरा गजवाहनाः}
{भेरीशङ्खनिनादैश्च मृदङ्गपणवानकैः} %15-22

\twolineshloka
{प्रययौ राघवश्रेष्ठस्तां पुरीं समलङ्कृताम्}
{ददृशुस्ते समायान्तं राघवं पुरवासिनः} %15-23

\fourlineindentedshloka
{दूर्वादलश्यामतनुं महार्ह-}
{किरीटरत्नाभरणाञ्चिताङ्गम्}
{आरक्तकञ्जायतलोचनान्तम्}
{दृष्ट्वा ययुर्मोदमतीव पुण्याः} %15-24

\fourlineindentedshloka
{विचित्ररत्नाञ्चितसूत्रनद्ध-}
{पीताम्बरं पीनभुजान्तरालम्}
{अनर्घ्यमुक्ताफलदिव्यहारैः}
{विरोचमानं रघुनन्दनं प्रजाः} %15-25

\fourlineindentedshloka
{सुग्रीवमुख्यैर्हरिभिः प्रशान्तैः}
{निषेव्यमाणं रवितुल्यभासम्}
{कस्तूरिकाचन्दनलिप्तगात्रम्}
{निवीतकल्पद्रुमपुष्पमालम्} %15-26

\fourlineindentedshloka
{श्रुत्वा स्त्रियो राममुपागतं मुदा}
{प्रहर्षवेगोत्कलिताननश्रियः}
{अपास्य सर्वं गृहकार्यमाहितम्}
{हर्म्याणि चैवारुरुहुः स्वलङ्कृताः} %15-27

\fourlineindentedshloka
{दृष्ट्वा हरिं सर्वदृगुत्सवाकृतिम्}
{पुष्पैः किरन्त्यः स्मितशोभिताननाः}
{दृग्भिः पुनर्नेत्रमनोरसायनम्}
{स्वानन्दमूर्तिं मनसाभिरेभिरे} %15-28

\fourlineindentedshloka
{रामः स्मितस्निग्धदृशा प्रजास्तथा}
{पश्यन् प्रजानाथ इवापरः प्रभुः}
{शनैर्जगामाथ पितुः स्वलङ्कृतम्}
{गृहं महेन्द्रालयसन्निभं हरिः} %15-29

\fourlineindentedshloka
{प्रविश्य वेश्मान्तरसंस्थितो मुदा}
{रामो ववन्दे चरणौ स्वमातुः}
{क्रमेण सर्वाः पितृयोषितः प्रभुः}
{ननाम भक्त्या रघुवंशकेतुः} %15-30

\twolineshloka
{ततो भरतमाहेदं रामः सत्यपराक्रमः}
{सर्वसम्पत्समायुक्तं मम मन्दिरमुत्तमम्} %15-31

\twolineshloka
{मित्राय वानरेन्द्राय सुग्रीवाय प्रदीयताम्}
{सर्वेभ्यः सुखवासार्थं मन्दिराणि प्रकल्पय} %15-32

\twolineshloka
{रामेणैवं समादिष्टो भरतश्च तथाऽकरोत्}
{उवाच च महातेजाः सुग्रीवं राघवानुजः} %15-33

\twolineshloka
{राघवस्याभिषेकार्थं चतुःसिन्धुजलं शुभम्}
{आनेतुं प्रेषयस्वाऽऽशु दूतांस्त्वरितविक्रमान्} %15-34

\twolineshloka
{प्रेषयामास सुग्रीवो जाम्बवन्तं मरुत्सुतम्}
{अङ्गदं च सुषेणं च ते गत्वा वायुवेगतः} %15-35

\twolineshloka
{जलपूर्णान् शातकुम्भकलशांश्च समानयन्}
{आनीतं तीर्थसलिलं शत्रुघ्नो मन्त्रिभिः सह} %15-36

\twolineshloka
{राघवस्याभिषेकार्थं वसिष्ठाय न्यवेदयत्}
{ततस्तु प्रयतो वृद्धो वसिष्ठो ब्राह्मणैः सह} %15-37

\twolineshloka
{रामं रत्नमये पीठे ससीतं सन्न्यवेशयत्}
{वसिष्ठो वामदेवश्च जाबालिर्गौतमस्तथा} %15-38

\twolineshloka
{वाल्मीकिश्च तथा चक्रुः सर्वे रामाभिषेचनम्}
{कुशाग्रतुलसीयुक्तपुण्यगन्धजलैर्मुदा} %15-39

\twolineshloka
{अभ्यषिञ्चन् रघुश्रेष्ठं वासवं वसवो यथा}
{ऋत्विग्भिर्ब्राह्मणैः श्रेष्ठैः कन्याभिः सह मन्त्रिभिः} %15-40

\twolineshloka
{सर्वौषधिरसैश्चैव दैवतैर्नभसि स्थितैः}
{चतुर्भिर्लोकपालैश्च स्तुवद्भिः सगणैस्तथा} %15-41

\twolineshloka
{छत्रं च तस्य जग्राह शत्रुघ्नः पाण्डुरं शुभम्}
{सुग्रीवराक्षसेन्द्रौ तौ दधतुः श्वेतचामरे} %15-42

\twolineshloka
{मालां च काञ्चनीं वायुर्ददौ वासवचोदितः}
{सर्वरत्नसमायुक्तं मणिकाञ्चनभूषितम्} %15-43

\twolineshloka
{ददौ हारं नरेन्द्राय स्वयं शक्रस्तु भक्तितः}
{प्रजगुर्देवगन्धर्वा ननृतुश्चाप्सरोगणाः} %15-44

\twolineshloka
{देवदुन्दुभयो नेदुः पुष्पवृष्टिः पपात खात्}
{नवदूर्वादलश्यामं पद्मपत्रायतेक्षणम्} %15-45

\twolineshloka
{रविकोटिप्रभायुक्तकिरीटेन विराजितम्}
{कोटिकन्दर्पलावण्यं पीताम्बरसमावृतम्} %15-46

\twolineshloka
{दिव्याभरणसम्पन्नं दिव्यचन्दनलेपनम्}
{अयुतादित्यसङ्काशं द्विभुजं रघुनन्दनम्} %15-47

\twolineshloka
{वामभागे समासीनां सीतां काञ्चनसन्निभाम्}
{सर्वाभरणसम्पन्नां वामाङ्के समुपस्थिताम्} %15-48

\twolineshloka
{रक्तोत्पलकराम्भोजां वामेनाऽऽलिङ्ग्य संस्थितम्}
{सर्वातिशयशोभाढ्यं दृष्ट्वा भक्तिसमन्वितः} %15-49

\twolineshloka
{उमया सहितो देवः शङ्करो रघुनन्दनम्}
{सर्वदेवगणैर्युक्तः स्तोतुं समुपचक्रमे} %15-50

\uvacha{श्रीमहादेव उवाच}

\fourlineindentedshloka
{नमोऽस्तु रामाय सशक्तिकाय}
{नीलोत्पलश्यामलकोमलाय}
{किरीटहाराङ्गदभूषणाय}
{सिंहासनस्थाय महाप्रभाय} %15-51

\fourlineindentedshloka
{त्वमादिमध्यान्तविहीन एकः}
{सृजस्यवस्यत्सि च लोकजातम्}
{स्वमायया तेन न लिप्यसे त्वम्}
{यत्स्वे सुखेऽजस्ररतोऽनवद्यः} %15-52

\fourlineindentedshloka
{लीलां विधत्से गुणसंवृतस्त्वम्}
{प्रपन्नभक्तानुविधानहेतोः}
{नानावतारैः सुरमानुषाद्यैः}
{प्रतीयसे ज्ञानिभिरेव नित्यम्} %15-53

\fourlineindentedshloka
{स्वांशेन लोकं सकलं विधाय तम्}
{बिभर्षि च त्वं तदधः फणीश्वरः}
{उपर्यधो भान्वनिलोडुपौषधि-}
{प्रवर्षरूपोऽवसि नैकधा जगत्} %15-54

\fourlineindentedshloka
{त्वमिह देहभृतां शिखिरूपः}
{पचसि भुक्तमशेषमजस्रम्}
{पवनपञ्चकरूपसहायो}
{जगदखण्डमनेन बिभर्षि} %15-55

\fourlineindentedshloka
{चन्द्रसूर्यशिखिमध्यगतं यत्}
{तेज ईश चिदशेषतनूनाम्}
{प्राभवत्तनुभृतामिव धैर्यम्}
{शौर्यमायुरखिलं तव सत्त्वम्} %15-56

\fourlineindentedshloka
{त्वं विरिञ्चिशिवविष्णुविभेदात्}
{कालकर्मशशिसूर्यविभागात्}
{वादिनां पृथगिवेश विभासि}
{ब्रह्म निश्चितमनन्यदिहैकम्} %15-57

\fourlineindentedshloka
{मत्स्यादिरूपेण यथा त्वमेकः}
{श्रुतौ पुराणेषु च लोकसिद्धः}
{तथैव सर्वं सदसद्विभाग-}
{स्त्वमेव नान्यद्भवतो विभाति} %15-58

\fourlineindentedshloka
{यद्यत्समुत्पन्नमनन्तसृष्टा-}
{वुत्पत्स्यते यच्च भवच्च यच्च}
{न दृश्यते स्थावरजङ्गमादौ}
{त्वया विनातःपरतः परस्त्वम्} %15-59

\fourlineindentedshloka
{तत्त्वं न जानन्ति परात्मनस्ते}
{जनाः समस्तास्तव माययातः}
{त्वद्भक्तसेवाऽमलमानसानाम्}
{विभाति तत्त्वं परमेकमैशम्} %15-60

\fourlineindentedshloka
{ब्रह्मादयस्ते न विदुः स्वरूपम्}
{चिदात्मतत्त्वं बहिरर्थभावाः}
{ततो बुधस्त्वामिदमेव रूपम्}
{भक्त्या भजन्मुक्तिमुपैत्यदुःखः} %15-61

\fourlineindentedshloka
{अहं भवन्नाम गृणन् कृतार्थो}
{वसामि काश्यामनिशं भवान्या}
{मुमूर्षमाणस्य विमुक्तयेऽहम्}
{दिशामि मन्त्रं तव राम नाम} %15-62

\fourlineindentedshloka
{इमं स्तवं नित्यमनन्यभक्त्या}
{शृण्वन्ति गायन्ति लिखन्ति ये वै}
{ते सर्वसौख्यं परमं च लब्ध्वा}
{भवत्पदं यान्तु भवत्प्रसादात्} %15-63

\begin{minipage}{\linewidth}
\centering
\uvacha{इन्द्र उवाच}

\fourlineindentedshloka
{रक्षोऽधिपेनाखिलदेव सौख्यम्}
{हृतं च मे ब्रह्मवरेण देव}
{पुनश्च सर्वं भवतः प्रसादात्}
{प्राप्तं हतो राक्षसदुष्टशत्रुः} %15-64
\end{minipage}

\begin{minipage}{\linewidth}
\centering
\uvacha{देवा ऊचुः}

\fourlineindentedshloka
{हृता यज्ञभागा धरादेवदत्ता}
{मुरारे खलेनादिदैत्येन विष्णो}
{हतोऽद्य त्वया नो वितानेषु भागाः}
{पुरावद्भविष्यन्ति युष्मत्प्रसादात्} %15-65
\end{minipage}

\begin{minipage}{\linewidth}
\centering
\uvacha{पितर ऊचुः}

\fourlineindentedshloka
{हतोऽद्य त्वया दुष्टदैत्यो महात्मन्}
{गयादौ नरैर्दत्तपिण्डादिकान्नः}
{बलादत्ति हत्वा गृहीत्वा समस्ता-}
{निदानीं पुनर्लब्धसत्त्वा भवामः} %15-66
\end{minipage}

\begin{minipage}{\linewidth}
\centering
\uvacha{यक्षा ऊचुः}

\fourlineindentedshloka
{सदा विष्टिकर्मण्यनेनाभियुक्ता}
{वहामो दशास्यं बलाद्दुःखयुक्ताः}
{दुरात्मा हतो रावणो राघवेश}
{त्वया ते वयं दुःखजाताद्विमुक्ताः} %15-67
\end{minipage}

\begin{minipage}{\linewidth}
\centering
\uvacha{गन्धर्वा ऊचुः}

\twolineshloka
{वयं सङ्गीतनिपुणा गायन्तस्ते कथामृतम्}
{आनन्दामृतसन्दोहयुक्ताः पूर्णाः स्थिताः पुरा} %15-68
\end{minipage}

\twolineshloka
{पश्चाद्दुरात्मना राम रावणेनाभिविद्रुताः}
{तमेव गायमानाश्च तदाराधनतत्पराः} %15-69

\twolineshloka
{स्थितास्त्वया परित्राता हतोऽयं दुष्टराक्षसः}
{एवं महोरगाः सिद्धाः किन्नरा मरुतस्तथा} %15-70

\twolineshloka
{वसवो मुनयो गावो गुह्यकाश्च पतत्त्रिणः}
{सप्रजापतयश्चैते तथा चाप्सरसां गणाः} %15-71

\twolineshloka
{सर्वे रामं समासाद्य दृष्ट्वा नेत्रमहोत्सवम्}
{स्तुत्वा पृथक् पृथक् सर्वे राघवेणाभिवन्दिताः} %15-72

\twolineshloka
{ययुः स्वं स्वं पदं सर्वे ब्रह्मरुद्रादयस्तथा}
{प्रशंसन्तो मुदा रामं गायन्तस्तस्य चेष्टितम्} %15-73

\twolineshloka
{ध्यायन्तस्त्वभिषेकार्द्रं सीतालक्ष्मणसंयुतम्}
{सिंहासनस्थं राजेन्द्रं ययुः सर्वे हृदि स्थितम्} %15-74

\fourlineindentedshloka
{खे वाद्येषु ध्वनत्सु प्रमुदितहृदयैर्देववृन्दैः स्तुवद्भिः}
{वर्षद्भिःपुष्पवृष्टिं दिवि मुनिनिकरैरीड्यमानः समन्तात्}
{रामः श्यामः प्रसन्नस्मितरुचिरमुखः सूर्यकोटिप्रकाशः}
{सीतासौमित्रिवातात्मजमुनिहरिभिः सेव्यमानो विभाति} %15-75

\iti{युद्धकाण्डे}{पञ्चदशः}
%%%%%%%%%%%%%%%%%%%%



\sect{षोडशः सर्गः}

\uvacha{श्रीमहादेव उवाच}

\twolineshloka
{रामेऽभिषिक्ते राजेन्द्रे सर्वलोकसुखावहे}
{वसुधा सस्यसम्पन्ना फलवन्तो महीरुहाः} %16-1

\twolineshloka
{गन्धहीनानि पुष्पाणि गन्धवन्ति चकाशिरे}
{सहस्रशतमश्वानां धेनूनां च गवां तथा} %16-2

\twolineshloka
{ददौ शतवृषान् पूर्वं द्विजेभ्यो रघुनन्दनः}
{त्रिंशत्कोटिं सुवर्णस्य ब्राह्मणेभ्यो ददौ पुनः} %16-3

\twolineshloka
{वस्त्राभरणरत्नानि ब्राह्मणेभ्यो मुदा तथा}
{सूर्यकान्तिसमप्रख्यां सर्वरत्नमयीं स्रजम्} %16-4

\twolineshloka
{सुग्रीवाय ददौ प्रीत्या राघवो भक्तवत्सलः}
{अङ्गदाय ददौ दिव्ये ह्यङ्गदे रघुनन्दनः} %16-5

\twolineshloka
{चन्द्रकोटिप्रतीकाशं मणिरत्नविभूषितम्}
{सीतायै प्रददौ हारं प्रीत्या रघुकुलोत्तमः} %16-6

\twolineshloka
{अवमुच्यात्मनः कण्ठाद्धारं जनकनन्दिनी}
{अवैक्षत हरीन् सर्वान् भर्तारं च मुहुर्मुहुः} %16-7

\twolineshloka
{रामस्तामाह वैदेहीमिङ्गितज्ञो विलोकयन्}
{वैदेहि यस्य तुष्टाऽसि देहि तस्मै वरानने} %16-8

\twolineshloka
{हनूमते ददौ हारं पश्यतो राघवस्य च}
{तेन हारेण शुशुभे मारुतिर्गौरवेण च} %16-9

\twolineshloka
{रामोऽपि मारुतिं दृष्ट्वा कृताञ्जलिमुपस्थितम्}
{भक्त्या परमया तुष्ट इदं वचनमब्रवीत्} %16-10

\twolineshloka
{हनूमंस्ते प्रसन्नोऽस्मि वरं वरय काङ्क्षितम्}
{दास्यामि देवैरपि यद्दुर्लभं भुवनत्रये} %16-11

\twolineshloka
{हनूमानपि तं प्राह नत्वा रामं प्रहृष्टधीः}
{त्वन्नाम स्मरतो राम न तृप्यति मनो मम} %16-12

\twolineshloka
{अतस्त्वन्नाम सततं स्मरन् स्थास्यामि भूतले}
{यावत्स्थास्यति ते नाम लोके तावत्कलेवरम्} %16-13

\twolineshloka
{मम तिष्ठतु राजेन्द्र वरोऽयं मेऽभिकाङ्क्षितः}
{रामस्तथेति तं प्राह मुक्तस्तिष्ठ यथासुखम्} %16-14

\twolineshloka
{कल्पान्ते मम सायुज्यं प्राप्स्यसे नात्र संशयः}
{तमाह जानकी प्रीता यत्र कुत्रापि मारुते} %16-15

\twolineshloka
{स्थितं त्वामनुयास्यन्ति भोगाः सर्वे ममाऽऽज्ञया}
{इत्युक्तो मारुतिस्ताभ्यामीश्वराभ्यां प्रहृष्टधीः} %16-16

\twolineshloka
{आनन्दाश्रुपरीताक्षो भूयो भूयः प्रणम्य तौ}
{कृच्छ्राद्ययौ तपस्तप्तुं हिमवन्तं महामतिः} %16-17

\twolineshloka
{ततो गुहं समासाद्य रामः प्राञ्जलिमब्रवीत्}
{सखे गच्छ पुरं रम्यं शृङ्गवेरमनुत्तमम्} %16-18

\twolineshloka
{मामेव चिन्तयन्नित्यं भुङ्क्ष्व भोगान्निजार्जितान्}
{अन्ते ममैव सारूप्यं प्राप्स्यसे त्वं न संशयः} %16-19

\twolineshloka
{इत्युक्त्वा प्रददौ तस्मै दिव्यान्याभरणानि च}
{राज्यं च विपुलं दत्त्वा विज्ञानं च ददौ विभुः} %16-20

\twolineshloka
{रामेणाऽऽलिङ्गितो हृष्टो ययौ स्वभवनं गुहः}
{ये चान्ये वानराः श्रेष्ठा अयोध्यां समुपागताः} %16-21

\twolineshloka
{अमूल्याभरणैर्वस्त्रैः पूजयामास राघवः}
{सुग्रीवप्रमुखाः सर्वे वानराः सविभीषणाः} %16-22

\twolineshloka
{यथार्हं पूजितास्तेन रामेण परमात्मना}
{प्रहृष्टमनसः सर्वे जग्मुरेव यथाऽऽगतम्} %16-23

\twolineshloka
{सुग्रीवप्रमुखाः सर्वे किष्किन्धां प्रययुर्मुदा}
{विभीषणस्तु सम्प्राप्य राज्यं निहतकण्टकम्} %16-24

\twolineshloka
{रामेण पूजितः प्रीत्या ययौ लङ्कामनिन्दितः}
{राघवो राज्यमखिलं शशासाखिलवत्सलः} %16-25

\twolineshloka
{अनिच्छन्नपि रामेण यौवराज्येऽभिषेचितः}
{लक्ष्मणः परया भक्त्या रामसेवापरोऽभवत्} %16-26

\twolineshloka
{रामस्तु परमात्माऽपि कर्माध्यक्षोऽपि निर्मलः}
{कर्तृत्वादि विहीनोऽपि निर्विकारोऽपि सर्वदा} %16-27

\twolineshloka
{स्वानन्देनापि तुष्टः सन् लोकानामुपदेशकृत्}
{अश्वमेधादियज्ञैश्च सर्वैर्विपुलदक्षिणैः} %16-28

\twolineshloka
{अयजत्परमानन्दो मानुषं वपुराश्रितः}
{न पर्यदेवन् विधवा न च व्यालकृतं भयम्} %16-29

\twolineshloka
{न व्याधिजं भयं चासीद्रामे राज्यं प्रशासति}
{लोके दस्युभयं नासीदनर्थो नास्ति कश्चन} %16-30

\twolineshloka
{वृद्धेषु सत्सु बालानां नासीन्मृत्युभयं तथा}
{रामपूजापराः सर्वे सर्वे राघवचिन्तकाः} %16-31

\twolineshloka
{ववर्षुर्जलदास्तोयं यथाकालं यथारुचि}
{प्रजाः स्वधर्मनिरता वर्णाश्रमगुणान्विताः} %16-32

\twolineshloka
{औरसानिव रामोऽपि जुगोप पितृवत्प्रजाः}
{सर्वलक्षणसंयुक्तः सर्वधर्मपरायणः} %16-33

{दशवर्षसहस्राणि रामो राज्यमुपास्त सः} %16-34


\fourlineindentedshloka
{इदं रहस्यं धनधान्यऋद्धिम-}
{द्दीर्घायुरारोग्यकरं सुपुण्यदम्}
{पवित्रमाध्यात्मिकसंज्ञितं पुरा}
{रामायणं भाषितमादिशम्भुना} %16-35

\fourlineindentedshloka
{शृणोति भक्त्या मनुजः समाहितो}
{भक्त्या पठेद्वा परितुष्टमानसः}
{सर्वाः समाप्नोति मनोगताशिषो}
{विमुच्यते पातककोटिभिः क्षणात्} %16-36

\fourlineindentedshloka
{रामाभिषेकं प्रयतः शृणोति यो}
{धनाभिलाषी लभते महद्धनम्}
{पुत्राभिलाषी सुतमार्यसम्मतम्}
{प्राप्नोति रामायणमादितः पठन्} %16-37

\fourlineindentedshloka
{शृणोति योऽध्यात्मिकरामसंहिताम्}
{प्राप्नोति राजा भुवमृद्धसम्पदम्}
{शत्रून् विजित्यारिभिरप्रधर्षितो}
{व्यपेतदुःखो विजयी भवेन्नृपः} %16-38

\fourlineindentedshloka
{स्त्रियोऽपि शृण्वन्त्यधिरामसंहिताम्}
{भवन्ति ता जीविसुताश्च पूजिताः}
{वन्ध्याऽपि पुत्रं लभते सुरूपिणम्}
{कथामिमां भक्तियुता शृणोति या} %16-39

\fourlineindentedshloka
{श्रद्धान्वितो यः शृणुयात्पठेन्नरो}
{विजित्य कोपं च तथा विमत्सरः}
{दुर्गाणि सर्वाणि विजित्य निर्भयो}
{भवेत्सुखी राघवभक्तिसंयुतः} %16-40

\fourlineindentedshloka
{सुराः समस्ता अपि यान्ति तुष्टताम्}
{विघ्नाः समस्ता अपयान्ति शृण्वताम्}
{अध्यात्मरामायणमादितो नृणाम्}
{भवन्ति सर्वा अपि सम्पदः पराः} %16-41

\fourlineindentedshloka
{रजस्वला वा यदि रामतत्परा}
{शृणोति रामायणमेतदादितः}
{पुत्रं प्रसूते ऋषभं चिरायुषम्}
{पतिव्रता लोकसुपूजिता भवेत्} %16-42

\twolineshloka
{पूजयित्वा तु ये भक्त्या नमस्कुर्वन्ति नित्यशः}
{सर्वैः पापैर्विनिर्मुक्ता विष्णोर्यान्ति परं पदम्} %16-43

\twolineshloka
{अध्यात्मरामचरितं कृत्स्नं शृण्वन्ति भक्तितः}
{पठन्ति वा स्वयं वक्त्रात्तेषां रामः प्रसीदति} %16-44

\twolineshloka
{राम एव परं ब्रह्म तस्मिंस्तुष्टेऽखिलात्मनि}
{धर्मार्थकाममोक्षाणां यद्यदिच्छति तद्भवेत्} %16-45

\twolineshloka
{श्रोतव्यं नियमेनैतद्रामायणमखण्डितम्}
{आयुष्यमारोग्यकरं कल्पकोट्यघनाशनम्} %16-46

\twolineshloka
{देवाश्च सर्वे तुष्यन्ति ग्रहाः सर्वे महर्षयः}
{रामायणस्य श्रवणे तृप्यन्ति पितरस्तथा} %16-47

\fourlineindentedshloka
{अध्यात्मरामायणमेतदद्भुतम्}
{वैराग्यविज्ञानयुतं पुरातनम्}
{पठन्ति शृण्वन्ति लिखन्ति ये नराः}
{तेषां भवेऽस्मिन्न पुनर्भवो भवेत्} %16-48

\fourlineindentedshloka
{आलोड्याखिलवेदराशिमसकृद्यत्तारकं ब्रह्म तद्-}
{रामो विष्णुरहस्यमूर्तिरिति यो विज्ञाय भूतेश्वरः}
{उद्धृत्याखिलसारसङ्ग्रहमिदं सङ्क्षेपतः प्रस्फुटम्}
{श्रीरामस्य निगूढतत्त्वमखिलं प्राह प्रियायै भवः} %16-49

\iti{युद्धकाण्डे}{षोडशः}
%%%%%%%%%%%%%%%%%%%%

\itikanda{इति श्रीमदध्यात्मरामायणे युद्धकाण्डः समाप्तः॥}